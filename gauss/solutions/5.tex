\begin{definition}
    PDF of normal distribution is given as
    \begin{align}
        \phi_Z\brak{z} = N\brak{\mu,\sigma^2} = \frac{1}{\sqrt{2\pi\sigma^2}}e^\frac{-\brak{z-\mu}^2}{2\sigma^2}
    \end{align}
    \end{definition}
    \begin{corollary}
    \begin{align}
        \phi_X\brak{x} = \frac{1}{\sqrt{2\pi}}e^\frac{-x^2}{2}\label{gauss/5/eq1}\\
        \phi_Y\brak{y} = \frac{1}{\sqrt{2\pi}}e^\frac{-y^2}{2}\label{gauss/5/eq2}
    \end{align}
    \end{corollary}
    \begin{proof}
    Since $\phi\brak{.}$ is the pdf of $N\brak{0,1}$ distribution (given in question),
    $\mu = 0$ and $\sigma^2 = 1$.
    \begin{align*}
        \implies \phi_X\brak{x} = \frac{1}{\sqrt{2\pi}}e^\frac{-x^2}{2}\,\text{and}\,
        \phi_Y\brak{y} = \frac{1}{\sqrt{2\pi}}e^\frac{-y^2}{2}
    \end{align*}
    \end{proof}
    \begin{lemma}
    For circle $C$,
    \begin{align}
        \iint_C{\phi\brak{x}\,\phi\brak{y}\,dx\,dy} = 1 - e^{\frac{-1}{2\pi}}  \label{gauss/5/eq9}
    \end{align}
    \end{lemma}
    \begin{proof}
    $C$ has unit area with centre at origin.
    \begin{align}
        \implies &\pi\times r^2=1\\
        \implies &|r| = \frac{1}{\sqrt{\pi}}
    \end{align}
    For the area inside circle $C$ we get,
    \begin{align}
        x^2+y^2\leq \frac{1}{\pi}
        \implies |y| \leq \sqrt{\frac{1}{\pi}-x^2}\label{gauss/5/eq3}
    \end{align}
    From \eqref{gauss/5/eq3} we get,
    \begin{align}
        \iint_C{\phi\brak{x}\,\phi\brak{y}\,dx\,dy} = \nonumber\\
        \int\limits_{x=\frac{-1}{\sqrt{\pi}}}^{x=\frac{1}{\sqrt{\pi}}}\, \int\limits_{y=-\sqrt{\frac{1}{\pi}-x^2}}^{y=\sqrt{\frac{1}{\pi}-x^2}} \phi\brak{x}\,\phi\brak{y}\,dy\,dx \label{gauss/5/eq4}
    \end{align}
    Using \eqref{gauss/5/eq1} and \eqref{gauss/5/eq2} in \eqref{gauss/5/eq4} we get,
    \begin{align}
        \int\limits_{x=\frac{-1}{\sqrt{\pi}}}^{x=\frac{1}{\sqrt{\pi}}} \int\limits_{y=-\sqrt{\frac{1}{\pi}-x^2}}^{y=\sqrt{\frac{1}{\pi}-x^2}} \frac{1}{2\pi}e^\frac{-\brak{x^2+y^2}}{2}\,dy\,dx
    \end{align}
    Converting it to polar coordinates we get,
    \begin{align}
        &\int\limits_{r=0}^{r=\frac{1}{\sqrt{\pi}}} \int\limits_{\theta=0}^{\theta=2\pi} \frac{1}{2\pi}e^\frac{-r^2}{2}\,r\,d\theta\,dr\\
        &= \int_{0}^{\frac{1}{\sqrt{\pi}}}e^\frac{-r^2}{2}\,r\,dr\\
        &= 1 - e^{\frac{-1}{2\pi}}
    \end{align}
    \end{proof}
    \begin{definition}
    The $Q$ function is defined as
    \begin{align}
        Q\brak{x} = \frac{1}{\sqrt{2\pi}}\int_x^\infty e^{\frac{-u^2}{2}}\, du\label{gauss/5/eq5}
    \end{align}
    \end{definition}
    \begin{lemma}
    \begin{align}
        \int_{-a}^{a} e^{\frac{-x^2}{2}}\, dx = \sqrt{2\pi}\,\brak{1-2Q\brak{a}}\label{gauss/5/eq6}
    \end{align}
    \end{lemma}
    \begin{proof}
    Since $e^{\frac{-x^2}{2}}$ is an even function,
    \begin{align}
        &\int_{-a}^{a} e^{\frac{-x^2}{2}}\, dx = 2\int_{0}^{a} e^{\frac{-x^2}{2}}\, dx\\
        &\implies 2\int_{0}^{a} e^{\frac{-x^2}{2}}\, dx = 2 \brak{\int_{0}^{\infty} e^{\frac{-x^2}{2}}\, dx - \int_{a}^{\infty} e^{\frac{-x^2}{2}}\, dx} \nonumber\\
        &= 2\sqrt{2\pi} \times \frac{1}{\sqrt{2\pi}} \brak{\int_{0}^{\infty} e^{\frac{-x^2}{2}}\, dx - \int_{a}^{\infty} e^{\frac{-x^2}{2}}\, dx} \label{gauss/5/eq7}
    \end{align}
    Comparing \eqref{gauss/5/eq7} with \eqref{gauss/5/eq5} we get,
    \begin{align}
        \int_{-a}^{a} e^{\frac{-x^2}{2}}\, dx &= 2\sqrt{2\pi}\brak{Q\brak{0} - Q\brak{a}}\\
        &= 2\sqrt{2\pi}\brak{\frac{1}{2} - Q\brak{a}}\\
        &= \sqrt{2\pi}\,\brak{1 - 2Q\brak{a}}
    \end{align}
    \end{proof}
    \begin{lemma}
    For square $S$,
    \begin{align}
        \iint_S{\phi\brak{x}\,\phi\brak{y}\,dx\,dy} = \brak{1-2Q\brak{\frac{1}{2}}}^2 \label{gauss/5/eq10}
    \end{align}
    \end{lemma}
    \begin{proof}
    For square $S$ (given in question),
    \begin{align}
        \frac{-1}{2}\leq x \leq \frac{1}{2}\\
        \frac{-1}{2}\leq y \leq \frac{1}{2}
    \end{align}
    From this we get,
    \begin{align}
        \iint_S{\phi\brak{x}\,\phi\brak{y}\,dx\,dy} = \nonumber\\
        \int\limits_{x=\frac{-1}{2}}^{x=\frac{1}{2}} \int\limits_{y=\frac{-1}{2}}^{y=\frac{1}{2}} \phi\brak{x}\,\phi\brak{y}\,dy\,dx \label{gauss/5/eq8}
    \end{align}
    Using \eqref{gauss/5/eq1} and \eqref{gauss/5/eq2} in \eqref{gauss/5/eq8} we get,
    \begin{align}
        \int\limits_{x=\frac{-1}{2}}^{x=\frac{1}{2}} \int\limits_{y=\frac{-1}{2}}^{y=\frac{1}{2}}\frac{1}{2\pi}e^\frac{-\brak{x^2+y^2}}{2}\,dy\,dx
    \end{align}
    Using \eqref{gauss/5/eq6} twice we get,
    \begin{align}
        \int\limits_{x=\frac{-1}{2}}^{x=\frac{1}{2}} \int\limits_{y=\frac{-1}{2}}^{y=\frac{1}{2}}\frac{1}{2\pi}e^\frac{-\brak{x^2+y^2}}{2}\,dy\,dx &= \brak{1-2Q\brak{\frac{1}{2}}}^2
    \end{align}
    \end{proof}
    %
    Calculating the values of \eqref{gauss/5/eq9} and \eqref{gauss/5/eq10} we get,
    \begin{align}
        1 - e^{\frac{-1}{2\pi}} &= 0.147136\\
        \brak{1-2Q\brak{\frac{1}{2}}}^2 &= 0.146631
    \end{align}
    This proves that 
    \begin{align}
        &1 - e^{\frac{-1}{2\pi}} \geq \brak{1-2Q\brak{\frac{1}{2}}}^2\\
        \implies  &\iint_C{\phi\brak{x}\,\phi\brak{y}\,dx\,dy} \geq \iint_S{\phi\brak{x}\,\phi\brak{y}\,dx\,dy}
    \end{align}
    