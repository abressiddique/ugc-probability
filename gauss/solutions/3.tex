\begin{lemma}
Let 
\begin{align}
    \vec{x}=\myvec{X_1\\
             X_2\\
             X_3\\
             X_4\\
             X_5}
             \sim \gauss{\vec{0}}{\vec{I}}
\end{align}
and 
\begin{align}
    \vec{u}=\myvec{1\\
                      1\\
                      1\\
                      1\\
                      1}
\end{align} 
%
Then 
\begin{align}
    \overline{X}&=\frac{1}{5}\vec{u}^{\top}\vec{x}
    \\
   T&=\vec{x}^{\top}\vec{M}\vec{x}
\end{align}
where
\begin{align}
    \vec{M}&=\myvec{\frac{4}{5}&-1/5&-1/5&-1/5&-1/5\\
                   -1/5&\frac{4}{5}&-1/5&-1/5&-1/5\\
                   -1/5&-1/5&\frac{4}{5}&-1/5&-1/5\\
                   -1/5&-1/5&-1/5&\frac{4}{5}&-1/5\\
                   -1/5&-1/5&-1/5&-1/5&\frac{4}{5}}
\end{align}
\end{lemma}
%
\begin{lemma}
\begin{align}
    \vec{M}^2&=\vec{M}
\end{align}
\end{lemma}
\begin{definition}{chi-square distribution}
 Let $X_1,X_2,..X_k$ be i.i.d  standard normal random variables.Define a random variable Y as
 \begin{align}
     Y=X_1^2+X_2^2+....+X_k^2
 \end{align}
 We say Y is chi-square distributed with k degrees of freedom.The mean and variance is given by
 \begin{align}
   \mean{Y}&=k\\
   Var\sbrak{Y}&=2k
 \end{align}
 
\end{definition}
\begin{lemma}
\label{gauss/3/l2.1}
\begin{align}
   \brak{\vec{u}^{\top}\vec{x}}\brak{\vec{x}^{\top}\vec{v}}=\vec{u}^{\top}\brak{\vec{x}\vec{x}^{\top}}\vec{v}
\end{align}
This lemma is verified using python simulation.
\end{lemma}
\begin{theorem}
\label{gauss/3/shit}
If the random variables X and Y are independent,then the random variables $Z=g\brak{X}$ and $W=h(Y)$ are also independent.
\end{theorem}
\begin{proof}
We denote by $A_z$ the set of points such that $g\brak{X}\leq \textit{z}$ and by  $B_w$ the set of points such that $h\brak{Y}\leq \textit{w}$.Clearly,
\begin{align}
    \cbrak{Z\leq \textit{z}}&=\cbrak{X\in A_z}\\
    \cbrak{W\leq \textit{w}}&=\cbrak{Y\in B_w}
\end{align}
Therefore the events $\cbrak{Z\leq \textit{z}}$ and $\cbrak{W\leq \textit{w}}$ are independent because the events $\cbrak{X\in A_z}$ and $\cbrak{Y\in B_w}$ are independent.
\end{proof}
\begin{definition}[cross-covariance]
\begin{align}
    Cov\sbrak{\vec{x},\vec{y}}=\mean{\brak{\vec{x}-\mean{\vec{x}}}\brak{\vec{y}-\mean{\vec{y}}}^{\top}}
\end{align}
\end{definition}
\begin{lemma}
Two jointly normal vectors are independent if and only if their cross-covariance is zero.
\end{lemma}
\begin{theorem}
\label{gauss/3/t2.2}
Let $\vec{x}$ be a $5\times 1$ standard multivariate normal random vector.Let $\vec{B}$ be an $l\times 5$ real matrix.Then the $l\times 1$ random vector $\vec{y}$ defined by $\vec{y}=\vec{B}\vec{x}$ has multivariate normal distribution with mean $\mean{\vec{y}}=\vec{0}$ and covariance matrix $Var\sbrak{\vec{y}}=\vec{B}\vec{B}^{\top}$ 
\end{theorem}
\begin{proof}
The joint moment generating function of $\vec{x}$ is
\begin{align}
    M_{\vec{x}}\brak{\vec{t}}=exp\brak{\vec{t}^{\top}\mu+\frac{1}{2}\vec{t}^{\top}\vec{V}\vec{t}}
\end{align}
since for standard normal distribution $\vec{\mu}=\vec{0}$ and $\vec{V}=\vec{I}$.So
\begin{align}
     M_{\vec{x}}\brak{\vec{t}}=exp\brak{\frac{1}{2}\vec{t}^{\top}\vec{I}\vec{t}}\label{gauss/3/2.11}
\end{align}
Therefore the joint moment generating function of $\vec{y}$ is
\begin{align}
    M_{\vec{y}}\brak{\vec{t}}&=M_{\vec{x}}\brak{\vec{B}^{\top}\vec{t}}\\
    &=exp\brak{\frac{1}{2}\vec{t}^{\top}\vec{B}\vec{B}^{\top}\vec{t}}
\end{align}
on comparing with $\eqref{gauss/3/2.11}$ we can say $\vec{y}$ has multivariate normal distribution. 
\end{proof}
\begin{theorem}
\label{gauss/3/t2.3}
Let $\vec{x}$ be a $5\times 1$ standard multivariate normal random vector.Let $\vec{A}$ ,$\vec{B}$ be two matrices.Define
\begin{align}
    \vec{T_1}&=\vec{A}\vec{x}\\
    \vec{T_2}&=\vec{B}\vec{x}
\end{align}
Then $\vec{T_1}$ and $\vec{T_2}$ are two independent random vectors if and only if $\vec{A}\vec{B}^{\top}=0$
\end{theorem}
\begin{proof}
From theorem $\ref{gauss/3/t2.2}$,$\vec{T_1}$ and $\vec{T_2}$ are jointly normal.Their cross-covariance is
\begin{align}
    Cov\sbrak{\vec{T_1},\vec{T_2}}&=\mean{\brak{\vec{T_1}-\mean{\vec{T_1}}}\brak{\vec{T_2}-\mean{\vec{T_2}}}^{\top}}\\
    &=\mean{\brak{\vec{A}\vec{x}-\mean{\vec{A}\vec{x}}}\brak{\vec{B}\vec{x}-\mean{\vec{B}\vec{x}}}^{\top}}\\
    &=\vec{A}\mean{\brak{\vec{x}-\mean{\vec{x}}}\brak{\vec{x}-\mean{\vec{x}}}^{\top}}\vec{B}^{\top}\\
    &=\vec{A}Var\sbrak{x}\vec{B}^{\top}\\
    &=\vec{A}\vec{B}^{\top}
\end{align}
So $\vec{T_1}$ and $\vec{T_2}$ are independent if and only if $\vec{A}\vec{B}^{\top}=0$
\end{proof}
\begin{theorem}
Let $\overline{X}$ be the sample mean of size 5 from a standard normal distribution.Then
\begin{enumerate}
    \item $\overline{X} \sim N(0,\frac{1}{5})$
    \item $\overline{X}$ and $T$ are independent.
    \item $T \sim \chi_{4}^2$
\end{enumerate} 
where $\chi_{4}^2$  is  chi-square distribution
with   $4$ degrees of freedom and $T$ is defined as
\begin{align}
    T=\sum_{i=1}^{5}(X_i-\overline{X})^2
\end{align}
\end{theorem}
\begin{proof}
\begin{enumerate}
\item 
\begin{align}
    \overline{X}=\frac{1}{5}\vec{u}^{\top}\vec{x}\label{gauss/3/3.1}
\end{align}
From theorem $\ref{gauss/3/t2.2}$ we can say $\overline{X}$ has normal distribution with mean $\mean{\overline{X}}=\vec{0}$ and covariance matrix
\begin{align}
    Var\sbrak{\overline{X}}&=\frac{1}{25}\vec{u}^{\top}\vec{u}\\
    &=\frac{1}{5}
\end{align}
\item since $\vec{M}$ is symmetric and idempotent we have
\begin{align}
   T&=\vec{x}^{\top}\vec{M}\vec{x}\\
    &=\vec{x}^{\top}\vec{M}\vec{M}\vec{x}\\
    &=\vec{x}^{\top}\vec{M}^{\top}\vec{M}\vec{x}\\
    &=\brak{\vec{M}\vec{x}}^{\top}\brak{\vec{M}\vec{x}}\label{gauss/3/3.5} 
\end{align}
\begin{multline}
    Cov\sbrak{\frac{1}{5}\vec{u}^{\top}\vec{x},\vec{M}\vec{x}}
    =E\lbrak{\sbrak{\frac{1}{5}\vec{u}^{\top}\vec{x}-\mean{\frac{1}{5}\vec{u}^{\top}\vec{x}}}}
    \\
\rbrak{\sbrak{\vec{M}\vec{x}-\mean{\vec{M}\vec{x}}}^{\top}}
\end{multline}
Using lemma $\ref{gauss/3/l2.1}$ we get
\begin{multline}
  Cov\sbrak{\frac{1}{5}\vec{u}^{\top}\vec{x},\vec{M}\vec{x}}\\=\frac{1}{5}\vec{u}^{\top}\mean{\brak{\vec{x}-\mean{\vec{x}}}\brak{\vec{x}-\mean{\vec{x}}}^{\top}}\vec{M}^{\top}  
\end{multline}
\begin{align}
    &=\frac{1}{5}\vec{u}^{\top}Var\sbrak{\vec{x}}\vec{M}\\
    &=\frac{1}{5}\vec{u}^{\top}\vec{M}\\
    &=0
\end{align}
So $\vec{M}\vec{x}$ and $\frac{1}{5}\vec{u}^{\top}\vec{x}$ are independent.From theorem $\ref{gauss/3/shit}$(functions of two independent variables are also independent), we can say $\overline{X}$ and T(A function of $\vec{M}\vec{x}$)are independent
\item Since $\vec{M}$ is symmetric it can be expressed as
\begin{align}
    \vec{M}=\vec{P}\vec{D}\vec{P}^{\top}
\end{align}
where $\vec{P}$ is orthogonal and $\vec{D}$ is diagonal.Then
\begin{align}
    T&=\vec{x}^{\top}\vec{M}\vec{x}\\
    &=\vec{x}^{\top}\vec{P}\vec{D}\vec{P}^{\top}\vec{x}\\
    &=\brak{\vec{P}^{\top}\vec{x}}^{\top}\vec{D}\vec{P}^{\top}\vec{x}\\
    &=\vec{y}^{\top}\vec{D}\vec{y}
\end{align}
where $\vec{y}=\vec{P}^{\top}\vec{x}$.Since $\vec{x}$ is standard normal, from theorem $\ref{gauss/3/t2.2}$ we can say $\vec{y}$ is also jointly normal with
\begin{align}
    \mean{\vec{y}}&=0\\
    Var\sbrak{\vec{y}}&=\vec{P}^{\top}\brak{\vec{P}^{\top}}^{\top}\\
    &=\vec{P}^{\top}\vec{P}\\
    &=\vec{I} \hspace{1cm} \brak{\text{Since P is orthogonal}}
\end{align}
So $\vec{y}\sim N\brak{0,\vec{I}}$ is standard normal.
\begin{enumerate}
    \item The eigen values of $\vec{M}$ are $1,1,1,1,0$.So $\vec{D}$ can be written as
    \begin{align}
        \vec{D}=\myvec{1&0&0&0&0\\
                       0&1&0&0&0\\
                       0&0&1&0&0\\
                       0&0&0&1&0\\
                       0&0&0&0&0}
    \end{align}
    Let $\vec{v_1},..,\vec{v_5}$ be corresponding eigen vectors.Then $\vec{P}=\brak{\vec{v_1},..,\vec{v_5}}$.Since $\vec{P}$ is orthogonal the dot product of any two eigen vectors is zero.i.e
    \begin{align}
        \vec{v_i}^{\top}\vec{v_j}=0 \label{gauss/3/ind}
    \end{align}
    for any $i\neq j$
    \begin{align}
     \vec{y}&=\vec{P}^{\top}\vec{x}\\
     \implies 
        \vec{y}&=\myvec{y_1=\vec{v_1}^{\top}\vec{x}\\
                       y_2=\vec{v_2}^{\top}\vec{x}\\
                       y_3=\vec{v_3}^{\top}\vec{x}\\
                       y_4=\vec{v_4}^{\top}\vec{x}\\
                       y_5=\vec{v_5}^{\top}\vec{x}}\\
    \end{align}
    From $\eqref{gauss/3/ind}$ and theorem $\ref{gauss/3/t2.3}$,it follows $y_1,y_2,y_3,y_4,y_5$ are mutually independent.
    \begin{align}
    T&=\vec{y}^{\top}\vec{D}\vec{y}\\
        \implies
         T&=y_1^2+y_2^2+y_3^2+y_4^2 
    \end{align}
    So T is sum of squares of four independent standard normal variables which is chi-square distribution with 4 degrees of freedom.
\end{enumerate}
\end{enumerate}
\end{proof}
\begin{align}
    \mean{T^2\overline{X}^2}=\mean{T^2}\mean{\overline{X}^2}\label{gauss/3/ra}
\end{align}
\begin{align}
    \mean{\overline{X}^2}&=Var\sbrak{\overline{X}}+\brak{\mean{\overline{X}}}^2\\
    &=\frac{1}{5}
\end{align}
 since T is chi-squared distributed with 4 degrees of freedom
 \begin{align}
     \mean{T}&=4\\
     Var\sbrak{T}&=8\\
     \implies \mean{T^2}&=Var\sbrak{T}+\brak{\mean{T}}^2\\
    &=24
 \end{align}
 From $\eqref{gauss/3/ra}$
\begin{align}
    \mean{T^2\overline{X}^2}=4.8
\end{align}