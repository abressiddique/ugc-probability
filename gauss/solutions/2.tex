Given that 
\begin{align}
 \vec{M} = \myvec{ X \\ Y}
 \sim  \gauss{\boldsymbol{\upmu}}{\boldsymbol{\Sigma}}
\end{align}
%
where 
%Here, Mean matrix of X and Y is:
\begin{align}
    \boldsymbol{\upmu} &= \myvec{0 \\ 0}
    \\
\boldsymbol{\Sigma} &= \myvec{
            1 & \rho\\
            \rho & 1 
        }
\end{align}
Also, 
\begin{align}
    X+Y &
    = \vec{A}^\top \vec{M}
    \\
    X-Y &
    =\vec{B}^\top \vec{M}
\end{align}
where
\begin{align}
 \vec{A} &= \myvec{1 \\ 1}, 
    \vec{B} &= \myvec{1 \\ -1}
\end{align}
% Defining Covariance in terms of expectation value:
% \begin{align}
%     Cov(X,Y)=& E[(X-\boldsymbol{\upmu}_x)(Y-\boldsymbol{\upmu}_y)] \\
%     =& E[(X-0)(Y-0)]\\
%     =& E(XY)
% \end{align}
Thus, 
\begin{align}
 Cov(X+Y,X-Y) =& \vec{A}^\top \boldsymbol{\Sigma} \vec{B} \\
    % =& \myvec{1 \\ 1}^\top
    % \myvec{
    %         1 & \rho\\
    %         \rho & 1 
    %     }
    %     \myvec{1 \\ -1}\\ 
    % =& \myvec{1+\rho \\ 1+\rho}^\top
    %     \myvec{1 \\ -1}\\
    % =& (1+\rho)-1(1+\rho) \\
    =&   0
\end{align}
% Note that 
% \begin{align}
% Var(X+Y) = Cov(X+Y , X+Y)\\ 
%  Var(X-Y)  = Cov(X-Y , X-Y)
% \end{align}
% Hence,
% \begin{align}
%     Var(X+Y) =&\vec{A^\top} \boldsymbol{\Sigma} \vec{A} \\
%         =& \myvec{1 \\ 1}^\top
%          \myvec{
%             1 & \rho\\
%             \rho & 1 
%         }
%         \myvec{1 \\ 1}\\
%     =& \myvec{1+\rho \\ 1+\rho}^\top
%         \myvec{1 \\ 1}\\
%     =& 1+\rho+1+\rho \\
%     =& 2+2\rho \neq 0
% \end{align}
% \begin{align}
%     Var(X-Y) =& \vec{B^\top} \boldsymbol{\Sigma} \vec{B} \\
%         =& \myvec{1 \\ -1}^\top
%     \myvec{
%             1 & \rho\\
%             \rho & 1 
%         }
%          \myvec{1 \\ -1}\\
%     =& \myvec{1-\rho \\ \rho-1}^\top
%         \myvec{1 \\ -1}\\
%     =& 1-\rho-\rho+1 \\
%     =& 2-2\rho \neq 0
% \end{align}
% So correlation coefficient is:
% \begin{align}
%     \rho(X+Y,X-Y) = \frac{Cov(X+Y,X-Y)}{\sqrt{var(X+Y) \times var(X-Y)}}
%     = 0
% \end{align}
$\therefore$ X+Y and X-Y are uncorrelated irrespective of value of $\rho$ where $\rho \in \brak{-1,1}$.
%$\therefore$ The correct answer is \textbf{option 4}.