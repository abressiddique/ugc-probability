\begin{definition}
    $\vec{x}\sim\mathcal{N}(\mu_{\vec{x}},\Sigma_{\vec{x}})$ is a bivariate random vector given by
    \begin{align}
        \vec{x}&=\myvec{X_1\\
                       X_2}\\
        \vec{\mu_{\vec{x}}}&=\myvec{\mu_1\\
                                    \mu_2}\\
        \vec{\Sigma_{\vec{x}}}&=\myvec{\sigma_1^2&\rho\sigma_1\sigma_2\\
                                        \rho\sigma_1\sigma_2&\sigma_2^2}
    \end{align}
    where $\vec{\Sigma_{\vec{x}}}$ is covariance matrix of $\vec{x}$ and $\rho$ is correlation of $X_1$ and $X_2$ which is given by
    \begin{align}
        \rho=\frac{\cov{X_1}{,X_2}}{\sigma_1\sigma_2}
    \end{align}
\end{definition}
\begin{lemma}
    \label{gauss/4/stats}
    Let $\vec{x}$ be a $K\times1$ multivariate normal random vector with mean $\vec{\mu}_x$ and covariance matrix $\vec{\Sigma_{\vec{x}}}$. Let $\vec{a}$ be an $L\times1$ real vector and $\vec{B}$ an $L\times K$ full rank real matrix with $K>L$ .Then the $L\times1$ random vector $\vec{y}$ defined by
    \begin{align}
        \vec{y}=\vec{a}+\vec{B}\vec{x}
    \end{align}
    has a multivariate normal distribution with mean
    \begin{align}
        \vec{\mu_\vec{y}}=\vec{a}+\vec{B}\vec{\mu_x}
    \end{align}
    and covariance matrix
    \begin{align}
        \vec{\Sigma_y}=\vec{B}\vec{\Sigma_x}\vec{B}^{\top}
    \end{align}
\end{lemma}
\begin{proof}
   The joint moment generating function of $\vec{x}$ is
   \begin{align}
       M_{\vec{x}}(\vec{t})&=\mathbb{E}[\exp{\brak{\vec{t}^{\top}\vec{x}}}]\\
       \implies M_{\vec{x}}(\vec{t})&=\exp{\brak{\vec{t}^{\top}\vec{\mu_x}+\frac{1}{2}\vec{t}^{\top}\vec{\Sigma_x}\vec{t}}}
   \end{align}
   $\therefore$, the joint moment generating function of $\vec{y}$ is
   \begin{align}
       M_{\vec{y}}(\vec{t})&=\mathbb{E}[\exp{\brak{\vec{t}^{\top}\vec{y}}}]\\
                           &=\mathbb{E}[\exp{\brak{\vec{t}^{\top}\brak{\vec{a}+\vec{B}\vec{x}}}}]\\ &=\exp{\brak{\vec{t}^{\top}\vec{a}}}\mathbb{E}[\brak{\vec{t}^{\top}{\vec{B}\vec{x}}}]\\ &=\exp{\brak{\vec{t}^{\top}\vec{a}}}\mathbb{E}[\brak{\vec{B}^{\top}\vec{t}}^{\top}\vec{x}]\\
                           &=\exp{\brak{\vec{t}^{\top}\vec{a}}}M_{\vec{x}}(\vec{B}^{\top}\vec{t})
        \end{align}
        which can be expressed as
        \begin{multline}
            M_{\vec{y}}(\vec{t})     \\                      =\exp{\brak{\vec{t}^{\top}\vec{a}}}\exp{\brak{(\vec{B}^{\top}\vec{t})^{\top}\vec{\mu_x}+\frac{1}{2}(\vec{B}^{\top}\vec{t})^{\top}\vec{\Sigma_x}\vec{B}^{\top}\vec{t}}}\\
                           =\exp\brak{\vec{t}^{\top}(\vec{a}+\vec{B}\vec{\mu_x})+\frac{1}{2}\vec{t}\vec{B}\vec{\Sigma_x}\vec{B}^{\top}\vec{t}^{\top}}
        \end{multline}
   
   which is the moment generating function of a multivariate normal distribution with mean $\vec{a}+\vec{B}\vec{\mu}$ and covariance matrix $\vec{B}\vec{\Sigma_x}\vec{B}^{\top}$.\\
   
   $\therefore$ $\vec{\mu_\vec{y}}=\vec{a}+\vec{B}\vec{\mu_x}$ and $\vec{\Sigma_y}=\vec{B}\vec{\Sigma_x}\vec{B}^{\top}$
\end{proof}
\begin{theorem}\label{1.1}
   If $(X_1,X_2)$ follows a bivariate distribution then 
   \begin{align}
    \pr{X_1-X_2>\alpha}=\Phi\brak{\frac{-(\alpha+\vec{u}^{\top}\vec{\mu_{\vec{x}})}}{\sqrt{\vec{u}^{\top}\vec{\Sigma_{\vec{x}}}\vec{u}}}}
   \end{align}
\end{theorem}
\begin{proof}
   Let $\vec{u}=\myvec{-1\\
                       1}$ and $\vec{x}=\myvec{X_1\\
                                           X_2}$.  Then 
   \begin{align}
       Y = X_2-X_1=\vec{u}^{\top}\vec{x}
   \end{align}
%    Now consider a random variable Y defined as follows
%    \begin{align}
%        Y&=X_2-X_1\\
%        \implies y&=\vec{u}^{\top}\vec{x}
%    \end{align}
%   then $Y$ has normal distribution with mean
with 
   \begin{align}
       \mu_{y}&=\vec{u}^{\top}\vec{\mu_{\vec{x}}} \\
%    \end{align}
%    and variance is given by
%    \begin{align}
       \sigma^2_{y}&=\vec{u}^{\top}\vec{\Sigma_{\vec{x}}}\brak{\vec{u}^{\top}}^{\top}\\
       \implies \sigma^2_{y}&=\vec{u}^{\top}\vec{\Sigma_{\vec{x}}}\vec{u}
       \\
       \text{or, } Y&\sim\mathcal{N}(\vec{u}^{\top}\vec{\mu_{\vec{x}}},\vec{u}^{\top}\vec{\Sigma_{\vec{x}}}\vec{u})
   \end{align}
   from Lemma     \ref{gauss/4/stats}.  Also, 
   \begin{align}
    Z  = \frac{Y-\mu_y}{\sigma_y} \sim \mathcal{N}(0,1) 
   \end{align}
   Thus, 
   
%    The Standard Normal, often written Z, is a Normal with $\mu=0$ and $\sigma^2=1$. Thus, $Z \sim \mathcal{N}(\mu=0,\sigma^2=1)$\\
%    Now $\pr{X_1-X_2>\alpha}$ can be written as $\pr{Y<-\alpha}$
   \begin{align}
       \pr{Y<-\alpha}&=\pr{\frac{Y-\mu_y}{\sigma_y}<\frac{-\alpha-\mu_y}{\sigma_y}}\\
       &=\pr{Z<\frac{-(\alpha+\vec{u}^{\top}\vec{\mu_{\vec{x}})}}{\sqrt{\vec{u}^{\top}\vec{\Sigma_{\vec{x}}}\vec{u}}}}
    \end{align}
    \begin{align}
       \implies  \pr{X_1-X_2>\alpha}=\Phi\brak{\frac{-(\alpha+\vec{u}^{\top}\vec{\mu_{\vec{x}})}}{\sqrt{\vec{u}^{\top}\vec{\Sigma_{\vec{x}}}\vec{u}}}}
       \label{gauss/4/temp}
   \end{align}
\end{proof}
   From the given information,
   \begin{align}
    \mu_1=\mu_2&=0\\
    \sigma^2_1=\sigma^2_2&=2\\
    Cov(X_1,X_2)&=-1\\
    \rho=\frac{Cov(X_1,X_2)}{\sigma_1\sigma_2}&=\frac{-1}{2}\\
    \vec{\mu_{\vec{x}}}=\myvec{\mu_1\\
                                    \mu_2}&=\myvec{0\\
                                                  0}\\
     \vec{\Sigma_{\vec{x}}}=\myvec{\sigma_1^2&\rho\sigma_1\sigma_2\\
                                        \rho\sigma_1\sigma_2&\sigma_2^2}&=\myvec{2&-1\\
                                                                                -1&2}
\end{align}
which, upon substituting in        \eqref{gauss/4/temp} yields 
\begin{align}
    \implies \pr{X_1-X_2>6}&=\Phi\brak{\frac{-6}{\sqrt{6}}}\\
    &=\Phi(-\sqrt{6})
\end{align}
%    \begin{lemma}
%        \begin{align}
%            \mu_{y}&=\vec{u}^{\top}\vec{\mu_{\vec{x}}}\\
%            \implies \mu_{\vec{y}}&=\myvec{-1\,\,1}\myvec{\mu_1\\
%                                                      \mu_2}\\
%            \implies \mu_{\vec{y}}&=\myvec{\mu_2-\mu_1}
%        \end{align}
%    \end{lemma}
%    \begin{lemma}
%        \begin{align}
%        \sigma^2_{y}&=\vec{u}^{\top}\vec{\Sigma_{\vec{x}}}\vec{u}\\
%        \implies \sigma^2_{y}&=\myvec{-1\,\,1}\myvec{\sigma^2_1&\rho\sigma_1\sigma_2\\
%                                                         \rho\sigma_1\sigma_2&\sigma^2_2}\myvec{-1\\
%                                                                                                1}\\
%        \implies \sigma^2_{y}&=\myvec{\sigma^2_1+\sigma^2_2-2\rho\sigma_1\sigma_2}
%        \end{align}
%    \end{lemma}
%    \begin{align}
%        \therefore \pr{X_1-X_2>\alpha}&=\Phi\brak{\frac{(\mu_1-\mu_2)-\alpha}{\sqrt{\sigma^2_1+\sigma^2_2-2\rho\sigma_1\sigma_2}}}
%    \end{align}
 %\end{proof}
% Given $(X_1,X_2)$ follow a bivariate normal distribution with
% \begin{align}
%    \mu_1=\mu_2&=0\\
%    \sigma^2_1=\sigma^2_2&=2\\
%    Cov(X_1,X_2)&=-1\\
%    \rho=\frac{Cov(X_1,X_2)}{\sigma_1\sigma_2}&=\frac{-1}{2}
% \end{align}
% for $\pr{X_1-X_2>6}$ we can use the above theorem as follows
% \begin{align}
%    \pr{X_1-X_2>\alpha}&=\Phi\brak{\frac{(\mu_1-\mu_2)-\alpha}{\sqrt{\sigma^2_1+\sigma^2_2-2\rho\sigma_1\sigma_2}}}\\
%    \implies \pr{X_1-X_2>6}&=\Phi\brak{\frac{(0-0)-6}{\sqrt{2+2-2\brak{\frac{-1}{2}}2}}}\\
%    \implies \pr{X_1-X_2>6}&=\Phi\brak{\frac{-6}{\sqrt{6}}}\\
%    \therefore \pr{X_1-X_2>6}&=\Phi(-\sqrt{6})
% \end{align}