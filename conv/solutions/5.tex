\begin{definition}[Convergence in Distribution or Weak convergence]
    For any given sequence of real-valued random variables $X_1,X_2,X_3, \dots ,X_n$ and a random variable X,
    \begin{equation}
    \lim_{n \rightarrow \infty} F_{X_n}(x) = F_X(x)
\end{equation}
where $F_{X_n}$ and $F_X$ are the cumulative probability distribution functions of $X_n$ and X respectively.
\end{definition}
\begin{definition}[Convergence in Probability]
    \begin{align}
    \lim_{n \rightarrow \infty} \pr{|X_n - X| > \epsilon} = 0 \forall \epsilon > 0
\end{align}
\end{definition}
This is stronger than the convergence in distribution but weaker than Almost sure convergence.
\begin{definition}[Almost sure Convergence]
    \begin{align}
  \pr{\lim_{n \rightarrow \infty} X_n = X} = 1  
\end{align}
\end{definition}
This type of convergence is stronger than both convergence in distribution and probability.
\begin{itemize}
    \item In general, stronger statements imply weaker statements but not vice versa, i.e. Convergence in probability implies convergence in distribution and Almost sure convergence implies convergence in probability.
    \item We shall use the following statement from Portmanteau's Lemma in the following proof:
    \end{itemize}
    \begin{lemma}[Portmanteau's Lemma]\label{conv/5/Portmanteau's Lemma}
             The sequence $X_1,X_2,X_3, \dots ,X_n$ converges in distribution to X if and only if \begin{equation}
                  \limsup \pr{X_{n}\in F} \leq \pr{X \in F}
             \end{equation} for every closed set F;
        \end{lemma}
\begin{lemma} Convergence in distribution implies convergence in probability if X is a constant.\end{lemma}
    \textbf{Proof}:
    \begin{itemize}
        \item Let $\epsilon > 0$. Let X = c and the sequence $X_1,X_2,X_3, \dots ,X_n$ converges to X in distribution.
        \item Let \begin{align}
            &S = \{X : |X - c| > \epsilon\} \\
            \implies &\pr{|X_n - c|>\epsilon} = \pr{X_ n\in S}
        \end{align}
        \item From Lemma \ref{conv/5/Portmanteau's Lemma}, 
        \begin{align}
           &\because \lim_{n \rightarrow \infty}X_n = c \\
            \implies &\limsup_{n \rightarrow \infty} \pr{X_n \in S} \leq  \pr{c \in S} \\
            &\because \pr{c \in S} = 0 \text{ (By defn)} \\
            \implies &\limsup_{n \rightarrow \infty} \pr{X_n \in S} \leq 0 \\
            \implies &\lim_{n \rightarrow \infty} \pr{X_n \in S} \leq 0\\
            \implies &\lim_{n \rightarrow \infty} \pr{X_n \in S} = 0 \text{ (Probability $\geq 0$)}
        \end{align}
        
        \item Thus, by definition,
        \begin{multline}
             \pr{|X_n-c|> \epsilon} = 0 \text{ for any }\epsilon > 0 \text{ given,} \\    
            \lim_{n \rightarrow \infty} F_{X_n}(x) = F_X(x)\text{ and X is constant}
        \end{multline}
    \end{itemize}
Let us look at each option one after another.
\begin{enumerate}
    \item Given,
    \begin{align}\nonumber
        p = 0 \implies X = 1
    \end{align}
    Since X is a constant, from Lemma \ref{conv/5/1}, we can say that option 1 is true.
    \item Given,
      \begin{align}\nonumber
        p = 1 \implies X = 0
    \end{align}
    Since X is a constant, from Lemma \ref{conv/5/1}, we can say that option 2 is true.
    \item Given,
      \begin{align}\nonumber
        0 < p < 1 \implies X \neq 0,1 
    \end{align}
    Since X is not a constant, we can say that option 3 is false.
    \item Since Convergence in probability is weaker than Almost sure convergence, we can say that option 4 is false as a weaker statement does not imply a stronger statement.
\end{enumerate}
Therefore, the true statements from the options are options 1 and 2.
