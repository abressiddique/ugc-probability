See Tables \ref{eq:solutions/2014/dec/78/table:1}
and \ref{eq:solutions/2014/dec/78/table:2}

\onecolumn
\begin{longtable}{|l|l|}
\hline
\multirow{3}{*}{Given} & \\
& $\vec{A}$ is 4 x 7 real matrix\\
& $\vec{B}$ is 7 x 4 real matrix\\
& $\vec{AB} =\vec{I}_4 $\\
&\\
\hline
\multirow{3}{*}{Option-1} & \\
& since $\vec{I}_4$ is a 4 x 4 identity matrix, $rank(\vec{I}_4) = 4 = rank(\vec{AB})$\\
& \\
& from the properties of matrices\\
& $rank(\vec{A}) \leq min\lbrace \# cloumns, \#rows \rbrace$\\
& $rank(\vec{A}) \leq 4$\\
& \\
& and\\
& \\
& $rank(\vec{AB}) \leq rank(\vec{A})$\\
& $ 4 \leq rank(\vec{A})$\\
& \\
& $\therefore rank(\vec{A}) = 4$\\
& Hence Option-1 is True. \\
& \\
\hline
\multirow{3}{*}{Option-2} & \\
& Similarly from the properties of matrices\\
& $rank(\vec{B}) \leq min\lbrace \# cloumns, \#rows \rbrace$\\
& $rank(\vec{B}) \leq 4$\\
& \\
& and\\
& \\
& $rank(\vec{AB}) \leq rank(\vec{B})$\\
& $ 4 \leq rank(\vec{B})$\\
& \\
& $\therefore rank(\vec{B}) = 4$\\
& Hence Option-2 is False. \\
& \\
\hline
\multirow{3}{*}{Option-3} & \\
& Since $rank(\vec{B}) = 4$, and $\vec{B}$ is a 7 x 4 matrix in \\
& finite dimensional vector space $\mathbb{V}$.\\
& the column space,$C(\vec{B})$ will form the basis.\\
& $\implies range(\vec{B}) = dim(\mathbb{V}) = 4$\\
& \\
& from rank-nullity theorem\\
& $ rank(\vec{B}) + nullity(\vec{B}) = dim(\mathbb{V})$\\
& by substituting above values\\
& $ nullity(\vec{B}) = 0$\\
& Hence Option-3 is True.\\
& \\
\hline
\multirow{3}{*}{Option-4} & \\
& Given $\vec{BA} = \vec{I}_7$\\
& $rank(\vec{I}_7) = 7 = rank(\vec{BA})$\\
& \\
& from the properties of matrices\\
& $rank(\vec{BA}) \leq rank(\vec{B})$\\
& $7 \leq rank(\vec{B})$\\
& the above conditioned can not be satisfied since we know\\
& $rank(\vec{B}) =4$.\\
& Hence Option-4 is False. \\
&\\
\hline
\multirow{3}{*}{Conclusion} & \\
& Option-1 and 3 are True\\
& Option-2 and 4 are False\\
&\\
\hline
\caption{Proof}
\label{eq:solutions/2014/dec/78/table:1}
\end{longtable}
\begin{longtable}{|l|l|}
\hline
\multirow{3}{*}{Example} & \\
& Proving the above results with example in lower dimensions as follows.\\
& Let $\vec{A}$ be a 2 x 3 matrix in vector space $\mathbb{V}$ and\\
& consider $\vec{A} = \myvec{1&2&0\\0&2&-4}$\\
& and $\vec{B}$ be a 3 x 2 matrix in vector space $\mathbb{V}$ and\\
& consider $\vec{B} = \myvec{1&0\\0&0\\0&-\frac{1}{4}}$\\
& so that $\vec{AB} = \vec{I} = \myvec{1&0\\0&1}$ is a 2 x 2 matrix\\
& \\
\hline
\multirow{3}{*}{Option-1} & \\
& row reduced echelon form of $\vec{A}$ is\\
& $rref(\vec{A}) = \myvec{1&0&4\\0&1&-2}$\\
& $\implies rank(\vec{A}) = 2$\\
& Hence Option-1 is True\\
&\\
\hline
\multirow{3}{*}{Option-2} & \\
& row reduced echelon form of $\vec{B}$ is\\
& $rref(\vec{B}) = \myvec{1&0\\0&1\\0&0}$\\
& $\implies rank(\vec{B}) = 2$\\
& Hence Option-2 is False\\
&\\
\hline
\multirow{3}{*}{Option-3} & \\
& from the above rref form of $\vec{B}$\\
& the $range(\vec{B}) = \myvec{1&0\\0&0\\0&-\frac{1}{4}}$\\
& $\implies dim(\mathbb{V}) = 2$\\
& $nullspace(\vec{B}) = \myvec{0\\0}$\\
& $\therefore$ from rank-nullity theorem\\
& $nullity(\vec{B}) = 0$\\
& Hence Option-3 is True\\
&\\
\hline
\multirow{3}{*}{Option-4} & \\
& $\vec{BA} = \myvec{1&2&0\\0&0&0\\0&-\frac{1}{2}&1}$\\
& $\implies \vec{BA} \neq \vec{I}$\\
& $rank(\vec{BA}) = \vec{I} = 2$\\
& Hence Option-4 is False\\
&\\
\hline
\caption{Example}
\label{eq:solutions/2014/dec/78/table:2}
\end{longtable}
\twocolumn
