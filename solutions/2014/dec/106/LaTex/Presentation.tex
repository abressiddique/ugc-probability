\documentclass{beamer}
\usepackage{listings}
\lstset{
%language=C,
frame=single, 
breaklines=true,
columns=fullflexible
}
\usepackage{subcaption}
\usepackage{url}
\usepackage{tikz}
\usepackage{tkz-euclide} % loads  TikZ and tkz-base
%\usetkzobj{all}
\usetikzlibrary{calc,math}
\usepackage{float}
\newcommand\norm[1]{\left\lVert#1\right\rVert}
\renewcommand{\vec}[1]{\mathbf{#1}}
\usepackage[export]{adjustbox}
\usepackage[utf8]{inputenc}
\usepackage{amsmath}
\usetheme{Boadilla}
\providecommand{\pr}[1]{\ensuremath{\Pr\left(#1\right)}}
\usepackage{mathtools}

\title{CSIR UGC NET DECEMBER 2014 Q.106}
\author{Nagubandi Krishna Sai \\ MS20BTECH11014}
\date{March 9th 2021}
\begin{document}

\begin{frame}
\titlepage
\end{frame}

\begin{frame}
\frametitle{Question}

\begin{block}{}
Consider a Markov chain with state space \{1,2,....,100\}. Suppose states 2i and 2j communicate with each other and states 2i-1 and 2j-1 communicate with each other for every i,j = 1,2,...,50. Further suppose that $p^{(2)}_{3,3}$ $>$ 0,$p^{(3)}_{4,4}$ $>$ 0 and $p^{(7)}_{2,5}$ $>$ 0. Then 
\begin{enumerate}
\item The Markov chain is irreducible.
\item The Markov chain is aperiodic.
\item State 8 is recurrent.
\item State 9 is recurrent.
\end{enumerate}
\end{block}
\end{frame}
\begin{frame}{}
\begin{block}{Prerequisites}
According to the question, all even and odd positioned states communicate with each other.
\end{block}
\begin{block}{Definition}
\begin{center}
    p^{(n)}_{i,j} > 0 ; n \ge 0 \\ 
    p^{n}_{i,j} = Pr[{X_t =j | X_{t-1} = i}]; \forall n \ge 0. \\
\end{center}
Where X is the collection of random variables and \textbf{index} t represents time.
X(t) represents the \textbf{state} of the process at time t.
This is the probability that the chain moves from state i to state j in exactly m steps.
\end{block}
\end{frame}
\begin{frame}{}
\begin{block}{Note}
If $p^{(n)}_{i,j}$ $>$ 0, for some n, then we say that the state j is accessible from state i.
\end{block}
\end{frame}
\begin{frame}{}
\begin{block}{Definition 1}
We say that \textbf{Markov\ chain} is \textbf{irreducible} if and only if all states belong to one communication class and all states communicate with each other.
\end{block}
\begin{block}{Definition 2}
In an \textbf{irreducible\ chain} all states belong to a single communicating class. This means that, if one of the states in an irreducible Markov chain is \textbf{aperiodic}. Then, all the remaining states are also aperiodic. 
\end{block}
\begin{block}{Condition}
\begin{center}
    d(K) = gcd\{m \ge 1: P{m}_{k,k} > 0\} \\
\end{center} 
If d(K) = 1, then the state K is aperiodic. \\
If d(K) = 0, then the state K is periodic.    
\end{block}
\end{frame}
\begin{frame}{Solution}
\begin{center}
Consider, S = \{1,2,3,....,100\}
\end{center}
\begin{center}
\begin{tikzpicture}
\filldraw[color=red!60, fill=red!5, very thick](1.5,0) circle (0.4);
\filldraw[color=red!60, fill=red!5, very thick](0,1.5) circle (0.4);
\filldraw[color=red!60, fill=red!5, very thick](-1.5,0) circle (0.4);
\filldraw[color=red!60, fill=red!5, very thick](0,-1.5) circle (0.4);
\node[roundnode] at (1.5,0) {3};
\node[roundnode] at (0,1.5) {1};
\node[roundnode] at (-1.5,0) {49};
\node[roundnode] at (0,-1.5) {5};
\node (A) at (0, 1.5) {};
\node (B) at (1.5, 0) {};
\draw[->, to path={-\ (\tikztotarget)}]
  (A) edge (B) ;
\node (A) at (0, 1.5) {};
\node (B) at (-1.5, 0) {};
\draw[->, to path={-\ (\tikztotarget)}]
  (A) edge (B) ;
\node (A) at (0,-1.5) {};
\node (B) at (1.5, 0) {};
\draw[->, to path={-\ (\tikztotarget)}]
  (A) edge (B) ;
\node (A) at (0,-1.5) {};
\node (B) at (-1.5, 0) {};
\draw[->, to path={-\ (\tikztotarget)}]
  (A) edge (B) ;
\node (A) at (1.5,0) {};
\node (B) at (0,-1.5) {};
\draw[->, to path={-\ (\tikztotarget)}]
  (A) edge (B) ;
\node (A) at (1.5,0) {};
\node (B) at (0,1.5) {};
\draw[->, to path={-\ (\tikztotarget)}]
  (A) edge (B) ;
\node (A) at (-1.5,0) {};
\node (B) at (0,1.5) {};
\draw[->, to path={-\ (\tikztotarget)}]
  (A) edge (B) ;
\node (A) at (-1.5,0) {};
\node (B) at (0,-1.5) {};
\draw[->, to path={-\ (\tikztotarget)}]
  (A) edge (B) ;
\node (A) at (1.5,0) {};
\node (B) at (-1.5,0) {};
\draw[->, to path={-\ (\tikztotarget)}]
  (A) edge (B) ;
\node (A) at (-1.5,0) {};
\node (B) at (1.5,0) {};
\draw[->, to path={-\ (\tikztotarget)}]
  (A) edge (B) ;
\draw[green,ultra thick,dashed] (0,-1.7) -- (-1.8,0);
\filldraw[color=red!60, fill=red!5, very thick](3,0) circle (0.4);
\filldraw[color=red!60, fill=red!5, very thick](4.5,1.5) circle (0.4);
\filldraw[color=red!60, fill=red!5, very thick](6,0) circle (0.4);
\filldraw[color=red!60, fill=red!5, very thick](4.5,-1.5) circle (0.4);
\node[roundnode] at (3,0) {50};
\node[roundnode] at (4.5,1.5) {2};
\node[roundnode] at (6,0) {4};
\node[roundnode] at (4.5,-1.5) {6};
\node (A) at (4.5, 1.5) {};
\node (B) at (3, 0) {};
\draw[->, to path={-\ (\tikztotarget)}]
  (A) edge (B) ;
\node (A) at (4.5, 1.5) {};
\node (B) at (6, 0) {};
\draw[->, to path={-\ (\tikztotarget)}]
  (A) edge (B) ;
\node (A) at (4.5,-1.5) {};
\node (B) at (3, 0) {};
\draw[->, to path={-\ (\tikztotarget)}]
  (A) edge (B) ;
\node (A) at (4.5,-1.5) {};
\node (B) at (6, 0) {};
\draw[->, to path={-\ (\tikztotarget)}]
  (A) edge (B) ;
\node (A) at (3,0) {};
\node (B) at (4.5,-1.5) {};
\draw[->, to path={-\ (\tikztotarget)}]
  (A) edge (B) ;
\node (A) at (3,0) {};
\node (B) at (4.5,1.5) {};
\draw[->, to path={-\ (\tikztotarget)}]
  (A) edge (B) ;
\node (A) at (6,0) {};
\node (B) at (4.5,1.5) {};
\draw[->, to path={-\ (\tikztotarget)}]
  (A) edge (B) ;
\node (A) at (6,0) {};
\node (B) at (4.5,-1.5) {};
\draw[->, to path={-\ (\tikztotarget)}]
  (A) edge (B) ;
\node (A) at (3,0) {};
\node (B) at (6,0) {};
\draw[->, to path={-\ (\tikztotarget)}]
  (A) edge (B) ;
\node (A) at (-1.5,0) {};
\node (B) at (1.5,0) {};
\draw[->, to path={-\ (\tikztotarget)}]
  (A) edge (B) ;
\draw[green,ultra thick,dashed] (4.5,-1.6) -- (2.8,0);
\node (A) at (4.5,1.7) {};
\node (B) at (0,-1.5) {};
\draw[->, to path={-\ (\tikztotarget)}]
  (A) edge (B) ;
\filldraw[color=red!60, fill=red!5, very thick](0,-3) circle (0.4);
\filldraw[color=red!60, fill=red!5, very thick](1.5,-3) circle (0.4);
\filldraw[color=red!60, fill=red!5, very thick](3,-3) circle (0.4);
\filldraw[color=red!60, fill=red!5, very thick](6,-3) circle (0.4);
\node[roundnode] at (0,-3) {51};
\node[roundnode] at (1.5,-3) {52};
\node[roundnode] at (3,-3) {53};
\node[roundnode] at (6,-3) {100};
\draw[green,ultra thick,dashed] (3.5,-3.2) -- (5.5,-3.2);
\end{tikzpicture}
\end{center}
\end{frame}
\begin{frame}{Solution}
\begin{block}{Classification of states}
\begin{table}[h!]
\resizebox{\columnwidth}{!}
{ 
\begin{tabular}{|c|c|}
\hline
Communication class & set of elements  \\  \hline
C_1(1) & \{1,3,5,7,....,49\}  \\  \hline
C_1(2) & \{2,4,6,8,....,50\}  \\  \hline
C_1(51) & \{51\}  \\  \hline
C_1(52) & \{52\}  \\  \hline
\vdots & \vdots  \\  \hline
C_1(100) & \{100\}  \\  \hline
\end{tabular}
}
\caption{Communication class} 
\label{table}
\end{table}
\end{block}
\end{frame}
\begin{frame}{Solution}
\begin{block}
$\therefore$ As there are 52 communication classes, the given Markov chain is reducible. 
\end{block}
\begin{block}{Periodicity of states}
\begin{table}[h!]
\resizebox{\columnwidth}{!}
{ 
\begin{tabular}{|c|c|}
\hline
Periodicity of states & set of states  \\  \hline
d(1) & \{1,3,5,7,....,49\}  \\  \hline
d(2) & \{2,4,6,8,....,50\}  \\  \hline
\end{tabular}
}
\caption{Periodicity of some of elements of set S} 
\label{table} 
\end{table}
\end{block}
\end{frame}
\begin{frame}{Solution}
\begin{block}{Periodicity}
\begin{center}
    d(1) = d(2) = gcd\{2,3,....,50\} = 1.  \\
\end{center}
$\therefore$ The states 1 to 50 are \textbf{Aperiodic}.
\end{block}
\begin{block}{Periodicity}
\begin{center}
    d(51) = d(52) = ....... = d(100) = 0.  \\
\end{center}
$\therefore$ The states 51 to 100 are \textbf{Periodic}.
\end{block}
\end{frame}
\begin{frame}{Solution}
\begin{block}{Explanation}
Hence, by the above figure, states from 1 to 50 are recurrent. Because the states 5 is recurrent, as it is given in question $p^{(7)}_{2,5}$ $>$ 0, this means state 5 is accessible from state 2. This means state 2 after communicating with even states, returns to state 5, and with state 5 all the odd states return to their states again. 
\end{block}
\begin{block}{Explanation}
From above, we can concluded that \{1,3,5,7,...,49,51,52,53,....,100\} are recurrent states. And \{2,4,6,8,....,50\} are transient states.
\end{block}
\end{frame}
\begin{frame}{Conclusion}
\begin{block}{Option 1}
As there are more than two communicating classes, The given \textbf{Markov\ chain} is \textbf{reducible}. \\
\begin{center}
  $\therefore$ \boxed{\text{\textbf{Option 1} is a \textbf{incorrect} answer}}
\end{center}
\end{block}
\begin{block}{Option 2}
As there are communicating classes of aperiodic and periodic, The given \textbf{Markov\ chain} is not a \textbf{aperiodic} chain. \\
\begin{center}
  $\therefore$ \boxed{\text{\textbf{Option 2} is a \textbf{incorrect} answer}}
\end{center}
\end{block}
\end{frame}
\begin{frame}{Conclusion}
\begin{block}{Option 3}
After accessing all states, from state 2, the process continues with state 5, while not returning to state 2 again. This means, state 8 is a transient state. \\
\begin{center}
  $\therefore$ \boxed{\text{\textbf{Option 3} is a \textbf{incorrect} answer}}
\end{center}
\end{block}
\begin{block}{Option 4}
As said above, state 5 re-occurs continuously. So, state 9 is a recurrent state. \\
\begin{center}
  $\therefore$ \boxed{\text{\textbf{Option 4} is a \textbf{correct} answer}}
\end{center}
\end{block}
\end{frame}
\end{document}