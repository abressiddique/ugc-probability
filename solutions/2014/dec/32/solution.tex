See Table \ref{eq:solutions/2014/dec/32/table1}

\begin{table*}[ht!]
\begin{center}
\begin{tabular}{|c|c|}
\hline
\textbf{Options} & \textbf{Explanation} \\
\hline
\text{REAL} & \\
Counter Example & $\vec{P}=\myvec{i&0\\0&i}$\\
&\\
&$\vec{P}^\theta=\myvec{-i&0\\0&-i}$\\
&\\
&$\vec{P}^\theta\vec{P}=\myvec{1&0\\0&1}=\vec{I}$\\
& Eigen values of $\vec{P}$ are $i,i$ which are not real\\
& Hence,incorrect.\\
\hline
\text{Complex Conjugates of each other.} & 
From above,$(i,i)$ are not complex conjugate\\& of each other\\
&\\
&Hence,incorrect.\\
\hline
\text{Reciprocals of each other}
& Reciprocal of $i=\frac{1}{i}=\frac{i^4}{i}=i^3\neq i$\\
&\\
& Hence,incorrect.\\
\hline

\text{of modulus 1}&\\
Proof&$\vec{P}\vec{V}=\lambda\vec{V}$\\
&where,$\vec{V}$ is eigen vector of $\vec{P}$ and \\&$\lambda$ is eigen value of $\vec{P}$ \\
&\\
&Taking conjugate transpose on both sides,we get\\
&$\vec{V}^\theta \vec{P}^\theta=\lambda^\theta\vec{V}^\theta$\\
&\\
&$\vec{V}^\theta \vec{P}^\theta\vec{P}\vec{V}=\lambda^\theta\vec{V}^\theta\lambda\vec{V}\qquad,\because \vec{P}\vec{V}=\lambda\vec{V}$\\
&\\
&$\vec{V}^\theta \vec{I}\vec{V}=\lambda^\theta\lambda\vec{V}^\theta\vec{V}\qquad,\because \vec{P^\theta}\vec{P}=\vec{I}$\\
&\\
&$(1-\lambda^\theta\lambda)\vec{V}^\theta\vec{V}=0$\\
&Since,$\vec{V}$ is not zero.\\
&$(1-\lambda^\theta\lambda)=0$\\
&$\lambda^\theta\lambda=1$\\
&\\
&$\norm{\lambda}^2=1$\\
&\\
&$\lambda=1$\\
&\\
&Hence,correct.\\
\hline
\end{tabular}
\end{center}
\caption{Finding Correct Option}
\label{eq:solutions/2014/dec/32/table1}
\end{table*}
 
