See Table     \ref{eq:solutions/2015/june/74/tab:Ans}

\onecolumn
\begin{longtable}{|l|l|}
    \hline
    Given & $p_n\brak{x}=x^n$ for $x\in\mathbb{R}$ and $\varrho=span\{p_0,p_1,p_2,...\}$.\\
    \hline
    Vector& The set $S$ consisting of all real continuous functions on $\mathbb{R}$ forms a vector space.\\
    space&Let $f$ and $g$ be two real continuous functions from the set $S$.\\
    of real&Since the sum of two continuous function is a continuous function.\\
    continuous&$i)$ Addition is commutative $f+g=g+f$\\
    functions&$ii)$ Addition is associative$f+(g+h)=(f+g)+h$\\
    on $\mathbb{R}$&$iii)$There is unique $O$, zero function which maps every element to 0.\\
    &$iv)$Additive inverse.For each $f$ in $S$, $-f$ is a function in $S$.\\
    &$v)$Properties of scalar multiplication.For $c,c_1,c_2\in \mathbb{R}$,\\
    &\qquad $a)$ $1f=f$ where the constant function $1$ maps every element to $1$.\\
    &\qquad $b)$ $(c_1c_2)f=c_1(c_2f)$\\
    &\qquad $c)$ $c(f+g)=cf+cg$\\
    &\qquad $d)$ $c_1+c_2)f=c_1f+c_2f$\\
    &Hence the set $S$ forms a vector space.\\
    \hline
    Option 1& $\varrho$ represents the vector space of polynomials. Polynomial functions are infintely \\
    & continuously differentiable.So any function that is continuous but not differentiable can \\
    & not be represented by polynomials.\\
    & Example the function $\abs{x}$ is continous but cannot be represented in \\
    &polynomial basis.Therefore option 1 is incorrect.\\
    \hline
    Option 2& $\varrho$ forms a subspace of all real valued continuous function on $\mathbb{R}$\\
    &Let $\alpha,\beta$ be two polynomial functions of order m and n, represented by the tuple of\\ &coefficients $(a_0,a_2,a_2..a_m)$ and $(b_0,b_1,b_2...b_n)$,then $c\alpha+\beta$ is also\\
    & a polynomial function whose coefficients are $(ca_0+b_0,ca_1+b_1,ca_2+b_2...)$\\
    &Therefore $\varrho$ is a subspace of all real valued continuous functions on $\mathbb{R}$.\\
    & For example consider two functions $f=\{2,0,4\}$ and $g=\{0,2,1,5\}$,then $2f+g$ \\
    & will be $2f+g=2\brak{2+4x^2}+\brak{2x+x^2+5x^3}=4+2x+9x^2+5x^3=\{4,2,9,5\}$.\\
    \hline
    Option 3&Consider the expression\\
    &$a_0p_0+a_1p_1+a_2p_2+...=O\implies a_0=a_1=a_2=...=0$\\
    &Hence $\{p_0,p_1,p_2,..\}$ are linearly independent set in the vector space of all real valued \\
    &continuous functions on $\mathbb{R}$.\\
    \hline
    Option 4&The fundamental period of trigonometric functions is finite, where as polynomials are \\
    &aperiodic. So, they cannot belong to the same class.\\
    &For example $\sin{x}$ has a fundamental period of $2\pi$. $\tan{x}$ is continuous in the interval \\
    &$(-\frac{\pi}{2},\frac{\pi}{2})$, but is not defined at $k\frac{\pi}{2}$ where $k\in odd(\mathbb{N})$.\\
    \hline
    \caption{Answer}
    \label{eq:solutions/2015/june/74/tab:Ans}
\end{longtable}
\twocolumn
