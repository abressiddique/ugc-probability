Let us consider Tails as a failure and Heads as a success, then $X$ and $Y$, both, can be seen to be \underline{geometric random variables}, with $p = \frac{1}{2}$ and:
\begin{align}
    \pr{X = k} = \pr{Y = k} = (1-p)^k p  
    \label{dec2015-105:formula}
\end{align}
\begin{enumerate}
    \item \underline{Option 1:}\\
    
    Substituting the value of $p$ in \eqref{dec2015-105:formula}:
{\small
\begin{align}
    \pr{X = k} = \pr{Y = k} = \frac{1}{2^{k+1}} = 2^{-(k+1)}
\end{align}
}
To test for independence of $X$ and $Y$, we calculate $\pr{X = k, Y = k}$, which means obtaining $k$ tails, 1 head, $k$ tails, and one head, in order. Thus,
{\small
\begin{align}
    \pr{X = k,Y = k} = (1-p)^k p \times (1-p)^k p\\
    =\pr{X = k}\pr{Y = k}
    \label{dec2015-105:independence}
    \end{align}
    }
    Thus, $X$ and $Y$ are independent, and hence, \textbf{Option 1 is correct}\\
    \item \underline{Option 2:}\\
    
    From \eqref{dec2015-105:formula} and \eqref{dec2015-105:independence}, we get:
\begin{align}
    \pr{N = k} = \pr{X+Y = k}\\
    = \sum_{i=0}^k\pr{X = i, Y = k-i} \\
    = \sum_{i=0}^k\pr{X = i} \pr{Y = k-i}\\
    =\sum_{i=0}^k (1-p)^i p (1-p)^{k-i} p\\
    = p^2(1-p)^k (k+1)\\
    =(k+1) 2^{-(k+2)}
\end{align}
Hence, \textbf{Option 2 is incorrect}\\
 \item \underline{Option 3:}\\
 
 We know, if a conditional distribution is independent, then:
\begin{align}
    \pr{X = x,Y=y | Y=y} = \pr{X=x}
\end{align}
Thus, the conditional distribution of $X$ given $N=n$:\\
{\small
\begin{align}
    \pr{X=k|N=n} = \frac{\pr{X=k,X+Y=n}}{\pr{N=n}}\\
    =\frac{\pr{X=k,Y=n-k}}{\pr{N=n}}\\
    =\frac{2^{-(k+1)}2^{-(n-k+1)}}{(n+1) 2^{-(n+2)}}\\
    = \frac{1}{n+1} 
    \label{dec2015-105:option4}\\
    \neq \pr{X = k}
\end{align}
}
Similarly,
\begin{align}
    \pr{Y=k|N=n} = \frac{1}{n+1}\\
    \neq \pr{Y=k}
\end{align}
Thus, the condition for independence fails and hence, \textbf{Option 3 is incorrect}\\
\item \underline{Option 4:}\\
From \eqref{dec2015-105:option4}, we see that \textbf{Option 4 is correct}\\
    
\end{enumerate}
The correct options are \textbf{(1)} and \textbf{(4)}