\begin{table}[htp]
\centering
    \resizebox{\columnwidth}{20mm}{
\begin{tabular}{ |c|c|c|} 
\hline
\textbf{Symbol} & \textbf{expression/definition} \\
\hline
$S_n$ & $\displaystyle \sum_{i=1}^{n}X_i$  \\
\hline
$\mu_n$ & $\displaystyle \frac{1}{n} \sum_{i=1}^{n}X_i$  \\
\hline
& Independent continuous random\\$X$&variable identical to $X_1,X_2,...,X_n$ \\
\hline
\end{tabular}}
\caption{Variables and their definitions}
\label{dec2015-106:table1}
\end{table}
\begin{enumerate}
\item 
Given
\begin{align}
\displaystyle S_n=\sum_{i=1}^{n}X_i , n\ge 1
\end{align}
Dividing by $n$ on both sides
\begin{align}
\dfrac{S_n}{n}=\displaystyle\frac{1}{n} \sum_{i=1}^{n}X_i=\mu_n
\end{align}
It can be said that $X_1,X_2,...,X_n$ are the trials of $X$. By definition
\begin{align}
E\sbrak{X}&=\displaystyle \lim_{n\to \infty} \frac{\sum_{i=1}^{n}X_i}{n}=\lim_{n\to \infty} \frac{S_n}{n}\\
&\lim_{n\to \infty} \frac{S_n}{n}=E\sbrak{X}=\frac{1}{2} \label{dec2015-106:eq:mu_value} \\
\therefore&\lim_{n\to \infty} \frac{S_n}{n\log{n}}=0
\end{align}
\item
Using weak law, \eqref{dec2015-106:eq:mu_value}, and table \eqref{dec2015-106:table1}
\begin{align}
\lim_{n\to\infty} \pr{\abs{\mu_n-E\sbrak{X}}>\epsilon}=0, \forall \epsilon >0\\
\displaystyle \lim_{n\to\infty} \pr{S_n=\frac{n}{2}}=1 \label{dec2015-106:eq:weak_law}
\end{align}
It can be easily implied from \eqref{dec2015-106:eq:weak_law} that option B is false.
\item 
It is easy to observe from \eqref{dec2015-106:eq:mu_value} that option C is false.
\item
Using \eqref{dec2015-106:eq:weak_law}, we get
\begin{align}
\pr{\brak{S_n>\frac{n}{3}}\text{occurs for infinitely many n}}=1
\end{align}
\end{enumerate}