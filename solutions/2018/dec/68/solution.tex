See Tables     \ref{eq:solutions/2018/dec/68/table:1}
and     \ref{eq:solutions/2018/dec/68/table:2}

\onecolumn
\begin{longtable}{|l|l|}
	\hline
	\multirow{3}{*}{Invertible} 
	& \\
	& A square matrix is invertible if and only if it does not have a zero eigenvalue.\\ 
	& So, from the definition of eigen vector we can write that \\
	&\parbox{10cm}
	{\begin{align}
	\vec{A}\vec{x} \neq 0 \label{eq:solutions/2018/dec/68/eq:1}
	\end{align}}\\ 
	&The transpose of an invertible matrix is also invertible with inverse $(\vec{A}^{-1})^T$.\\
	&\parbox{10cm}
	{\begin{align}
	\vec{A}\vec{A}^{-1}=\vec{I}
	\implies(\vec{A}^{-1})^T\vec{A}^T=\vec{I}^T=\vec{I}\\
	\text{So,similarly we can say that} \nonumber \\
	\vec{A}^T\vec{y} \neq 0 \label{eq:solutions/2018/dec/68/eq:1.1}
	\end{align}}\\ 
	\hline
	\multirow{3}{*}{Derivative of F} 
	&\\
	& Suppose F: $\mathbb{R}^n\rightarrow\mathbb{R}^m$,the derivative of a function F is given by the\\
	&Jacobian matrix\\
	&\parbox{10cm}
	{\begin{align}
	 \myvec{\frac{\partial f_1}{\partial x_1} & \frac{\partial f_1}{\partial x_2} &\dots & \frac{\partial f_1}{\partial x_n} \\
	 \frac{\partial f_2}{\partial x_1} & \frac{\partial f_2}{\partial x_2} &\dots & \frac{\partial f_2}{\partial x_n} \\
    \vdots &\vdots& \ddots & \vdots\\
    \frac{\partial f_m}{\partial x_1} & \frac{\partial f_m}{\partial x_2} &\dots & \frac{\partial f_m}{\partial x_n}} \label{eq:solutions/2018/dec/68/eq:2}
	\end{align}}\\ 
	&\\
	\hline
	\multirow{3}{*}{Inner product} 
	&\\
	& The inner product of $\vec{x}$ and $\vec{y}$ is given by\\
    &\parbox{10cm}
	{\begin{align}
	\langle \vec{x},\vec{y} \rangle =\vec{x}^T\vec{y}=\vec{y}^T\vec{x}  
	\end{align}}\\ 
	&\\
    \hline
    \caption{Definition and Properties used}
    \label{eq:solutions/2018/dec/68/table:1}
\end{longtable}
\begin{longtable}{|l|l|}
	\hline
	\multirow{3}{*}{Given}
	&\\
    &\parbox{10cm}
	{\begin{align}
	F(\vec{x},\vec{y})=\langle \vec{A}\vec{x},\vec{y} \rangle 
	\end{align}}\\
    \hline
	\multirow{3}{*}{using inner product definition}
	& \\
	&\parbox{10cm}
	{\begin{align}
	F(\vec{x},\vec{y})=(\vec{A}\vec{x})^T\vec{y}=\vec{x}^T\vec{A}^T\vec{y}\\
	F(\vec{x},\vec{y})=\vec{y}^T\vec{A}\vec{x}
	\end{align}}\\
	&\\
	\hline
	\multirow{3}{*}{Derivative of F}
	&\\
	&using \eqref{eq:solutions/2018/dec/68/eq:2}, We can write that\\
	&\parbox{10cm}
	{\begin{align}
	DF(\vec{x},\vec{y})=\myvec{\frac{\partial F}{\partial x}&\frac{\partial F}{\partial y}}
	=\myvec{\vec{y}^T\vec{A} & \vec{x}^T\vec{A}^T} \label{eq:solutions/2018/dec/68/eq:main}
	\end{align}}\\
	&\\
	\hline
	\multirow{3}{*}{If $\vec{x} \neq 0$,then $DF(\vec{x},0 ) \neq 0$}
	&\\
	&using \eqref{eq:solutions/2018/dec/68/eq:main},\\
	&\parbox{10cm}
	{\begin{align}
	DF(\vec{x},0)=\myvec{0 & \vec{x}^T\vec{A}^T} \\
	\text{From $\eqref{eq:solutions/2018/dec/68/eq:1}$,we know that} \nonumber \\
	\vec{A}\vec{x} \neq 0\\
	\implies \vec{x}^T\vec{A}^T \neq 0
	\end{align}}\\
	&So, We can say that \\
	&\parbox{10cm}
	{\begin{align}
    DF(\vec{x},0)\neq 0 \label{eq:solutions/2018/dec/68/option1}
	\end{align}}\\
	&\\
    \hline
    \multirow{3}{*}{If $\vec{y} \neq 0$,then $DF(0,\vec{y} ) \neq 0$}
	&\\
	&using \eqref{eq:solutions/2018/dec/68/eq:main},\\
	&\parbox{10cm}
	{\begin{align}
	DF(0,\vec{y})=\myvec{\vec{y}^T\vec{A}  & 0} \\
	\text{From $\eqref{eq:solutions/2018/dec/68/eq:1.1}$,we know that} \nonumber \\
	\vec{A}^T\vec{y} \neq 0\\
	\implies \vec{y}^T\vec{A} \neq 0
	\end{align}}\\
	&So, We can say that \\
	&\parbox{10cm}
	{\begin{align}
    DF(0,\vec{y}) \neq 0 \label{eq:solutions/2018/dec/68/option2}
	\end{align}}\\
	&\\
	 \hline
    \multirow{3}{*}{If $(\vec{x},\vec{y}) \neq 0$,then $DF(\vec{x},\vec{y} ) \neq 0$}
	&\\
	&using \eqref{eq:solutions/2018/dec/68/eq:main},\\
	&\parbox{10cm}
	{\begin{align}
	DF(\vec{x},\vec{y}) =\myvec{\vec{y}^T\vec{A} & \vec{x}^T\vec{A}^T}\\
	\text{As $(\vec{x},\vec{y}) \neq 0,DF(\vec{x},\vec{y}) = 0 $ iff $\vec{A}$=0 }\nonumber \\
	\text{From $\eqref{eq:solutions/2018/dec/68/eq:1}$,we know that} \nonumber \\
	\vec{A} \neq 0
	\end{align}}\\
	&So, We can say that \\
	&\parbox{10cm}
	{\begin{align}
    DF(\vec{x},\vec{y}) \neq 0\label{eq:solutions/2018/dec/68/option3}
	\end{align}}\\
	&\\
    \hline
    \multirow{3}{*}{If $\vec{x} = 0$ or $\vec{y}=0$,then $DF(\vec{x},\vec{y})= 0$}
	&\\
	&From \eqref{eq:solutions/2018/dec/68/option2},\\
	&\parbox{10cm}
	{\begin{align}
    DF(0,\vec{y}) \neq 0  
	\end{align}}\\
	&From \eqref{eq:solutions/2018/dec/68/option1},\\
	&\parbox{10cm}
	{\begin{align}
    DF(\vec{x},0)\neq 0 
	\end{align}}\\
	&So, if $\vec{x} = 0$ or $\vec{y}=0$,\\
	&\parbox{10cm}
	{\begin{align}
    DF(\vec{x},\vec{y}) \neq 0
	\end{align}}\\
    &\\
    \hline
	\multirow{3}{*}{Conclusion} & \\
	& From above,we can say that options 1),2),3) are correct.\\
    &\\
	\hline
	\caption{Finding derivative of linear transformation}
    \label{eq:solutions/2018/dec/68/table:2}
\end{longtable}
\twocolumn
