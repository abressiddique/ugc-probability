\begin{definition}[Almost sure convergence]
A sequence of random variables $\cbrak{X_n}_{n\in N}$ is said to converge almost surely or with probability 1 (denoted by a.s or w.p 1) to X if \label{dec2018-104:with prob 1}
\begin{equation}
    \Pr(\omega |X_n(\omega) \to X(\omega))=1
\end{equation}
\end{definition}
\begin{definition}[Convergence in probability]
A sequence of random variables $\cbrak{X_n}_{n\in N}$ is said to converge in probability (denoted by i.p) to X if
\begin{equation}
    \lim_{n \to \infty} \Pr(\left| X_{n}-X\right|>\epsilon)=0 ,\forall \epsilon>0
\end{equation}\label{dec2018-104:in prob}
\end{definition}
\begin{theorem}[Weak law of large numbers]
\label{dec2018-104:theorem}
Let $X_1,X_2,\cdots $ be i.i.d random variables with same expectation($\mu$) and finite variance($\sigma^2$).Let $S_{n}=X_1+X_2+\cdots X_n$,Then as $n \to \infty$
\begin{equation}
    \frac{S_n}{n} \xrightarrow{i.p}  \mu,
\end{equation}
in probability
\end{theorem}
\begin{theorem}[Strong law of large numbers]
\label{dec2018-104:theorem2}
Let $X_1,X_2,\cdots $ be i.i.d random variables with same expectation($\mu$) and finite variance($\sigma^2$).Let $S_{n}=X_1+X_2+\cdots X_n$,Then as $n \to \infty$
\begin{equation}
    \frac{S_n}{n} \xrightarrow{a.s}  \mu,
\end{equation}
almost surely.
\end{theorem}
\begin{theorem}[Central limit theorem]
\label{dec2018-104:theorem3}
The Central limit theorem states that the distribution of the sample approximates a normal distribution as the sample size becomes larger,given that all the samples are equal in size,regardless of the distribution of the individual samples.
\end{theorem}
Given $X_1,X_2, \cdots$ follow normal distribution with mean 0 and variance 1.
\begin{equation}
    f_{X_i}(x)=\frac{1}{\sqrt{2}\pi}e^{-\frac{x^2}{2}} ,i \in \cbrak{1,2,\cdots}
\end{equation}
As $X_1,X_2,\cdots $ are i.i.d random variables therefore $X_{1}^2,X_
{2}^2,\cdots$ are also identical and independent.
We can write
\begin{equation}
    E(X^2)=Var(X) \label{dec2018-104:eq:x2}
\end{equation}
\begin{enumerate}
\item \begin{align}
    E\brak{\frac{S_{n}-n}{\sqrt{2}}}&=E\brak{\frac{\sum_{i}{(X_{i}^{2}-1)}}{\sqrt{2}}}\\
    &={\frac{\sum_{i}E{(X_{i}^{2}-1)}}{\sqrt{2}}}\label{dec2018-104:eq:expectation}
\end{align}
From \eqref{dec2018-104:eq:x2} we can write
\begin{equation}
    E\brak{\frac{S_{n}-n}{\sqrt{2}}}=0
\end{equation}
\begin{align}
    Var\brak{\frac{S_{n}-n}{\sqrt{2}}}&=Var\brak{\frac{\sum_{i}{(X_{i}^{2}-1)}}{\sqrt{2}}}\\
    &={\frac{\sum_{i}Var{(X_{i}^{2}-1)}}{\sqrt{2}}}
\end{align}
\begin{align}
    Var(X_{i}^2-1)&=\int_{-\infty}^{\infty}(X_{i}^2-1)^2 f_{X_{i}}(x)dx\\
    &=\int_{-\infty}^{\infty}(X_{i}^4+1-2X_{i}^{2}) f_{X_{i}}(x)dx\\
    &=2\label{dec2018-104:eq:var}
\end{align}
\begin{align}
    Var\brak{\frac{S_{n}-n}{\sqrt{2}}}&=n\sqrt{2}    
\end{align}
Hence from theorem \ref{dec2018-104:theorem2} as $n \to \infty$
\begin{equation}
    \brak{\frac{S_{n}-n}{\sqrt{2}}}\sim N(0,n\sqrt{2})
\end{equation}
Hence \textbf{Option A is false.}
\item Given 
\begin{equation}
    S_{n}=X_{1}^2+X_{2}^2+\cdots+X_{n}^2.\forall n\geq 1
\end{equation}
Hence from theorem \ref{dec2018-104:theorem} we can write 
\begin{align}
    \frac{S_n}{n} \xrightarrow{i.p} Var(X)
\end{align}
\begin{equation}
    \implies \frac{S_n}{n} \xrightarrow{i.p} 1
\end{equation}
in probability.From definition \ref{dec2018-104:in prob} we can write,
\begin{equation}
    \implies \Pr{\brak{\left|\frac{S_n}{n}-1\right|>\epsilon}}\to 0,\forall \epsilon>0
\end{equation}
Hence \textbf{Option B is false .}
\item Given 
\begin{equation}
    S_{n}=X_{1}^2+X_{2}^2+\cdots+X_{n}^2.\forall n\geq 1
\end{equation}
Hence from theorem \ref{dec2018-104:theorem} we can write 
\begin{align}
    \frac{S_n}{n} \xrightarrow{i.p} Var(X)
\end{align}
\begin{equation}
    \implies \frac{S_n}{n} \xrightarrow{a.s} 1
\end{equation}
almost surely.From definition \ref{dec2018-104:with prob 1} we can write,
\begin{equation}
    \frac{S_{n}}{n} \xrightarrow{w.p.1} 1
\end{equation}
with probability 1.
Hence \textbf{Option C is true}.
\item Consider,
\begin{equation}
    E\brak{\frac{S_{n}-n}{\sqrt{n}}}=0
\end{equation}
using \eqref{dec2018-104:eq:x2} and \eqref{dec2018-104:eq:expectation}.
\begin{align}
     Var\brak{\frac{S_{n}-n}{\sqrt{n}}}&=\frac{2n}{\sqrt{n}}\\
     &=2\sqrt{n}.
\end{align}
using \eqref{dec2018-104:eq:var}.From theorem \ref{dec2018-104:theorem3} we can write,
\begin{equation}
    \brak{\frac{S_{n}-n}{\sqrt{n}}} \sim N(0,2 \sqrt{n})\label{dec2018-104:eq:D}
\end{equation}
\begin{equation}
     \Pr{\brak{\frac{S_{n}-n}{\sqrt{n}} \leq x}}= \Pr{\brak{S_{n} \leq n+\sqrt{n}x}}
\end{equation}
Hence using \eqref{dec2018-104:eq:D}, \textbf{Option D is false.}
\end {enumerate}