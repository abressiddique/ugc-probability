See Tables     \ref{eq:solutions/2018/june/77/Table.1}
and     \ref{eq:solutions/2018/june/77/Table.2}

\onecolumn
\begin{longtable}{|c|l|}
    \hline
	\multirow{3}{*}{Quadratic Form of a matrix} 
	& \\
	& Let $\vec{V}$ be a vector space over $\mathbb{R}$. $\vec{A}$ be a symmetric matrix $n\times n$.\\& Quadratic form on $\vec{V}$ is a real function, $(\vec{F}:\vec{V}\rightarrow\mathbb{R})$ defined as 
	\\& $F(x)= \vec{x}\vec{A}\vec{x}^T= \sum_{i,j=1}^{n}a_{ij}x_ix_j$  where $\vec{x} \in \vec{V}$ \\

	\hline
	\multirow{3}{*}{Signature of Quadratic form} 
	& \\
	& The signature of quadratic form is $(n_{+},n_{-},n_{0})$ \\ & where $n_{+}$ is the number of positive entries, $n_{-}$ is number of negative entries and \\ & $n_{0}$ is number of zero's in $a_{ii}$ \\
	\hline
	\multirow{3}{*}{Rank of quadratic form} 
	& \\
	& Rank of quadratic form is the rank of its matrix \\
	& which is maximum number of linearly independent rows/columns of a matrix \\
	\hline
	\caption{Definitions}
    \label{eq:solutions/2018/june/77/Table.1}
\end{longtable}
\begin{longtable}{|c|l|}
\hline
\multirow{3}{*}{\textbf{Option 1}} & \\ & The matrix of second order partial derivatives of the quadratic form of $\vec{Q}$ is $2\vec{A}$.\\
\hline
\multirow{3}{*}{Solution} & $\vec{Q}(x,y,z)$= \myvec{x&y&z}$\vec{A}$\myvec{x\\y\\z} = \myvec{x&y&z}\myvec{x+2y\\-2z\\z} = $x^2+z^2+2xy-2yz$ \\
& First order partial derivaties: $\frac{\partial\vec{Q}}{\partial x}= 2x+2y \quad \frac{\partial\vec{Q}}{\partial y} = 2x-2z \quad \frac{\partial\vec{Q}}{\partial z}= 2z-2y $\\
& Second order partial derivatives of: $\frac{\partial^2\vec{Q}}{\partial x^2}= 2 \quad \frac{\partial^2\vec{Q}}{\partial y^2}= 0 \quad \frac{\partial^2\vec{Q}}{\partial z^2}= 2$\\
&$ \quad \frac{\partial^2 \vec{Q}}{\partial x \partial y}= \frac{\partial^2 \vec{Q}}{\partial y \partial x}= 2 \quad  \frac{\partial^2 \vec{Q}}{\partial x \partial z}= \frac{\partial^2 \vec{Q}}{\partial z \partial x}= 0 \quad \frac{\partial^2 \vec{Q}}{\partial y \partial z}= \frac{\partial^2 \vec{Q}}{\partial z \partial y}= -2$  \\
& Matrix of second order partial derivatives $\vec{Q}$: \myvec{\frac{\partial^2\vec{Q}}{\partial x^2} &\frac{\partial^2 \vec{Q}}{\partial x \partial y}& \frac{\partial^2 \vec{Q}}{\partial x \partial z} \\ \frac{\partial^2 \vec{Q}}{\partial y \partial x} & \frac{\partial^2\vec{Q}}{\partial y^2}&  \frac{\partial^2 \vec{Q}}{\partial y \partial z}\\ \frac{\partial^2 \vec{Q}}{\partial z \partial x} & \frac{\partial^2 \vec{Q}}{\partial z \partial y} &\frac{\partial^2\vec{Q}}{\partial z^2}} = \myvec{2&2&0\\2&0&-2\\0&-2&2} $\neq 2\vec{A}$\\
& Hence, $\textbf{Option 1}$ is not correct.\\

\hline


\multirow{3}{*}{\textbf{Option 2}} &\\ & The rank of the quadratic form of $\vec{Q}$ is 2\\
\hline
\multirow{3}{*}{Solution} & From above we have quadratic form of $\vec{Q}=\myvec{2&2&0\\2&0&-2\\0&-2&2} $\\
& Echelon form reduction: \myvec{2&2&0\\2&0&-2\\0&-2&2}$\xleftrightarrow{R_1=\frac{1}{2}}$\myvec{1&1&0\\2&0&-2\\0&-2&2}
 $\xleftrightarrow{R_2 \rightarrow R_2-2R_1}$ \myvec{1&1&0\\0&-2&-2\\0&-2&2} \\& $\xleftrightarrow{R_2 \rightarrow \frac{-1}{2}R_2}$ \myvec{1&1&0\\0&1&1\\0&-2&2} $\xleftrightarrow{R_3 \rightarrow R_3+2R_2}$ \myvec{1&1&0\\0&1&1\\0&0&4} $\xleftrightarrow{R_3 \rightarrow \frac{1}{4}R_3}$ \myvec{1&1&0\\0&1&1\\0&0&1}
\\& $\xleftrightarrow{R_1 \rightarrow R_1-R_2}$ \myvec{1&0&0\\0&1&1\\0&0&1}$\xleftrightarrow{R_2 \rightarrow R_2-R_3}$ \myvec{1&0&0\\0&1&0\\0&0&1} \\
& Rank = Number of non-zero rows = 3 $\neq$ 2 \\
&  Hence, $\textbf{Option 2}$ is not correct.\\
\hline

\multirow{3}{*}{\textbf{Option 3}} & \\ & The signature of the quadratic form $\vec{Q}$ is $(++0)$\\
\hline
\multirow{3}{*}{Solution} & From above we have quadratic form of $\vec{Q}$ = \myvec{2&2&0\\2&0&-2\\0&-2&2}\\
& Finding eigen values: \mydet{\vec{Q}-\lambda\vec{I}}= \myvec{2-\lambda& 2&0\\2&-\lambda&-2\\0&-2&2-\lambda}\\&
$\implies (2-\lambda)\brak{-2\lambda +\lambda^2+ 4} +8 = 0$\\&
$\implies \lambda^3-4\lambda^2-4\lambda+16=0 $ \\&
$\lambda_1 = 4 \quad \lambda_2= 2 \quad \lambda_3 = -2$ \\&
Signature = $(n_{+},n_{-},n_{0}) = (2,1,0)\neq (++0)$\\&
Hence, $\textbf{Option 3}$ is not correct.\\
\hline

\multirow{3}{*}{\textbf{Option 4}} &\\ & The quadratic form $\vec{Q}$ takes the value 0 for some non-zero vector $(x,y,z)$\\
\hline
\multirow{3}{*}{Solution} & From above we have quadratic form of $\vec{Q}$ = \myvec{2&2&0\\2&0&-2\\0&-2&2}\\ &
we can see that few elements are zero even though the vectors are non-zero. \\ & Therefore, \textbf{Option 4} is correct.\\

	\hline
	\caption{Solution}
    \label{eq:solutions/2018/june/77/Table.2}
\end{longtable}
  
\twocolumn
