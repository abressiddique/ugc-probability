See Tables     \ref{eq:solutions/2018/june/75/table:1} and     \ref{eq:solutions/2018/june/75/table:1}


\onecolumn
\begin{longtable}{|l|l|}
	\hline
	\multirow{3}{*}{Jordan canonical form} 
	&\\
	& If $\vec{A}$ is a matrix of order n$\times$n ,then the Jordan canonical form\\
	&of $\vec{A}$  is a matrix of order n$\times$n expressed as \\
	&\parbox{10cm}
	{\begin{align}
	\vec{J} = \myvec{\vec{J_1} & & \\
    & \ddots & \\
    & &
    \vec{J_{k}}} \label{eq:solutions/2018/june/75/eq:1}
	\end{align}}\\ 
	&\text{where} $\vec{J_1},...,\vec{J_k}$ are the Jordan blocks. \\
	&\\
	\hline
	\multirow{3}{*}{Algebraic multiplicity $A_M$} 
	& \\
	& Algebraic multiplicity of characteristic value $\lambda$ in the characteristic \\ 
	& polynomial determines the size of Jordan block for that eigen value\\
	&\parbox{10cm}
	{\begin{align}
	A_M= \text{Size of Jordan block for that $\lambda$ } \label{eq:solutions/2018/june/75/eq:2}
	\end{align}}\\ 
	&\\
	\hline
	\multirow{3}{*}{Geometric multiplicity $G_M$} 
	&\\
	& Geometric multiplicity determines the number of Jordan sub-blocks \\
	&in a Jordan block for $\lambda$\\
	&\\
    \hline
    	\multirow{3}{*}{Minimal Polynomial}
	&\\
	&The multiplicity of $\lambda$ in the minimal polynomial determines the\\
	&  size of the largest sub-block.\\
	&\\
    \hline
    \caption{Definition and Properties used}
    \label{eq:solutions/2018/june/75/table:1}
\end{longtable}
\begin{longtable}{|l|l|}
	\hline
	\multirow{3}{*}{Characteristic polynomial}
	&\\
    &\parbox{10cm}
	{\begin{align}
	p\brak{x}=\brak{x-3}^2\brak{x-2}^4
	\end{align}}\\
    \hline
	\multirow{3}{*}{Algebraic Multiplicity}
	& \\
	&\parbox{10cm}
	{\begin{align}
	\text{For }\lambda=3, A_M=2\\ \text{For }\lambda=2, A_M=4 
	\end{align}}\\
	&\\
	\hline
	\multirow{3}{*}{Minimal polynomial}
	&\\
    &\parbox{10cm}
	{\begin{align}
	m\brak{x}=\brak{x-3}\brak{x-2}^2
	\end{align}}\\
	&\\
	\hline
	\multirow{3}{*}{Finding Jordan blocks for $\lambda_{1}$=3}
	&\\
	&For $\lambda_{1}$=3,We can write from table\ref{eq:solutions/2018/june/75/table:1} that\\
	&\parbox{10cm}
	{\begin{align}
	\text{The highest order of Jordan block}=1 \nonumber\\
	\text{Size of Jordan block}=A_M=2 \nonumber
	\end{align}}\\
	&The Jordan blocks for $\lambda_{1}$=3\\
	&\parbox{10cm}
	{\begin{align}
	\vec{J_1}=\myvec{3},
	\vec{J_2}=\myvec{3}
	\end{align}}\\
	&\\
	\hline
	\multirow{3}{*}{Finding Jordan blocks for $\lambda_{1}$=2}
	&\\
	&For $\lambda_{1}$=2,We can write from table\ref{eq:solutions/2018/june/75/table:1} that\\
	&\parbox{10cm}
	{\begin{align}
	\text{The highest order of Jordan block}=2 \nonumber\\
	\text{Size of Jordan block}=A_M=4 \nonumber
	\end{align}}\\
	&The Jordan blocks for $\lambda_{1}$=3\\
	&\parbox{10cm}
	{\begin{align}
	\vec{J_3}=\myvec{2&1\\0&2},
	\vec{J_4}=\myvec{2&1\\0&2}\\
	\text{or} \nonumber\\ 
	\vec{J_3}=\myvec{2&1\\0&2},
	\vec{J_4}=\myvec{2},\vec{J_5}=\myvec{2}
	\end{align}}\\
	&\\
    \hline
	\multirow{3}{*}{Jordan canonical form} 
	&\\
	& Jordan canonical form of $\vec{A}$ is \\
	&\parbox{10cm}
	{\begin{align}
	\vec{J} = \myvec{\vec{J_1} & & \\
	& \vec{J_2} & \\
    && \vec{J_3} & \\
    &&& \vec{J_{4}} } \text{or}
    \myvec{\vec{J_1} & & \\
	& \vec{J_2} & \\
    && \vec{J_3} & \\
    &&& \vec{J_{4}} & \\
    &&&& \vec{J_{5} } }
	\end{align}}\\ 
	&\parbox{10cm}
	{\begin{align}
	\myvec{3&0&0&0&0&0\\0&3&0&0&0&0\\0&0&2&1&0&0\\0&0&0&2&0&0\\0&0&0&0&2&1\\0&0&0&0&0&2}\text{or}
	\myvec{3&0&0&0&0&0\\0&3&0&0&0&0\\0&0&2&1&0&0\\0&0&0&2&0&0\\0&0&0&0&2&0\\0&0&0&0&0&2}
	\end{align}}\\ 
	& \\
    \hline
	\multirow{3}{*}{Conclusion} & \\
	& From above,we can say that options 2) and 3) are correct.\\
    &\\
	\hline
	\caption{Finding Jordan canonical form}
    \label{eq:solutions/2018/june/75/table:2}
\end{longtable}
\twocolumn
