\begin{figure}[h!]
    \caption{Markov chain}
    \resizebox{\columnwidth}{!}{%
\begin{tikzpicture}[font=\sffamily]
        % Setup the style for the states
        \tikzset{node style/.style={state, 
                                    minimum width=2.5cm,
                                    line width=1mm,
                                    fill=gray!20!white}}
        % Draw the states
        \node[node style] at (0, 0)     (A)     {X=5 or X=6};
        \node[node style] at (6, 0)     (B)     {X=2};
        \node[node style] at (6, -5.196) (end) {End};
        \node[node style] at (0, -5.196) (C) {X=1 or X=3 or X=4};
        % Connect the states with arrows
        \draw[every loop,
              auto=right,
              line width=1mm,
              >=latex,
              draw=orange,
              fill=orange]
              (A) edge[bend left=0] node {$\frac{1}{2}$} (C)
              (C) edge[bend left=0] node {1} (end)
              (A) edge[loop left] node {$\frac{1}{3}$} (A)
            (A)     edge[bend right=0, auto=left] node {$\frac{1}{6}$} (B)
            (B)     edge[bend left=0, auto=left] node {1} (end);
    \end{tikzpicture}
    }
    \end{figure}
Let us assume the following table.
\begin{table}[h!]
\centering
\caption{}
\label{june2018-49:table:1}
\resizebox{\columnwidth}{!}{%
\begin{tabular}{|c|c|c|c|}
    \hline
    state 1&state 2 &state 3 &state 4\\
    \hline
$X=5$ or $X=6$&$X=2$&$X=1$ or $X=3$ or $X=4$ &end \\    
    \hline
\end{tabular}}
\end{table}
Let us represent the markov chain diagram in a matrix.Let $P_{ij}$ represent the element of a matrix which is in $i^{th}$ row and $j^{th}$ column.The value of $P_{ij}$ is equal to probability of transition from state $i$ to state $j$
\begin{equation}
P=\begin{bmatrix}
\frac{1}{3}&\frac{1}{6}&\frac{1}{2}&0\\
0&0&0&1\\
0&0&0&1\\
0&0&0&0\\
\end{bmatrix}
\end{equation}
We need the probability that $X=2$.Hence required probability is
\begin{equation}
    P_{12}+(P_{12})^{2}+\cdots+\infty \label{june2018-49:eq:reqprob}
\end{equation}
where $P_{12}^{n}$ represents the 1st row ,2nd column element in the $P^{n}$
\begin{align}
P^2&=\begin{bmatrix}
\frac{1}{3}&\frac{1}{6}&\frac{1}{2}&0\\
0&0&0&1\\
0&0&0&1\\
0&0&0&0\\
\end{bmatrix} \times
\begin{bmatrix}
\frac{1}{3}&\frac{1}{6}&\frac{1}{2}&0\\
0&0&0&1\\
0&0&0&1\\
0&0&0&0\\
\end{bmatrix}\\
&=\begin{bmatrix}
\frac{1}{9}&\frac{1}{18}&\frac{1}{6}&0\\
0&0&0&0\\
0&0&0&0\\
0&0&0&0\\
\end{bmatrix}
\end{align}
\begin{align}
    P^3&=(P^2)(P^1)\\
    &=\begin{bmatrix}
\frac{1}{9}&\frac{1}{18}&\frac{1}{6}&0\\
0&0&0&0\\
0&0&0&0\\
0&0&0&0\\
\end{bmatrix}\times
\begin{bmatrix}
\frac{1}{3}&\frac{1}{6}&\frac{1}{2}&0\\
0&0&0&1\\
0&0&0&1\\
0&0&0&0\\
\end{bmatrix}\\
&=\begin{bmatrix}
\frac{1}{27}&\frac{1}{54}&\frac{1}{18}&0\\
0&0&0&0\\
0&0&0&0\\
0&0&0&0\\
\end{bmatrix}
\end{align}
From above we can notice that each time $P_{12}$ reduces by $\frac{1}{3}$.Hence from \eqref{june2018-49:eq:reqprob},
\begin{equation}
    \sum_{i=0}^{\infty}\brak{\frac{1}{3}}^i \frac{1}{6}
\end{equation}
From Geometric progression we can write ,required probability =$\frac{1}{4}$
$\therefore$ \textbf{option C is correct}