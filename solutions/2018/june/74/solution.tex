See Table \ref{eq:solutions/2018/june/74/table:1}

\onecolumn
\begin{longtable}{|l|l|}
\hline
\text{Given} &
\text{$\vec{A}$ be the $n\times n$ matrix where $n>1$ satisfying the following equation } \\
& \parbox{10cm}{\begin{align}
    \vec{A}^2-7\vec{A}+12\vec{I}_{n\times n}=\vec{0}_{n\times n} 
\label{eq:solutions/2018/june/74/given}
\end{align}}\\
\hline
\text{Explanation} & \text{The Cayley Hamilton Theorem states that every square matrix satisfies its own characteristic}\\
& \text{equation}.\\
& \text{Using this theorem the given equation \eqref{eq:solutions/2018/june/74/given} can be written as ,}\\
& \parbox{10cm}{\begin{align}
    \lambda^2-7\lambda+12&=0\label{eq:solutions/2018/june/74/4}\\
    (\lambda-4)(\lambda-3)&=0\\
    \lambda_1&=3\\
    \lambda_2&=4
\end{align}}\\
& \text{Here $\lambda_1$ and $\lambda_2$ were eigen values of matrix $\vec{A}$}\\
& \text{We know that determinant is product of eigen values.}\\
& \parbox{10cm}{\begin{align}
    d&=Det(\vec{A})\\
    \implies d &=\lambda_1\lambda_2\\
     \implies d&=12\neq 0 \label{eq:solutions/2018/june/74/1}
\end{align}}\\
\hline
\textbf{Statement 1} & \text{$\vec{A}$ is invertible}\\
\hline
& \text{From equation \eqref{eq:solutions/2018/june/74/1}, since $d \neq 0$ the given matrix $\vec{A}$ is Invertible}.\\
& \parbox{10cm}{\begin{center}
\textbf{True Statement }
\end{center}}\\
\hline 
\textbf{Statement 2} & \parbox{10cm}{\begin{align}
    t^2-7t+12n=0 \label{eq:solutions/2018/june/74/trace}
\end{align}}\\
\hline
& We know that the trace is the sum of the eigen values.\\
& \parbox{10cm}{\begin{align}
    t&=Tr(\vec{A})\\
    \implies t &=\lambda_1+\lambda_2\\
     \implies t&=7 \label{eq:solutions/2018/june/74/t}
\end{align}}\\
& Substituting the equation \eqref{eq:solutions/2018/june/74/t} in \eqref{eq:solutions/2018/june/74/trace} we get,\\
& \parbox{10cm}{\begin{align}
    7^2-7(7)+12n&=0\\
    12n&=0\label{eq:solutions/2018/june/74/2}
\end{align}}\\
& Since given that $n>1$ the equation \eqref{eq:solutions/2018/june/74/2} is not possible i.e $12n\neq 0$.\\
& \parbox{10cm}{\begin{center}
Therefore, $t^2-7t+12n=0$ is a \textbf{False Statement }
\end{center}}\\
\hline 
\textbf{Statement 3} & \parbox{10cm}{\begin{align}
    d^2-7d+12=0 \label{eq:solutions/2018/june/74/det}
\end{align}}\\
\hline
& Substituting the equation \eqref{eq:solutions/2018/june/74/1} in \eqref{eq:solutions/2018/june/74/det}, we get, \\
& \parbox{10cm}{\begin{align}
    12^2-7(12)+12&=0\\
    72&=0 
%\label{eq:solutions/2018/june/74/2}
\end{align}} \\
& From equation \eqref{eq:solutions/2018/june/74/2} it is clear that the above statement 3 is invalid.\\
& \parbox{10cm}{\begin{center}
\textbf{False Statement} 
\end{center}}\\
\hline
\textbf{Statement 4} & \parbox{10cm}{\begin{align}
    \lambda^2-7\lambda+12=0 \label{eq:solutions/2018/june/74/eigen}
\end{align}}\\
\hline
& \text{By Cayley Hamilton Theorem, equation \eqref{eq:solutions/2018/june/74/4} shows that the above statement 4 is valid.}\\
& \parbox{10cm}{\begin{center}
\textbf{True Statement }
\end{center}}\\
\hline
\caption{Explanation}
\label{eq:solutions/2018/june/74/table:1}
\end{longtable}
\twocolumn
