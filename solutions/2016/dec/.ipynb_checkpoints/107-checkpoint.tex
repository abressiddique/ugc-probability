Expected value\brak{E\brak{X}}:
It is nothing but weighted average
Median\brak{median\brak{X}}:
It is the value separating the higher half from the lower half of a data sample
Variance\brak{V\brak{X}}:
It is the expectation of the squared deviation of a random variable from its mean
\begin{enumerate}
    \item Let's consider $X$ has an exponential distribution.
    \begin{align}
        X \sim Exp\brak{\lambda}
    \end{align}
    where $\lambda$ is rate parameter.
    
    Probability function of exponential distribution,
    \begin{align}
        f_X\brak{x}=
        \begin{cases}
            \lambda e^{-\lambda x} & x\geq0\\
            0 & x<0
        \end{cases}
    \end{align}
    The expected value of $X \sim Exp\brak{\lambda}$,
    \begin{align}
        E\brak{X}=\frac{1}{\lambda}
    \end{align}
    The median of $X \sim Exp\brak{\lambda}$,
    \begin{align}
        median\brak{X}=\frac{\ln{2}}{\lambda}
    \end{align}
    \begin{align}
        \ln{2}<1\\
        \frac{\ln{2}}{\lambda}<\frac{1}{\lambda}\\
         median\brak{X}<E\brak{X}
    \end{align}
    Hence, option $1$ is correct.
    
    \item Let's consider $X$ has a uniform distribution in interval $[a,b]$,
    \begin{align}
        X \sim U\brak{a,b}
    \end{align}
    where,
    $a=$ lower limit
    
    $b=$ upper limit
    
    Probability function of uniform distribution,
    \begin{align}
        f_X\brak{k}=
        \begin{cases}
            \frac{1}{b-a} & a\leq x\leq b\\
            0 & x < a, x > b
        \end{cases}
    \end{align}
    The expected value of $X \sim U\brak{a,b}$,
    \begin{align}
        E\brak{X}=\frac{1}{2}\brak{a+b}
    \end{align}
    The median of $X \sim U\brak{a,b}$,
    \begin{align}
        median\brak{X}=\frac{1}{2}\brak{a+b}
    \end{align}
    \begin{align}
        E\brak{X}=median\brak{X}
    \end{align}
    Hence, option $2$ is incorrect.
    
    \item Let's consider $X$ has a binomial distribution,
    \begin{align}
        X \sim B\brak{n,p}
    \end{align}
    where,
    $n=$ no. of trails
    
    $p=$ success parameter
    
    Probability function of binomial distribution,
    \begin{align}
        f_X\brak{k}=
        \begin{cases}
            {^n C_k}p^k(1-p)^{n-k} & 0\leq k\leq n\\
            0 & otherwise
        \end{cases}
    \end{align}
    The expected value of $X \sim B\brak{n,p}$,
    \begin{align}
        E\brak{X}=np
    \end{align}
    The variance of $X \sim B\brak{n,p}$,
    \begin{align}
        V\brak{X}=\sigma^2=n p(1-p)
    \end{align}
    \begin{align}
        1-p\leq1\\
        n p(1-p)\leq n p\\
        V\brak{X}\leq E\brak{X}
    \end{align}
    Hence, option $3$ is incorrect.
    
    \item Let's consider $X$ has a normal distribution,
    \begin{align}
        X \sim N\brak{\mu,\sigma^2}
    \end{align}
    where,
    $\mu=$ mean of distribution
    
    $\sigma^2=$ variance
    
    Probability function of normal distribution,
    \begin{align}
        f_X\brak{k}=\frac{1}{\sigma\sqrt{2\pi}}e^{-\brak{\frac{x-\mu}{2\sigma}}^2}
    \end{align}
    The expected value of $X \sim N\brak{\mu,\sigma^2}$,
    \begin{align}
        E\brak{X}=\mu
    \end{align}
    The variance of $X \sim N\brak{\mu,\sigma^2}$,
    \begin{align}
        V\brak{X}=\sigma^2
    \end{align}
    $E\brak{X}$ and $V\brak{X}$ are user defined. So, they can take any value.
    
    Hence, option $4$ is incorrect.
    \end{enumerate}
    