Given, a fair coin is tossed till heads turns up.
\begin{align}
\tag{104.1}
\label{dec2016-104eq:0}
    p=\dfrac{1}{2},q=\dfrac{1}{2}
\end{align}
Let's define a Markov chain $\{X_{n},n=0,1,2,\dots\}$, where $X_{n}\in S=\{1,2,3,4,5\}$, such that
\begin{table}[h!]
\centering
\caption{States and their notations}
\label{dec2016-104table:1}
\begin{tabular}{|c|c|}
    \hline
    Notation & State \\
    \hline
    $S=1$ & $A$'s turn\\[1ex]
    \hline
    $S=2$ & $B$'s first turn\\[1ex]
    \hline
    $S=3$ & $B$'s second turn\\[1ex]
    \hline
    $S=4$ & $A$ wins\\[1ex]
    \hline
    $S=4$ & $B$ wins\\[1ex]
    \hline
\end{tabular}
\end{table}
The state transition matrix for the Markov chain is
\begin{align}
\tag{104.2}
\label{dec2016-104eq:p}
    P=\begin{blockarray}{cccccc}
&1 & 2 & 3 & 4 & 5 \\
\begin{block}{c[ccccc]}
  1 & 0 & 0.5 & 0 & 0.5 & 0 \\
  2 & 0 & 0 & 0.5 & 0 & 0.5 \\
  3 & 0.5 & 0 & 0 & 0 & 0.5 \\
  4 & 0 & 0 & 0 & 1 & 0 \\
  5 & 0 & 0 & 0 & 0 & 1 \\
\end{block}
\end{blockarray}
\end{align}
Clearly, the states $1,2,3$ are transient, while $4,5$ are absorbing. The standard form of a state transition matrix is
\begin{align}
\tag{104.3}
\label{dec2016-104eq:std}
    P=\begin{blockarray}{ccc}
&A & N \\
\begin{block}{c[cc]}
  A & I & O  \\
  N & R & Q \\
\end{block}
\end{blockarray}
\end{align}
where,
\begin{table}[h!]
\centering
\caption{Notations and their meanings}
\label{dec2016-104table:2}
\begin{tabular}{|c|c|}
    \hline
    Notation & Meaning \\
    \hline
    $A$ & All absorbing states\\[1ex]
    \hline
    $N$ & All non-absorbing states\\[1ex]
    \hline
    $I$ & Identity matrix\\[1ex]
    \hline
    $O$ & Zero matrix\\[1ex]
    \hline
    $R,Q$ & Other submatices\\[1ex]
    \hline
\end{tabular}
\end{table}
Converting \eqref{dec2016-104eq:p} to standard form, we get
\begin{align}
\tag{104.4}
\label{dec2016-104eq:pstd}
    P=\begin{blockarray}{cccccc}
&4 & 5 & 1 & 2 & 3 \\
\begin{block}{c[ccccc]}
  4 & 1 & 0 & 0 & 0 & 0 \\
  5 & 0 & 1 & 0 & 0 & 0 \\
  1 & 0.5 & 0 & 0 & 0.5 & 0 \\
  2 & 0 & 0.5 & 0 & 0 & 0.5 \\
  3 & 0 & 0.5 & 0.5 & 0 & 0 \\
\end{block}
\end{blockarray}
\end{align}
From \eqref{dec2016-104eq:pstd},
\begin{align}
\tag{104.5}
\label{dec2016-104eq:r,q}
    R=\begin{bmatrix}
    0.5 & 0\\
    0 & 0.5\\
    0 & 0.5\\
    \end{bmatrix},
    Q=\begin{bmatrix}
    0 & 0.5 & 0\\
    0 & 0 & 0.5\\
    0.5 & 0 & 0\\
    \end{bmatrix}
\end{align}
\newpage
The limiting matrix for absorbing Markov chain is
\begin{align}
\tag{104.6}
\label{dec2016-104eq:pbar}
    \bar P=\begin{bmatrix}
    I & O\\
    FR & O\\
    \end{bmatrix}
\end{align}
where,
\begin{align}
\tag{104.7}
\label{dec2016-104eq:f}
    F=(I-Q)^{-1}
\end{align}
is called the fundamental matrix of $P$. \\
On solving, we get
\begin{align}
\tag{104.8}
\label{dec2016-104eq:ans}
    \bar P=\begin{blockarray}{cccccc}
&4 & 5 & 1 & 2 & 3 \\
\begin{block}{c[ccccc]}
    4 & 1 & 0 & 0 & 0 & 0 \\
    5 & 0 & 1 & 0 & 0 & 0 \\
    1 & 0.5714 & 0.4285 & 0 & 0 & 0 \\
    2 & 0.1428 & 0.8571 & 0 & 0 & 0 \\
    3 & 0.2857 & 0.7142 & 0 & 0 & 0 \\
   \end{block}
\end{blockarray}
\end{align}
A element $\bar p_{ij}$ of $\bar P$ denotes the absorption probability in state $j$, starting from state $i$. Then,
\begin{enumerate}
    \item $Pr(A$ wins)=$\bar p_{14}\approx0.5714$
    \item $Pr(B$ wins)=$\bar p_{15}\approx0.4285$
\end{enumerate}
\begin{align}
\tag{104.9}
\therefore \bar p_{14} > \bar p_{15}
\end{align}
Also, in $\bar P$, all the terms in every row should sum to 1.
\begin{align}
\tag{104.10}
\Rightarrow \bar p_{14} + \bar p_{15}+ 0+0+0=1\\
\tag{104.11}
\therefore \bar p_{14}=1-\bar p_{15}
\end{align}
Therefore, options $3),4)$ are correct.\\
\begin{figure}[h]
\caption*{\textbf{Markov chain diagram}}
\centering
\begin{tikzpicture}
    % Setup the style for the states
        \tikzset{node style/.style={state, 
                                    minimum width=1.5cm,
                                    line width=1mm,
                                    fill=gray!20!white}}
        % Draw the states
        \node[node style] at (3, -4)      (bull)     {1};
        \node[node style] at (0, -8)      (bear)     {2};
        \node[node style] at (6, -8) (stagnant) {3};
        \node[node style] at (3, 0) (over1) {4};
        \node[node style] at (3, -12) (over2) {5};
        % Connect the states with arrows
        \draw[every loop,
              auto=right,
              line width=0.7mm,
              >=latex,
              draw=orange,
              fill=orange]
            (stagnant)     edge[bend right=20]            node {$\dfrac{1}{2}$} (bull)
            (stagnant)     edge[bend left=20]            node {$\dfrac{1}{2}$} (over2)
            (bull)     edge[bend right=20] node {$\dfrac{1}{2}$} (bear)
            (bull)     edge node {$\dfrac{1}{2}$} (over1)
            (bear)     edge[bend right=20] node {$\dfrac{1}{2}$} (over2)
            (bear)     edge node {$\dfrac{1}{2}$} (stagnant)
            (over1) edge[loop above]             node  {1} (over1)
            (over2) edge[loop below]             node  {1} (over2);
\end{tikzpicture}
\end{figure}