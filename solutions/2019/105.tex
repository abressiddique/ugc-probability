This is a Markov Chain ,Where the state space consists of the integers $(i=0,\pm1,\pm2,\pm3,...)$ and transition  probability is given as
\begin{align}
   P_{i,i+1} &= p =\frac{1}{2}
   \\
   P_{i,i-1} & =q =\frac{1}{2}
\end{align}
\begin{tikzpicture}
    
    \node[state]  (p) {i-1};
    \node[state,right=of p]  (q) {i};
    \node[state,right=of q]  (r) {i+1};
    \draw[every loop]
    (p) edge[bend right] node {$p $} (q)
    (q) edge[bend right] node {$q $} (p)
    (q) edge[bend right] node {$p$} (r)
    (r) edge[bend right] node {$q $} (q);
   
\end{tikzpicture}
Let $P_{i,j}^{n}$ denotes the probability of being in state j after nth transition starting from state i.
\begin{enumerate}
    \item We know that for state j in Markov chain to be \textbf{aperiodic} ,Then their  exist k such that $P_{j,j}^{n} > 0$ for all $n \geq k$. but for to return to same state j after n transitions  ,Number of forward steps should be equal to Backward steps , i.e for odd n in (2m +1)form
    \begin{align}
        P_{j,j}^{2m+1}&=0  
        \label{2019/105/eq:eq1}
    \end{align}
    when n is even in 2m form
    \begin{align}
        P_{j,j}^{2m}&=\binom{2m}{m}p^{m}q^{m}
     \\
     &=\frac{(2m)!}{m!.m!}p^{m}q^{m}
     \label{2019/105/eq:eq2}
    \end{align}
    ,As for odd n $P_{j,j}^{n} = 0$ ,$P_{j,j}^{n} > 0$ for all $n \geq k$ is not possible .which implies  all states are \textbf{Periodic}
  
    Option (1) is \textbf{incorrect}. 
    \item 
In a Markov Chain for state j to be recurrent then it should satisfy following condition
\begin{align}
     \lim_{t\to\infty}\sum_{n=1}^t P_{j,j}^{n}&=\infty
\end{align}
using Stirling approximation in equation \eqref{2019/105/eq:eq2} 
\begin{align}
    P_{j,j}^{2m} &=\frac{((2m)^{2m +\frac{1}{2}}).e^{-2m}.(2\pi)^{\frac{1}{2}}}{m^{m+\frac{1}{2}}.e^{-m}.m^{m+\frac{1}{2}}.e^{-m}.2\pi}.p^{m}q^{m}
    \\
    &=\frac{(4pq)^{2m}}{(m\pi)^{\frac{1}{2}}}
    \label{2019/105/eq:eq3}
\end{align}
In this question $p =\frac{1}{2}=q$,then using \eqref{2019/105/eq:eq1} and \eqref{2019/105/eq:eq3}
\begin{align}
    \lim_{t\to\infty}\sum_{n=1}^t P_{j,j}^{n}&=\sum_{n=2k,k=1}^{\infty}\frac{1}{(\frac{n}{2}\pi)^{\frac{1}{2}}}
\end{align}
Since $\frac{1}{n^{\frac{1}{2}}}$ is divergent,
\begin{align}
     \lim_{t\to\infty}\sum_{n=1}^t P_{j,j}^{n}& = \infty
\end{align}
Therefore state j is recurrent ,as what we calculated is independent of j ,all states are \textbf{recurrent }.
The first-passage-time probability, $f_{i,j}(n)$, of a Markov chain is the probability,given as 
\begin{equation}
\resizebox{.9\hsize}{!}{$f_{i,j}(n)=\pr{X_{n}=j,X_{n-1}\neq j,X_{n-2}\neq j,\dots X_{1}\neq j|X_{0}=i}$}
\end{equation}
The first-passage time $T_{j,j}$from a state j back to itself is of particular importance. It has the PMF $f_{j,j}(n)$ amd Distribution function $F_{j,j}(n)$
\begin{align}
    F_{j,j}(n)&=\sum_{k=0}^n f_{j,j}(k)
    \label{2019/105/eq:eq4}
\end{align}
We Know that all states are recurrent .Now i will find whether they are null recurrent or positive recurrent .
For positive recurrent 
\begin{align}
    \overline{T_{j,j}}& < \infty
\end{align}
For null recurrent 
\begin{align}
    \overline{T_{j,j}}& = \infty
\end{align}
Where $\overline{T_{j,j}}$ is mean time to enter state j starting from j.
Now calculating $\overline{T_{j,j}}$ using below formula,
\begin{align}
    \overline{T_{j,j}}&=1 + \sum _{k=0}^n (1 - F_{j,j}(k))
    \label{2019/105/eq:eq5}
\end{align}
Using \eqref{2019/105/eq:eq5} and \eqref{2019/105/eq:eq4},We get 
\begin{align}
    \overline{T_{j,j}}& = \infty
\end{align}
Therefore all states are null recurrent.
Option(3) is \textbf{correct}
 \item
    Since all states are recurrent,they communicate with each other ,therefore Markov chain is irreducible , option (2) is \textbf{correct}
 \item As all states are null recurrent , option (4) is \textbf{incorrect} 
\end{enumerate}
Therefore correct options are \textbf{2,3}
