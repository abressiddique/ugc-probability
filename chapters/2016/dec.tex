\renewcommand{\theequation}{\theenumi}
\renewcommand{\thefigure}{\theenumi}
\renewcommand{\thetable}{\theenumi}
\begin{enumerate}[label=\thesection.\arabic*.,ref=\thesection.\theenumi]
\numberwithin{equation}{enumi}
\numberwithin{figure}{enumi}
\numberwithin{table}{enumi}

\item $X_1,X_2,\ldots,X_n$ are independent and identically
distributed as $N(\mu, \sigma^2)$, $-\infty < \mu < \infty$, $\sigma^2 > 0$. Then
\begin{enumerate}
    \item $\sum_1^n\frac{(X_i-\bar{X})^2}{n-1}$ is the Minimum Variance Unbiased Estimate of $\sigma^2$\\
    \item $\sqrt{\sum_1^n\frac{(X_i-\bar{X})^2}{n-1}}$ is the Minimum Variance Unbiased Estimate of $\sigma$\\
    \item $\sum_1^n\frac{(X_i-\bar{X})^2}{n}$ is the Maximum Likelihood Estimate of $\sigma^2$\\
    \item $\sqrt{\sum_1^n\frac{(X_i-\bar{X})^2}{n}}$ is the Maximum Likelihood \qquad Estimate of $\sigma$
\end{enumerate}
%
\solution
The pdf for each random variable is same as they are all identical and independent Normal Distributions with same $\mu$ and $\sigma^2$.
\begin{align}
    f_X(x) = \frac{1}{\sqrt{2\pi\sigma^2}}\exp{\frac{(x-\mu)^2}{2\sigma^2}}
\end{align}
Let us take our maximum likelihood function for given random variable $X_i$
\begin{align}
    L(\mu ; \sigma | X_i) = \frac{1}{\sqrt{2\pi\sigma^2}}\exp{\frac{(X_i-\mu)^2}{2\sigma^2}}\label{dec2016-109:gen_eq}
\end{align}
Since all the random variables are i.i.d
\begin{align}
    L(\mu ; \sigma | X_1,X_2,\ldots,X_n) = \prod_{i=1}^nL(\mu ; \sigma | X_i)\label{dec2016-109:main_eq}
\end{align}
Let us denote:
\begin{align}
    L_m : L(\mu ; \sigma | X_1,X_2,\ldots,X_n)
\end{align}
Substituting \eqref{dec2016-109:gen_eq} for each Random Variable in \eqref{dec2016-109:main_eq}
\begin{align}
    L_m = \prod_{i=1}^n\frac{1}{\sqrt{2\pi\sigma^2}}\exp{\frac{(X_i-\mu)^2}{2\sigma^2}}
\end{align}
Taking natural log on both sides and simplifying
\begin{align}
    \ln{L_m} = \frac{-n}{2}\ln{2\pi} -n\ln{\sigma} - \sum_{i = 1}^n\frac{(X_i - \mu)^2}{2\sigma^2}
\end{align}
In order to find Maximum Likelihood we need to maximise $\mu$ and $\sigma$ w.r.t. all Random variables. Taking partial derivative w.r.t $\mu$ and taking $\sigma$ as constant 
\begin{align}
    \frac{\partial \ln{L_m}}{\partial \mu} = \sum_{i = 1}^n\frac{(X_i - \mu)}{\sigma^2}
\end{align}
The value for $\mu$ at which $L_m$ achieves maximum value is same in $\ln{L_m}$
\begin{align}
   \because \frac{\partial \ln{L_m}}{\partial \mu} &=0\\
   \therefore \sum_{i = 1}^n\frac{(X_i -\mu)}{\sigma^2}&=0
\end{align}
On simplifying the expression we get:
\begin{align}
    n\mu &= \sum_{i=1}^nX_i\\
    \mu &= \frac{1}{n}\sum_{i=1}^nX_i\label{dec2016-109:mu_value}
\end{align}
Let us denote the value achieved in \eqref{dec2016-109:mu_value} as $\bar{X}$. Taking partial derivative w.r.t $\sigma$ and taking $\mu$ as constant
\begin{align}
    \frac{\partial \ln{L_m}}{\partial \sigma} &= \frac{-n}{\sigma} + \sum_{i=1}^n\frac{(X_i - \mu)^2}{\sigma^3}
\end{align}
The value for $\sigma$ at which $L_m$ achieves maximum value is same in $\ln{L_m}$
\begin{align}
    \frac{\partial \ln{L_m}}{\partial \sigma} &= 0\\
    \frac{-n}{\sigma} + \sum_{i=1}^n\frac{(X_i - \mu)^2}{\sigma^3} &=0
\end{align}
Upon simplifying the expression
\begin{align}
\frac{n}{\sigma} &= \sum_{i=1}^n \frac{(X_i -\mu)^2}{\sigma^3}\\
{\sigma^2} &= \sum_{i=1}^n\frac{(X_i-\mu)^2}{n}\label{dec2016-109:sig_value}
\end{align}
Substituting \eqref{dec2016-109:mu_value} in \eqref{dec2016-109:sig_value}
\begin{align}
    {\sigma^2} &= \sum_{i=1}^n\frac{(X_i-\bar{X})^2}{n}\\
    {\sigma} &= \sqrt{\sum_{i=1}^n\frac{(X_i-\bar{X})^2}{n}}
\end{align}
Hence \textbf{Option 3} and \textbf{Option 4} are correct
%
\item There are two boxes. Box-$1$ contains $2$ red balls and $4$ green balls. Box-$2$ contains $4$ red balls and $2$ green balls. A box is selected at random and a ball is chosen randomly from the selected box. If the ball turns out to be red, what is the probability that Box-$1$ had been selected?
%
\solution
Box-$1$ has $2$ red balls and $4$ green balls.\\
Box-$2$ has $4$ red balls and $2$ green balls.\\
Let $B \in \{1,2\} $ represent a random variable where $1$ represents selecting box-$1$ and $2$ represents selecting box-$2$.
\begin{table}[h!]
\resizebox{\columnwidth}{!}
{ 
\begin{tabular}{|c|c|c|}
\hline
Event & definition & value\\
\hline
$ \pr{B=1} $ & Probability of selecting  & $\frac{1}{2}$\\
&Box-$1$ & \\
\hline
$ \pr{B=2} $ & Probability of selecting & $\frac{1}{2}$ \\
& Box-$2$& \\
\hline
$\pr{R=1|B=1}$ & Probability of drawing &  $\frac{1}{3}$   \\
&  red ball from Box-$1$ &\\
\hline
$\pr{G=1|B=1}$ & Probability of drawing &  $\frac{2}{3}$   \\
&  green ball from Box-$1$ &\\
\hline
$\pr{R=1|B=2}$ & Probability of drawing &  $\frac{2}{3}$   \\
&  red ball from Box-$2$ &\\
\hline
$\pr{G=1|B=2}$ & Probability of drawing &  $\frac{1}{3}$   \\
&  green ball from Box-$2$ &\\
\hline
\end{tabular}
}
\caption{Table 1} 
\label{dec2016-49:tab:1}
\end{table}
From Baye's theorem
\begin{align}
\pr{R=1}&=\pr{R=1|B=1} \times \pr{B=1} \notag \\
 & +\pr{R=1|B=2} \times \pr{B=2}  \label{dec2016-49:1}
 \end{align}
Substiting values from table \eqref{dec2016-49:tab:1} in \eqref{dec2016-49:1}
\begin{align}
\pr{R=1} &=\frac{1}{2} \label{dec2016-49:2} \\
\pr{(R=1)(B=1)}&=\pr{R=1|B=1}\notag \\
&  \times \pr{B=1} \\ 
&= \frac{1}{6}  \label{dec2016-49:3}
\end{align}
We need to find $\pr{B=1|R=1}$ \\
\begin{align}
\pr{B=1|R=1} &= \frac{\pr{(R=1 ) (B=1)}}{\pr{R=1}} \\
&=\frac{1}{3}
\end{align}
 $\therefore$ The desired probability that box-$1$ is selected $= \frac{1}{3}$ \\
 
%
\item Suppose customers arrive in a shop according to a Poisson process with rate $4$ per hour. The shop opens at $10:00$ am. If it is given that the second customer arrives at $10:40 $ am, what is the probability that no customer arrived before $10:30 $ am? 
%
    \begin{enumerate}
        \item $\frac{1}{4}$
        \item $e^{-2}$
        \item $\frac{1}{2}$
        \item $e^{\frac{1}{2}}$
    \end{enumerate}
%
%
\solution
\begin{table}[h]
\resizebox{\columnwidth}{!}{
    \begin{tabular}{|c|c|}
    \hline
         Random Variable &  Time at which people arrive\\
         \hline
        $X_p$ & $p=10:00 - 10:30$\\
        $X_q$ & $q=10:30 - 10:40$\\
        $X_r$ & $r=10:00 - 10:40$\\
        $Y$ & $10:40$\\
        \hline
    \end{tabular}
}
    \caption{Random Variables}
    \label{tab:my_label}
\end{table}
We need to find
\begin{align}
    \pr{X_p = 0 | Y = 2} \label{main_eq}
\end{align}
In the world where the $2^{nd}$ person arrives at $10:40$ am the \eqref{main_eq} becomes:
\begin{align}
    &=\frac{\pr{X_p = 0, X_q = 1}}{\pr{X_r = 1}}\\
    &= \frac{\pr{X_p = 0} \times \pr{X_q = 1}}{\pr{X_r = 1}}\label{sub_eq}
\end{align}
The Poisson function distribution for time interval $t$ and rate $\lambda$ for a random variable $X$:
\[
    f_X(x;t) = \frac{(\lambda t)^x \exp{(-\lambda t)}}{x!}
\]
For the time interval $p$:
\begin{align}
    \lambda = 4, t &= 0.5, x= 0\\
    \pr{X_p = 0} &= f_X\brak{0;\frac{1}{2}}\\
    &= e^{-2}\label{eq_1}\\
\end{align}
For the time interval $q$:
\begin{align}
    \lambda = 4, t &= \frac{1}{6},x = 1\\
    \pr{X_q = 1} &= f_X\brak{1; \frac{1}{6}}\\
    &= \frac{2}{3}e^{\frac{-2}{3}} \label{eq_2}
\end{align}
For the time interval $r$:
\begin{align}
    \lambda = 4, t &= \frac{2}{3}, x = 1\\
    \pr{X_r = 1} &= f_X\brak{1; \frac{2}{3}}\\
    &= \frac{8}{3}e^{\frac{-8}{3}} \label{eq_3}
\end{align}
Substituting \eqref{eq_1} \eqref{eq_2} \eqref{eq_3} in \eqref{sub_eq}:
\begin{align}
    \pr{X_p = 0 | Y = 2} = \frac{1}{4}
\end{align}
%
\item A fair die is thrown two times independently. Let $X,Y$ be the outcomes of these two throws and $Z=X+Y$. Let $U$ be the remainder obtained when $Z$ is divided by 6. Then which of the following statement(s) is/are true?
\begin{enumerate}
    \item $X$ and $Z$ are independent \label{dec2016-103:option 1}
    \item $X$ and $U$ are independent \label{dec2016-103:option 2}
    \item $Z$ and $U$ are independent \label{dec2016-103:option 3}
    \item $Y$ and $Z$ are not independent \label{dec2016-103:option 4}
\end{enumerate}
%
\solution
Let $X \in \{1,2,3,4,5,6\}$ represent the random variable which represents the outcome of the first throw of a dice. Similarly, $Y \in \{1,2,3,4,5,6\}$ represents the random variable which represents the outcome of the second throw of a dice.
\begin{align}
    n(X=i) = 1, \quad i \in \{1, 2, 3, 4, 5, 6\}
\end{align}
\begin{align}
    \Pr(X=i) = 
	\begin{cases}
	\frac{1}{6}   &  i \in \{1, 2, 3, 4, 5, 6\}\\ ~\\[-1em]
	0 & \text{otherwise}
	\end{cases}
\end{align}
Similarly, 
\begin{align}
    \Pr(Y=i) = 
	\begin{cases}
	\frac{1}{6}   &  i \in \{1, 2, 3, 4, 5, 6\}\\ ~\\[-1em]
	0 & \text{otherwise}
	\end{cases}
\end{align}
\begin{align}
    Z &= X+Y
    \\ \text{Let } z &\in \{1, 2, \hdots, 11, 12\}
    \\\Pr{(Z=z)} &= \Pr{(X+Y = z)}
    \\ &= \sum_{x=0}^z \Pr{(X=x)}\Pr{(Y=z-x)}
    \\ &= (6 - \abs{z-7}) \times\frac{1}{6}\times\frac{1}{6}
    \\ &= \frac{6 - \abs{z-7}}{36}
    \\\Pr(Z=z) &= 
	\begin{cases}
	\frac{6 - \abs{z-7}}{36}   &  z \in \{1, 2, \hdots, 11, 12\}\\ ~\\[-1em]
	0 & \text{otherwise}
	\end{cases}
\end{align}
$U$ is the remainder obtained when $Z$ is divided by 6.
\begin{align}
    \text{Let } u &\in \{0, 1, 2, 3, 4, 5\}
    \\\Pr{(U=u)} &= \sum_{k=0}^2\Pr{(Z = 6k+u)}
    \\\Pr{(U=0)} &= \Pr{(Z = 0)} + \Pr{(Z = 6)} + \Pr{(Z = 12)}
    \\ &= 0 + \frac{5}{36} + \frac{1}{36} = \frac{1}{6}
    \\ \text{for } u &\in \{1, 2, 3, 4, 5\}
    \\\Pr{(U=u)} &= \Pr{(Z = 0+u)} + \Pr{(Z = 6+u)}
    \\&= \frac{6 - \abs{u-7}}{36} +  \frac{6 - \abs{6+u-7}}{36}
    \\&= \frac{6 - (7-u)}{36} +  \frac{6 - (u-1)}{36}
    \\&= \frac{u - 1 + 7 - u}{36} = \frac{6}{36}
    \\&=\frac{1}{6}
    \\\Pr(U=u) &= 
	\begin{cases}
    \frac{1}{6}   &  u \in \{0, 1, 2, 3, 4, 5\}\\ ~\\[-1em]
	0 & \text{otherwise}
	\end{cases}
\end{align}
Now, for checking each option,
\begin{enumerate}
    
\item Checking if $X$ and $Z$ are independent
\begin{align}
    p_1 &= \Pr{(Z=z, X=x)}
    \\ &= \Pr{(Y=z-x, X=x)}
    \\ &= \Pr{(Y=z-x)} \times \Pr{(X=x)}
    \\ &= \begin{cases}
        \frac{1}{36} & z-x \in \{1, 2, 3, 4, 5, 6\}\\ ~\\[-1em]
        0 & \text{otherwise}
    \end{cases}
\end{align}
\begin{align}
    \Pr{(Z=z)}\times \Pr{(X=x)} &= \frac{6 - \abs{z-7}}{36} \times \frac{1}{6}
    \\&= \frac{6 - \abs{z-7}}{216}
    \\\Pr{(Z=z)}\Pr{(X=x)} &\neq \Pr{(Z=z, X=x)}  \label{dec2016-103:equation 1}
\end{align}
$X$ and $Z$ are not independent from \eqref{dec2016-103:equation 1} and hence option \eqref{dec2016-103:option 1} is false.
\item Checking if $X$ and $U$ are independent
\begin{align}
    p_2 = \Pr{(U=u, X=x)}
\end{align}
\begin{multline}
    p_2 = \Pr{((Z=u) + (Z=6+u)}
    \\+ (Z=12+u), X=x)
\end{multline}
\begin{multline}
    p_2 = \Pr{((Y=u-x) + (Y=6+u-x)}
    \\+ (Y=12+u-x), X=x)
\end{multline}
\begin{align}
    p_2 &= \frac{1}{6} \times \frac{1}{6}
    \\&= \frac{1}{36}
\end{align}
\begin{align}
    \Pr{(U=u)}\times \Pr{(X=x)} &= \frac{1}{6} \times \frac{1}{6}
    \\&= \frac{1}{36}
    \\\Pr{(U=u)}\Pr{(X=x)} &= \Pr{(U=u, X=x)}  \label{dec2016-103:equation 2}
\end{align}
$X$ and $U$ are independent from \eqref{dec2016-103:equation 2} and hence option \eqref{dec2016-103:option 2} is true.
\item Checking if $Z$ and $U$ are independent
\begin{align}
    p_3 &= \Pr{(Z=z| U=u)}
    \\p_3 &= 
    \begin{cases}
        1 & u=1 \text{ and } z=7\\ ~\\[-1em]
        \frac{1}{2} & u=0 \text{ and } z\in\{6,12\}\\ ~\\[-1em]
        \frac{1}{2} & u\in\{2,3,4,5\}  \text{ and } \\&z=u \text{ or } z=6+u\\ ~\\[-1em]
        0 & \text{otherwise}
    \end{cases}
    \\\Pr{(Z=z)} &= \frac{6 - \abs{z-7}}{36}
\end{align}
If $Z$ and $U$ are independent, then
\begin{align}
    \Pr{(Z=z| U=u)} &= \frac{\Pr{(Z=z, U=u)}}{\Pr{(U=u)}}
    \\&= \frac{\Pr{(Z=z)}\Pr{(U=u)}}{\Pr{(U=u)}}
    \\&= \Pr{(Z=z)}
\end{align}
But,
\begin{align}
    \Pr{(Z=z| U=u)} \neq \Pr{(Z=z)} \label{dec2016-103:equation 3}
\end{align}
$X$ and $U$ are not independent from \eqref{dec2016-103:equation 3} and hence option \eqref{dec2016-103:option 3} is false.
\item Checking if $Y$ and $Z$ are independent
\begin{align}
    p_1 &= \Pr{(Z=z, Y=y)}
    \\ &= \Pr{(X=z-y, Y=y)}
    \\ &= \Pr{(X=z-y)} \times \Pr{(Y=y)}
    \\ &= \begin{cases}
        \frac{1}{36} & z-y \in \{1, 2, 3, 4, 5, 6\}\\ ~\\[-1em]
        0 & \text{otherwise}
    \end{cases}
\end{align}
\begin{align}
    \Pr{(Z=z)}\times \Pr{(Y=y)} &= \frac{6 - \abs{z-7}}{36} \times \frac{1}{6}
    \\&= \frac{6 - \abs{z-7}}{216}
    \\\Pr{(Z=z)}\Pr{(Y=y)} &\neq \Pr{(Z=z, Y=y)}  \label{dec2016-103:equation 4}
\end{align}
$X$ and $Z$ are not independent from \eqref{dec2016-103:equation 4} and hence option \eqref{dec2016-103:option 4} is true.
\end{enumerate}
Thus, options \eqref{dec2016-103:option 2} and \eqref{dec2016-103:option 4} are true.
%
\item Let $X$ be a random variable with a certain non-degenerate distribution. Then identify the correct statements
\begin{enumerate}
    \item If $X$ has an exponential distribution then $median\brak{X}<E\brak{X}$
    \item If $X$ has a uniform distribution on an interval $[a,b]$, then $E\brak{X}<median\brak{X}$
    \item If $X$ has a Binomial distribution then $V\brak{X}<E\brak{X}$
    \item If $X$ has a normal distribution, then $E\brak{X}<V\brak{X}$
\end{enumerate}
%
\solution
Expected value\brak{E\brak{X}}:
It is nothing but weighted average
Median\brak{median\brak{X}}:
It is the value separating the higher half from the lower half of a data sample
Variance\brak{V\brak{X}}:
It is the expectation of the squared deviation of a random variable from its mean
\begin{enumerate}
    \item Let's consider $X$ has an exponential distribution.
    \begin{align}
        X \sim Exp\brak{\lambda}
    \end{align}
    where $\lambda$ is rate parameter.
    
    Probability function of exponential distribution,
    \begin{align}
        f_X\brak{x}=
        \begin{cases}
            \lambda e^{-\lambda x} & x\geq0\\
            0 & x<0
        \end{cases}
    \end{align}
    The expected value of $X \sim Exp\brak{\lambda}$,
    \begin{align}
        E\brak{X}=\frac{1}{\lambda}
    \end{align}
    The median of $X \sim Exp\brak{\lambda}$,
    \begin{align}
        median\brak{X}=\frac{\ln{2}}{\lambda}
    \end{align}
    \begin{align}
        \ln{2}<1\\
        \frac{\ln{2}}{\lambda}<\frac{1}{\lambda}\\
         median\brak{X}<E\brak{X}
    \end{align}
    Hence, option $1$ is correct.
    
    \item Let's consider $X$ has a uniform distribution in interval $[a,b]$,
    \begin{align}
        X \sim U\brak{a,b}
    \end{align}
    where,
    $a=$ lower limit
    
    $b=$ upper limit
    
    Probability function of uniform distribution,
    \begin{align}
        f_X\brak{k}=
        \begin{cases}
            \frac{1}{b-a} & a\leq x\leq b\\
            0 & x < a, x > b
        \end{cases}
    \end{align}
    The expected value of $X \sim U\brak{a,b}$,
    \begin{align}
        E\brak{X}=\frac{1}{2}\brak{a+b}
    \end{align}
    The median of $X \sim U\brak{a,b}$,
    \begin{align}
        median\brak{X}=\frac{1}{2}\brak{a+b}
    \end{align}
    \begin{align}
        E\brak{X}=median\brak{X}
    \end{align}
    Hence, option $2$ is incorrect.
    
    \item Let's consider $X$ has a binomial distribution,
    \begin{align}
        X \sim B\brak{n,p}
    \end{align}
    where,
    $n=$ no. of trails
    
    $p=$ success parameter
    
    Probability function of binomial distribution,
    \begin{align}
        f_X\brak{k}=
        \begin{cases}
            {^n C_k}p^k(1-p)^{n-k} & 0\leq k\leq n\\
            0 & otherwise
        \end{cases}
    \end{align}
    The expected value of $X \sim B\brak{n,p}$,
    \begin{align}
        E\brak{X}=np
    \end{align}
    The variance of $X \sim B\brak{n,p}$,
    \begin{align}
        V\brak{X}=\sigma^2=n p(1-p)
    \end{align}
    \begin{align}
        1-p\leq1\\
        n p(1-p)\leq n p\\
        V\brak{X}\leq E\brak{X}
    \end{align}
    Hence, option $3$ is incorrect.
    
    \item Let's consider $X$ has a normal distribution,
    \begin{align}
        X \sim N\brak{\mu,\sigma^2}
    \end{align}
    where,
    $\mu=$ mean of distribution
    
    $\sigma^2=$ variance
    
    Probability function of normal distribution,
    \begin{align}
        f_X\brak{k}=\frac{1}{\sigma\sqrt{2\pi}}e^{-\brak{\frac{x-\mu}{2\sigma}}^2}
    \end{align}
    The expected value of $X \sim N\brak{\mu,\sigma^2}$,
    \begin{align}
        E\brak{X}=\mu
    \end{align}
    The variance of $X \sim N\brak{\mu,\sigma^2}$,
    \begin{align}
        V\brak{X}=\sigma^2
    \end{align}
    $E\brak{X}$ and $V\brak{X}$ are user defined. So, they can take any value.
    
    Hence, option $4$ is incorrect.
    \end{enumerate}
    
%
\item $A$ and $B$ play a game of tossing a fair coin. $A$ starts the game by tossing the coin once and $B$ then tosses the coin twice, followed by $A$ tossing the coin once and $B$ tossing the coin twice and this continues until a head turns up. Whoever gets the first head wins the game. Then, 
\begin{enumerate}
    \item $P(B$ Wins) $> P(A$ Wins)
    \item $P(B$ Wins) $= 2P(A$ Wins)
    \item $P(A$ Wins) $> P(B$ Wins)
    \item $P(A$ Wins) $= 1-P(B$ Wins)
\end{enumerate}
%
%
\solution
Given, a fair coin is tossed till heads turns up.
\begin{align}
\tag{104.1}
\label{dec2016-104eq:0}
    p=\dfrac{1}{2},q=\dfrac{1}{2}
\end{align}
Let's define a Markov chain $\{X_{n},n=0,1,2,\dots\}$, where $X_{n}\in S=\{1,2,3,4,5\}$, such that
\begin{table}[h!]
\centering
\caption{States and their notations}
\label{dec2016-104table:1}
\begin{tabular}{|c|c|}
    \hline
    Notation & State \\
    \hline
    $S=1$ & $A$'s turn\\[1ex]
    \hline
    $S=2$ & $B$'s first turn\\[1ex]
    \hline
    $S=3$ & $B$'s second turn\\[1ex]
    \hline
    $S=4$ & $A$ wins\\[1ex]
    \hline
    $S=4$ & $B$ wins\\[1ex]
    \hline
\end{tabular}
\end{table}
The state transition matrix for the Markov chain is
\begin{align}
\tag{104.2}
\label{dec2016-104eq:p}
    P=\begin{blockarray}{cccccc}
&1 & 2 & 3 & 4 & 5 \\
\begin{block}{c[ccccc]}
  1 & 0 & 0.5 & 0 & 0.5 & 0 \\
  2 & 0 & 0 & 0.5 & 0 & 0.5 \\
  3 & 0.5 & 0 & 0 & 0 & 0.5 \\
  4 & 0 & 0 & 0 & 1 & 0 \\
  5 & 0 & 0 & 0 & 0 & 1 \\
\end{block}
\end{blockarray}
\end{align}
Clearly, the states $1,2,3$ are transient, while $4,5$ are absorbing. The standard form of a state transition matrix is
\begin{align}
\tag{104.3}
\label{dec2016-104eq:std}
    P=\begin{blockarray}{ccc}
&A & N \\
\begin{block}{c[cc]}
  A & I & O  \\
  N & R & Q \\
\end{block}
\end{blockarray}
\end{align}
where,
\begin{table}[h!]
\centering
\caption{Notations and their meanings}
\label{dec2016-104table:2}
\begin{tabular}{|c|c|}
    \hline
    Notation & Meaning \\
    \hline
    $A$ & All absorbing states\\[1ex]
    \hline
    $N$ & All non-absorbing states\\[1ex]
    \hline
    $I$ & Identity matrix\\[1ex]
    \hline
    $O$ & Zero matrix\\[1ex]
    \hline
    $R,Q$ & Other submatices\\[1ex]
    \hline
\end{tabular}
\end{table}
Converting \eqref{dec2016-104eq:p} to standard form, we get
\begin{align}
\tag{104.4}
\label{dec2016-104eq:pstd}
    P=\begin{blockarray}{cccccc}
&4 & 5 & 1 & 2 & 3 \\
\begin{block}{c[ccccc]}
  4 & 1 & 0 & 0 & 0 & 0 \\
  5 & 0 & 1 & 0 & 0 & 0 \\
  1 & 0.5 & 0 & 0 & 0.5 & 0 \\
  2 & 0 & 0.5 & 0 & 0 & 0.5 \\
  3 & 0 & 0.5 & 0.5 & 0 & 0 \\
\end{block}
\end{blockarray}
\end{align}
From \eqref{dec2016-104eq:pstd},
\begin{align}
\tag{104.5}
\label{dec2016-104eq:r,q}
    R=\begin{bmatrix}
    0.5 & 0\\
    0 & 0.5\\
    0 & 0.5\\
    \end{bmatrix},
    Q=\begin{bmatrix}
    0 & 0.5 & 0\\
    0 & 0 & 0.5\\
    0.5 & 0 & 0\\
    \end{bmatrix}
\end{align}
\newpage
The limiting matrix for absorbing Markov chain is
\begin{align}
\tag{104.6}
\label{dec2016-104eq:pbar}
    \bar P=\begin{bmatrix}
    I & O\\
    FR & O\\
    \end{bmatrix}
\end{align}
where,
\begin{align}
\tag{104.7}
\label{dec2016-104eq:f}
    F=(I-Q)^{-1}
\end{align}
is called the fundamental matrix of $P$. \\
On solving, we get
\begin{align}
\tag{104.8}
\label{dec2016-104eq:ans}
    \bar P=\begin{blockarray}{cccccc}
&4 & 5 & 1 & 2 & 3 \\
\begin{block}{c[ccccc]}
    4 & 1 & 0 & 0 & 0 & 0 \\
    5 & 0 & 1 & 0 & 0 & 0 \\
    1 & 0.5714 & 0.4285 & 0 & 0 & 0 \\
    2 & 0.1428 & 0.8571 & 0 & 0 & 0 \\
    3 & 0.2857 & 0.7142 & 0 & 0 & 0 \\
   \end{block}
\end{blockarray}
\end{align}
A element $\bar p_{ij}$ of $\bar P$ denotes the absorption probability in state $j$, starting from state $i$. Then,
\begin{enumerate}
    \item $Pr(A$ wins)=$\bar p_{14}\approx0.5714$
    \item $Pr(B$ wins)=$\bar p_{15}\approx0.4285$
\end{enumerate}
\begin{align}
\tag{104.9}
\therefore \bar p_{14} > \bar p_{15}
\end{align}
Also, in $\bar P$, all the terms in every row should sum to 1.
\begin{align}
\tag{104.10}
\Rightarrow \bar p_{14} + \bar p_{15}+ 0+0+0=1\\
\tag{104.11}
\therefore \bar p_{14}=1-\bar p_{15}
\end{align}
Therefore, options $3),4)$ are correct.\\
\begin{figure}[h]
\caption*{\textbf{Markov chain diagram}}
\centering
\begin{tikzpicture}
    % Setup the style for the states
        \tikzset{node style/.style={state, 
                                    minimum width=1.5cm,
                                    line width=1mm,
                                    fill=gray!20!white}}
        % Draw the states
        \node[node style] at (3, -4)      (bull)     {1};
        \node[node style] at (0, -8)      (bear)     {2};
        \node[node style] at (6, -8) (stagnant) {3};
        \node[node style] at (3, 0) (over1) {4};
        \node[node style] at (3, -12) (over2) {5};
        % Connect the states with arrows
        \draw[every loop,
              auto=right,
              line width=0.7mm,
              >=latex,
              draw=orange,
              fill=orange]
            (stagnant)     edge[bend right=20]            node {$\dfrac{1}{2}$} (bull)
            (stagnant)     edge[bend left=20]            node {$\dfrac{1}{2}$} (over2)
            (bull)     edge[bend right=20] node {$\dfrac{1}{2}$} (bear)
            (bull)     edge node {$\dfrac{1}{2}$} (over1)
            (bear)     edge[bend right=20] node {$\dfrac{1}{2}$} (over2)
            (bear)     edge node {$\dfrac{1}{2}$} (stagnant)
            (over1) edge[loop above]             node  {1} (over1)
            (over2) edge[loop below]             node  {1} (over2);
\end{tikzpicture}
\end{figure}
%

%\item Consider a Markov Chain with state space $\cbrak{0,1,2}$ and transition matrix
%\begin{align}
%P = 
%\begin{blockarray}{c@{\hspace{1pt}}rrr@{\hspace{3pt}}}
%         & 0   & 1   & 2 \\
%        \begin{block}{r@{\hspace{3pt}}@{\hspace{1pt}}
%    (@{\hspace{1pt}}rrr@{\hspace{1pt}}@{\hspace{1pt}})}
%        0 & \frac{1}{2} & \frac{1}{2} & 0  \\
%        1 & 0 &\frac{1}{2}  & \frac{3}{4}  \\
%%
%        2 &  \frac{1}{3} & \frac{1}{3} & \frac{1}{3}  \\
%        \end{block}
%    \end{blockarray}
%\end{align}
%For any two states $i$ and $j$, let $p_{ij}^{(n)}$ denote the $n$-step transition probability of going from $i$ to $j$.  Identify correct statements.
%\begin{enumerate}
%\item $\lim_{n \to \infty} p_{11}^{(n)} = \frac{2}{9}$
%\item $\lim_{n \to \infty} p_{21}^{(n)} = 0$
%\item $\lim_{n \to \infty} p_{32}^{(n)} = \frac{1}{3}$
%\item $\lim_{n \to \infty} p_{13}^{(n)} = \frac{1}{3}$
%\end{enumerate}

\end{enumerate}
