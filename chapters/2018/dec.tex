\renewcommand{\theequation}{\theenumi}
\renewcommand{\thefigure}{\theenumi}
\begin{enumerate}[label=\thesection.\arabic*.,ref=\thesection.\theenumi]
\numberwithin{equation}{enumi}
\numberwithin{figure}{enumi}
\numberwithin{table}{enumi}

\item Let X and Y be i.i.d random variables uniformly distributed on (0,4).Then \pr{X>Y|X<2Y} is
\begin{enumerate}
    \item 1/3
    \item 5/6
    \item 1/4
    \item 2/3
\end{enumerate}
\solution

The PDF is given by
\begin{align}
   &f_X (x)=f_Y (x)=\nonumber \begin{cases}
         \frac{1}{4}, &\text{if 0 \(< x <\) 4}\\
         0, &\text{otherwise}\\
   \end{cases} 
\end{align}    
The CDF is given by
\begin{align}
   \nonumber& F(x)=\int_{-\infty}^{x} f(x)dx \\ \nonumber
   &F_X (x)=F_Y (x)=\nonumber \begin{cases}
          0, & x\leq 0\\
         \frac{x}{4}, &\text{if 0 \(< x <\) 4}\\
          1, &x\geq4\\
   \end{cases}    
\end{align}
Using definition of conditional probability 
\begin{align}
    &\pr{X>Y|X<2Y}=\frac{\pr{Y < X< 2Y}}{\pr{X<2Y}} \label{eqn1}
\end{align}
Now finding \pr{X<2Y}
\begin{align}
    &\pr{X<2y}=F_X (2y)\\
    \implies& \pr{X<2Y}=\int_{-\infty}^{\infty} f_Y(x) \times F_X (2x)dx\\
    \implies& \pr{X<2Y}=\int_{0}^{2} \frac{x}{8}dx +\int_{2}^{4}\frac{1}{4}dx\\
    \implies& \pr{X<2Y}=\frac{3}{4}=0.75 \label{eqn2}
\end{align}
Now to find \pr{Y<X<2Y}
\begin{align}
    &\pr{y<X<2y}=F_X (2y)- F_X (y) \\
    \implies &\pr{Y<X<2Y}\\ \nonumber 
    &=\int_{-\infty}^{\infty} f_Y (x)( F_X (2x)- F_X(x))dx \\
   \implies &\int_{0}^{2}\frac{1}{4}\brak{\frac{x}{2}-\frac{x}{4}} dx +\int_{2}^{4}\frac{1}{4}\brak{1-\frac{x}{4}} dx\\
   \implies &\pr{Y<X<2Y}=\frac{1}{4}=0.25 \label{eqn3}
\end{align}
Now using \eqref{eqn1},\eqref{eqn2} and \eqref{eqn3}
\begin{align}
    \pr{X>Y|X<2Y}=\frac{1/4}{3/4}=\frac{1}{3}
\end{align}
Hence final solution is option 1) or 1/3 
%
\item Suppose $X$ is a positive random variable with the following probability density function,
\begin{align*}
f(x) = (\alpha x^{\alpha -1} + \beta x^{\beta-1} ) e^{-x^{\alpha}-x^{\beta}} ; x>0
\end{align*}
for $ \alpha >0, \beta >0$.
Then the hazard function of $X$ for some choices of $\alpha$ and $\beta$ can be
\begin{enumerate}
    \item an increasing function.
    \item a decreasing function.
    \item a constant function.
    \item a non monotonic function
\end{enumerate}
%
\solution
\input{solutions/2018/dec/118/Assignment7.tex}
%
\item Suppose n units are drawn from a population of N units sequentially as follows. A random sample
\begin{align}
    U_1, U_2, ... U_N \text{ of size N, drawn from }U\brak{0, 1} 
\end{align} 
The k-th population unit is selected if 
\begin{align}
    U_k<\frac{n - n_k}{N-k+1}, k = 1, 2, ..N. \text{where, } n_1=0, n_k = 
\end{align}
number of units selected out of first k-1 units for each k = 2, 3, ..N. Then,
\begin{enumerate}
    \item The probability of inclusion of the second unit in the sample
    \begin{align}
        \text{ is } \frac{n}{N}
    \end{align}
    \item The probability of inclusion of the first and the second unit in the sample
    \begin{align}
        \text{ is } \frac{n \brak{n-1}}{N \brak{N-1}}
    \end{align}
    \item The probability of not including the first and including the second unit in the sample
    \begin{align}
        \text{ is } \frac{n \brak{N-n}}{N \brak{N-1}}
    \end{align}
    \item The probability of including the first and not including the second unit in the sample
    \begin{align}
        \text{ is } \frac{n \brak{n-1}}{N \brak{N-1}}
    \end{align}
\end{enumerate}
%
\solution
\input{solutions/2018/dec/116.tex}
%
\item Consider a Markov chain with state space {1,2,....,100}. Suppose states 2i and 2j communicate with each other and states 2i-1 and 2j-1 communicate with each other for every i,j = 1,2,...,50. Further suppose that $p^{(2)}_{3,3}$ > 0,$p^{(3)}_{4,4}$ > 0 and $p^{(7)}_{2,5}$ > 0. Then 
\begin{enumerate}
\item The Markov chain is irreducible.
\item The Markov chain is aperiodic.
\item State 8 is recurrent.
\item State 9 is recurrent.
\end{enumerate}
%
\solution
\input{solutions/2014/dec/106/LaTex/Assignment_7.tex}
%
\item 


% \item Consider a Markov Chain with state space $\cbrak{0,1,2}$ and transition matrix
% \begin{align}
% P = 
% \begin{blockarray}{c@{\hspace{1pt}}rrr@{\hspace{3pt}}}
%          & 0   & 1   & 2 \\
%         \begin{block}{r@{\hspace{3pt}}@{\hspace{1pt}}
%     (@{\hspace{1pt}}rrr@{\hspace{1pt}}@{\hspace{1pt}})}
%         0 & \frac{1}{2} & \frac{1}{2} & 0  \\
%         1 & 0 &\frac{1}{2}  & \frac{3}{4}  \\
% %
%         2 &  \frac{1}{3} & \frac{1}{3} & \frac{1}{3}  \\
%         \end{block}
%     \end{blockarray}
% \end{align}
% For any two states $i$ and $j$, let $p_{ij}^{(n)}$ denote the $n$-step transition probability of going from $i$ to $j$.  Identify correct statements.
% \begin{enumerate}
% \item $\lim_{n \to \infty} p_{11}^{(n)} = \frac{2}{9}$
% \item $\lim_{n \to \infty} p_{21}^{(n)} = 0$
% \item $\lim_{n \to \infty} p_{32}^{(n)} = \frac{1}{3}$
% \item $\lim_{n \to \infty} p_{13}^{(n)} = \frac{1}{3}$
% \end{enumerate}
% \solution
% \input{solutions/2018/dec/106/solution.tex}

\end{enumerate}
