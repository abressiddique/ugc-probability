\renewcommand{\theequation}{\theenumi}
\renewcommand{\thefigure}{\theenumi}
\begin{enumerate}[label=\thesection.\arabic*.,ref=\thesection.\theenumi]
\numberwithin{equation}{enumi}
\numberwithin{figure}{enumi}

\item 	Let $\vec{A}$ be a $4 \times 4$ matrix. Suppose that the null space $N(\vec{A})$ of $\vec{A}$ is
	\begin{align}
		\cbrak{(x,y,z,w) \in \vec{R}^4 : x + y + z = 0 , x + y + w = 0}
	\end{align}
Then which one of the following is correct
\begin{enumerate}
\item  dim(column space$(\vec{A})) = 1$ 
\item dim(column space$(\vec{A})) = 2$
\item rank$(\vec{A}) = 1$
\item $\vec{S}= \cbrak{(1, 1, 1, 0), (1, 1, 0, 1)}$ is a basis of $N(\vec{A})$
\end{enumerate}
%
%
\solution
The nullspace is given by 
\begin{align}
	\myvec{1 & 1 & 1 & 0 \\ 1 & 1 & 0 & 1\\ 0 & 0 & 0 & 0\\0 & 0 & 0 & 0}\myvec{x\\y\\z\\w} = \myvec{0 \\ 0 \\ 0 \\ 0}
\end{align}	
Row reducing the above matrix we get,
\begin{align}
	\myvec{1 & 1 & 1 & 0 \\ 1 & 1 & 0 & 1\\ 0 & 0 & 0 & 0\\0 & 0 & 0 & 0}
	\xleftrightarrow[R_2 \leftarrow R_2 \times -1]{R_2 \leftarrow R_2 - R_1}
	\myvec{1 & 1 & 1 & 0 \\ 0 & 0 & 1 & -1\\ 0 & 0 & 0 & 0\\0 & 0 & 0 & 0}\\
	\xleftrightarrow{R_1 \leftarrow R_1- R_2}
	\myvec{1 & 1 & 0 & 1 \\ 0 & 0 & 1 & -1\\ 0 & 0 & 0 & 0\\0 & 0 & 0 & 0} \label{eq:solutions/2017/dec/27/eq:rref}
\end{align}
See Table \ref{eq:solutions/2017/dec/27/tab}

\begin{table*}[!ht]
	\begin{tabular}{|m{4.5cm}|l|}
		\hline
		&\\
		dim(C$(\vec{A})) = 1$ 
		& \textbf{False}. Because the number of pivot variables are 2 as obtained in \eqref{eq:solutions/2017/dec/27/eq:rref}\\
		&\\
		\hline
		&\\
		dim(C$(\vec{A})) = 2$
		& \textbf{True}. Since the number of pivot variables are 2, the rank of $\vec{A}$ is 2.\\
		&$\therefore dim(C(\vec{A})) = 2 \quad [\because dim(C(\vec{A})) = rank(\vec{A})]$ \\
		&\\
		\hline
		&\\
	     rank$(\vec{A}) = 1$
		& \textbf{False}. Because the rank$(\vec{A}) = 2$, as the number of pivot variables are 2\\
		&\\
		\hline
		&\\
		$\vec{S}$ = $\cbrak{(1, 1, 1, 0), (1, 1, 0, 1)}$ is a basis of $N(\vec{A})$
		& \textbf{False}. \\
		& Let, \\
		&  $\vec{u} = \myvec{1\\1\\1\\0}, \vec{v} = \myvec{1\\1\\0\\1}$\\ 
		&Consider, \\
		&$\myvec{1 & 1 & 1 & 0 \\ 1 & 1 & 0 & 1\\ 0 & 0 & 0 & 0\\0 & 0 & 0 & 0}\myvec{1\\1\\1\\0} = \myvec{3\\2\\0\\0} \not = \myvec{0\\0\\0\\0}$\\
		& Similarly,\\
		&$\myvec{1 & 1 & 1 & 0 \\ 1 & 1 & 0 & 1\\ 0 & 0 & 0 & 0\\0 & 0 & 0 & 0}\myvec{1\\1\\0\\1} = \myvec{2\\3\\0\\0} \not = \myvec{0\\0\\0\\0}$ \\
		&Hence, the given vectors do not form the basis.\\
		\hline
	\end{tabular}
\caption{}
\label{eq:solutions/2017/dec/27/tab}
\end{table*}

\item Let $\vec{A}$ and $\vec{B}$ be real invertible matrices such that 
\begin{align}
    \vec{AB}=-\vec{BA}\label{eq:eq:solutions/2017/june/28/eq1}.
\end{align}
Then
\begin{enumerate}
    \item trace{$\vec{A}$} = trace($\vec{B}$) = 0
    \item trace{$\vec{A}$} = trace($\vec{B}$) = 1
    \item trace{$\vec{A}$} = 0, trace($\vec{B}$) = 1
    \item trace($\vec{A}$) = 1, trace($\vec{B}$) = 0
\end{enumerate}
%
\solution
See Tables     \ref{eq:solutions/2017/june/28/tab:my_label}
and     \ref{eq:solutions/2017/june/28/tab:my_label1}

\begin{table*}[ht!]
\begin{center}
\begin{tabular}{|m{2cm}|m{6cm}|}\hline
        Definition & Matrix $\vec{A}$ is said to be similar to matrix $\vec{B}$ if there exists matrix $\vec{P}$\\& such that $\vec{A}$ = $\vec{P}\vec{B}\vec{P}^{-1}$\\
        \hline
        Properties& Similar matrices have same eigenvalues\\&
        Sum of eigenvalue of a matrix equals its trace\\&
        From above two properties we can conclude that similar matrices have same trace\\
        \hline
\end{tabular}
\end{center}
\caption{Similar matrices and Properties}
\label{eq:solutions/2017/june/28/tab:my_label}
\end{table*}
\begin{table*}[ht!]
\begin{center}
\begin{tabular}{|m{2.2cm}|m{6.3cm}|}\hline
        trace($\vec{A}$) = 0 trace($\vec{B}$) = 0 & From $\eqref{eq:eq:solutions/2017/june/28/eq1}$ we have {\begin{align*}
            \vec{AB} &= -\vec{BA}\\
            \implies\vec{A} &= \vec{B}(-\vec{A})\vec{B}^{-1}
        \end{align*}}So, matrix $\vec{A}$ and (-$\vec{A}$) are similar.$\therefore$ {\begin{align*}
            trace(\vec{A}) &= trace(-\vec{A})\\
            \implies trace(\vec{A}) &= 0
        \end{align*}}Similarly From $\eqref{eq:eq:solutions/2017/june/28/eq1}$ we have {\begin{align*}
            \vec{AB} &= -\vec{BA}\\
            \implies\vec{B} &= \vec{A}^{-1}(-\vec{B})\vec{A}
        \end{align*}}So, matrix $\vec{B}$ and (-$\vec{B}$) are similar.$\therefore$ {\begin{align*}
            trace(\vec{B}) &= trace(-\vec{B})\\
            \implies trace(\vec{B}) &= 0
        \end{align*}} So this statement is true \\
        \hline
        trace($\vec{A}$) = 1 trace($\vec{B}$) = 1 & From $\eqref{eq:eq:solutions/2017/june/28/eq1}$ we have {\begin{align*}
            \vec{AB} &= -\vec{BA}\\
            \implies\vec{A} &= \vec{B}(-\vec{A})\vec{B}^{-1}
        \end{align*}}So, matrix $\vec{A}$ and (-$\vec{A}$) are similar.$\therefore$ {\begin{align*}
            trace(\vec{A}) &= trace(-\vec{A})\\
            \implies trace(\vec{A}) &= 0.
            \end{align*}} As trace($\vec{A}$) = 0 this statement is false\\
        \hline
        trace($\vec{A}$) = 0 trace($\vec{B}$) = 1 & From $\eqref{eq:eq:solutions/2017/june/28/eq1}$ we have {\begin{align*}
            \vec{AB} &= -\vec{BA}\\
            \implies\vec{B} &= \vec{A}^{-1}(-\vec{B})\vec{A}
        \end{align*}}So, matrix $\vec{B}$ and (-$\vec{B}$) are similar.$\therefore$ {\begin{align*}
            trace(\vec{B}) &= trace(-\vec{B})\\
            \implies trace(\vec{B}) &= 0.
            \end{align*}} As trace($\vec{B}$) = 0 this statement is false\\
        \hline
        trace($\vec{A}$) = 1 trace($\vec{B}$) = 0 & From $\eqref{eq:eq:solutions/2017/june/28/eq1}$ we have {\begin{align*}
            \vec{AB} &= -\vec{BA}\\
            \implies\vec{A} &= \vec{B}(-\vec{A})\vec{B}^{-1}
        \end{align*}}So, matrix $\vec{A}$ and (-$\vec{A}$) are similar.$\therefore$ {\begin{align*}
            trace(\vec{A}) &= trace(-\vec{A})\\
            \implies trace(\vec{A}) &= 0.
            \end{align*}} As trace($\vec{A}$) = 0 this statement is false\\
        \hline
        
    \end{tabular}
    \end{center}
    \caption{Calculation of trace}
    \label{eq:solutions/2017/june/28/tab:my_label1}
\end{table*}

\item Let $\vec{A}$ be an n $\times$ n self-adjoint matrix with eigenvalues $\lambda_1, \cdots, \lambda_2$.
Let,\begin{align} \norm{\vec{X}}_2=\sqrt{\vert\vec{X}_{1}^{2}\vert+\cdots+\vert\vec{X}_{n}^{2}\vert}\end{align} for $\vec{X}$=$(\vec{X}_{1},\cdots,\vec{X}_{n})\in \mathbb{C}^n$. If \begin{align}
p(\vec{A})=a_0\vec{I}+a_1\vec{A}+\cdots+a_n\vec{A}^n
\end{align}
then $sup_{\norm{\vec{X}}_{2}=1}\norm{p(\vec{A})\vec{X}}_2$ is equal to
%
\\
\solution
We know that $\vec{A}$ is a self adjoint matrix and hence $\vec{A}=\vec{A}^{*}$ with eigen values $\lambda_1,\lambda_2,\cdots,\lambda_n$.Now as we are given,
\begin{align}
p(\vec{A})=a_0\vec{I}+a_1\vec{A}+\cdots+a_n\vec{A}^n
\end{align}
then,
\begin{align}
(p(\vec{A}))^{*}=a_0\vec{I}^{*}+a_1\vec{A}^{*}+\cdots+a_n\vec{(A^{*})}^n
\end{align}
Since, $\vec{A}=\vec{A}^{*}$ we can state that,
\begin{align}
p(\vec{A})(p(\vec{A}))^{*}=(p(\vec{A}))^{*}p(\vec{A})
\end{align}
Hence p($\vec{A}$) is a normal matrix. Now using spectral theorem for a normal matrix,
\begin{align}
\norm{p(\vec{A})}_{2}=\rho{(p(\vec{A}))}\end{align}
sup refers to the smallest element that is greater than or equal to every number in the set.Hence, sup of $\norm{p(\vec{A})}_{2}$ will be, 
\begin{align}
=\max\cbrak{\vert\alpha\vert: \alpha \text{ is the eigen value of p(A)}}\\
=\max\{\vert p(\lambda_j)\vert:j=1, 2, \cdots n\}\\
=\max\{\vert a_0+a_1\lambda_j+\cdots+a_n{\lambda_j}^n\vert: j=1, 2, \cdots n\}
\end{align}
Now, to find $sup\norm{p(\vec{A})\vec{X}}_{2}$,\begin{align}
=max\{\vert a_0+a_1\lambda_j+\cdots+a_n{\lambda_j}^n\vert: j=1, 2, \cdots n\}\norm{\vec{X}}_2
\end{align}
Since, we have to find $sup_{\norm{\vec{X}}_{2}=1}$ i.e,
\begin{align} \norm{\vec{X}}_2=\sqrt{\vert\vec{X}_{1}^{2}\vert+\cdots+\vert\vec{X}_{n}^{2}\vert}=1\end{align} 
Hence the final answer will be,
\begin{align}
=max\{\vert a_0+a_1\lambda_j+\cdots+a_n{\lambda_j}^n\vert: j=1, 2, \cdots n\}
\end{align}

%

\item Let $p \brak{x}= \alpha x^2+\beta x + \gamma$ be a polynomial, where $\alpha,\beta,\gamma \epsilon R$. Fix $X_0 \epsilon R$. Let $S=\{\brak{a,b,c}  \epsilon R^3: p \brak{x}= a \brak{x-x_0}^2+b \brak{x-x_0}+ c\}$ for all $x\epsilon R$. Find the number of elements in S is
\begin{enumerate}
    \item 0
    \item 1
    \item Strictly greater than 1 but finite
    \item Infinite
\end{enumerate}
%
\solution
\begin{align}
p \brak{x}= \alpha x^2+\beta x + \gamma\\
\implies p \brak{x}=  \brak{\alpha \beta \gamma}\brak{x^2 x 1}^T \label{eq:solutions/2017/june/30/eq1}
\end{align}
\begin{align*}
S= \{\brak{a,b,c}  \epsilon R^3: p \brak{x}= a \brak{x-x_0}^2+b \brak{x-x_0}+ c\},
\end{align*}
\begin{align}
\forall  \vec{x} \epsilon R \brak{Fix X_0}
\end{align}
\begin{align}
p \brak{x}= \brak{a b c}\brak{(x-x_0)^2 (x-x_0) 1}^T\\
=a \brak{x^2-2x_0x+x_0^2}+b \brak{x-x_0}+c
\end{align}
\begin{align}
= a x^2+\brak{b-2 a x_0}x+ \brak{a{x_0}^2-b x_0+c}\label{eq:solutions/2017/june/30/eq2}
\end{align}
Refer \eqref{eq:solutions/2017/june/30/eq1} and \eqref{eq:solutions/2017/june/30/eq2} and comparing the co-coefficients of powers of x,
\begin{align}
\alpha=a,
\beta=b-2ax_0,
\gamma= a{x_0}^2-b x_0+c\\
a=\alpha ,b= \beta+2\alpha x_0, c=\gamma-\alpha {x_0}^2+\brak{\beta+2\alpha x_0}x_0
\end{align}
Here $\alpha, \beta, \gamma$ and $x_0$ are the real fixed numbers. So $a,b,c$ have unique values.\\
Hence S contain only 1 element.So option 2 is correct


\item Let
\begin{align}
\vec{A}=\myvec{1&0&2\\1&-2&0\\0&0&-3}
\end{align}
and $\vec{I}$ be the $3\times3$ identity matrix. If 
\begin{align}
6\vec{A}^{-1}=a\vec{A}^2+b\vec{A}+c\vec{I} \label{eq:solutions/2017/june/31/eq:1}
\end{align} for $a,b,c \in \mathbb{R}$ then (a,b,c) equals
\begin{enumerate}
\item (1,2,1)\\
\item (1,-1,2)\\
\item (4,1,1)\\
\item (1,4,1)
\end{enumerate}
\solution
Finding the characteristic equation,
\begin{align}
\mydet{\vec{A}-\lambda\vec{I}} =\mydet{1-\lambda&0&2\\1&-2-\lambda&0\\0&0&-3-\lambda}\\
\implies (1-\lambda)(-2-\lambda)(-3-\lambda)=0\\
\implies (\lambda^2+\lambda-2)(-3-\lambda)=0\\
\implies \lambda^3+4\lambda^2+\lambda-6 = 0
\end{align}
Using Cayley-Hamilton Theorem we get,
\begin{align}
\vec{A}^3+4\vec{A}^2+\vec{A}-6\vec{I}=0\\
\implies\vec{A}^3+4\vec{A}^2+\vec{A} = 6\vec{I}\\
\implies\vec{A}(\vec{A}^2+4\vec{A}+\vec{I})= 6\vec{I} \label{eq:solutions/2017/june/31/eq:2}
\end{align}
$\mydet{\vec{A}} =6 \neq 0$ hence inverse exists. Hence \eqref{eq:solutions/2017/june/31/eq:2} we get,
\begin{align}
6\vec{A}^{-1} = \vec{A}^2+4\vec{A}+\vec{I} \label{eq:solutions/2017/june/31/eq:3}
\end{align} 
Comparing \eqref{eq:solutions/2017/june/31/eq:1} and \eqref{eq:solutions/2017/june/31/eq:3} we get,
\begin{align}
a=1 \quad b=4 \quad c=1
\end{align}
Hence $(a,b,c)= (1,4,1)$  


%
\item Find the Eigenvalues of the matrix,
\begin{align}
\vec{A} = \myvec{1 & 1 & 2 \\ 1 & -2 & 5 \\ 2 & 5 & -3 }\label{eq:solutions/2017/june/32/eq:1}
\end{align}
\begin{enumerate}
\item -4, 3, -3
\item 4, 3, 1
\item 4, -4$\pm\sqrt{13}$
\item 4, -2$\pm\sqrt{7}$
\end{enumerate}
%
%
\solution
Using the characteristic equation of the matrix can find the Eigenvalues,
\begin{align}
\mydet{\lambda\vec{I} - \vec{A}} = 0\\[1em]
\implies \mydet{\lambda-1 & -1 & -2 \\ -1 & \lambda+2 & -5 \\ -2 & -5 & \lambda+3} = 0
\end{align}
The expression that is obtained after expanding the determinant and simplifying it is,
\begin{align}
(\lambda-1)(\lambda^{2}+5\lambda-19) - (5\lambda+31) = 0 
\end{align}
Further simplifying this we obtain the cubic equation,
\begin{align}
\lambda^{3}+4\lambda^{2}-29\lambda-12 = 0
\end{align}
Solving this equation, the Eigenvalues obtained are,
\begin{align}
\lambda_{1} = -7.605,\ \lambda_{2} = -0.394 \ and \ \lambda_{3} = 4
\end{align}
Therefore, the Eigenvalues of the given matrix are 4, -4 $\pm\sqrt{13}$ (Option 3)

\item Consider the vector space V of real polynomials of degree less than or equal to n. Fix distinct real numbers $a_0, a_1, \cdots, a_k$. For $p \in V$
\begin{align}
    max\cbrak{\abs{p(a_j)}: 0\leq j \leq k}
\end{align}
defines a norm on V
\begin{enumerate}
    \item only if $k<n$
    \item only if $k\ge n$
    \item if $ k+1\leq n$ 
    \item if $k \ge n+1$
\end{enumerate}
%
\solution
Options 2 and 4 are correct as verified in the table \ref{eq:solutions/2017/june/70/table2}
\begin{table*}[ht!]
\begin{center}
\begin{tabular}{|c|c|}
\hline
\textbf{Properties}&\textbf{Norm $\forall x \in V$}\\
\hline
Positivity & $\norm{x}\ge 0, \norm{x} = 0 \iff x=0 $ \\
\hline
Scalar Multiplication & $\norm{\alpha x} = \abs{\alpha}\norm{x}, \alpha \in F $\\
\hline
Triangle Inequality & $\norm{x+y} \le \norm{x} + \norm{y} $\\
\hline
\end{tabular}
\caption{Properties of Norm}
\label{eq:solutions/2017/june/70/table1}
\end{center}
\end{table*}

\begin{table*}[ht!]
\begin{center}
\begin{tabular}{|c|c|}
\hline
\multicolumn{2}{|c|}{
For $p \in V$ then the norm, 
$max\cbrak{\abs{p(a_j)}: 0 \leq j \leq k}=0 \iff \abs{p(a_j)}_{0 \leq j \leq k}=0$
} \\[3ex]
\hline
\textbf{Conditions} & \textbf{Explanation} \\[0.5ex]
\hline
\text{only if $k < n$} & 
A polynomial doesn't necessarily have $k$ distinct real roots,\\
&i.e., it may have repeated, complex roots. \\
Example:& let $p$ be polynomial of degree $n=2$ and $k=1$ given by:-\\
&  \parbox{12cm}{\begin{align}
    p(x) &= x^2 + 4x + 4 \\
    %p(x) &=0 \implies x=-2, -2 \\
    \abs{p(a_j)}_{0\le j \le 1} &= 0 \implies a_0 = -2, a_1 = -2
\end{align}}\\ 
& but $a_0, a_1, \cdots, a_k$ should be distinct real numbers.\\
& This contradicts the property of Norm. Thus condition fails.
\\ [0.5ex]
\hline
\text{only if $k\ge n$} & 
p is a polynomial of degree $\le$n,\\
& it can't have more than $n$ roots and is only possible when,\\
&$p(x)=0 \implies \abs{p(a_j)}_{0 \leq j \leq k}=0$\\
& hence $p$ is identically zero. Thus condition satisfies.
\\ [0.5ex]
\hline
\text{if $k+1 \leq n$} & 
Not a norm for $k<n$. Hence incorrect. 
\\ [0.5ex]
\hline
\text{if $k \ge n+1$} &
Norm for $k \ge n$. Hence correct.
\\[0.5ex]
\hline
\end{tabular}
\caption{Verifying Positivity Property of Norm}
\label{eq:solutions/2017/june/70/table2}
\end{center}
\end{table*}

The scalar multiplication and triangle inequality properties holds true for all $k$.
\begin{align}
    &max\cbrak{\abs{\alpha p(a_j)}} = \abs{\alpha}max\cbrak{\abs{p(a_j)}}\\
    &max\cbrak{\abs{p(a_i)+p(a_j)}} \le max\cbrak{\abs{p(a_i)}} + max\cbrak{\abs{p(a_j)}}
\end{align}
The positivity property holds true only if $k \ge n$ as more than $n$ roots are possible when, 
\begin{align}
    p(x) &= 0 \implies \abs{p(a_j)}_{0 \leq j \leq k}=0 
\end{align}
\begin{align}
    \implies max\cbrak{\abs{p(a_j)}: 0 \leq j \leq k}=0
\end{align}




\item Let \textbf{V} be the vector space of polynomials of degree at most 3 in a variable x with coefficients in $\mathbb{R}$. Let \textbf{T}=d/dx be the linear transformation of \textbf{V} to itself given by differentiation.\\

Which of the following are correct?\\
\begin{enumerate}
\item $\vec{T}$ is invertible
\item 0 is an eigenvalue of $\vec{T}$
\item There is a basis with respect to which the matrix of \textbf{T} is nilpotent.
\item The matrix of \textbf{T} with respect to the basis \myvec{1,1+x,1+x+x^2,1+x+x^2+x^3} is diagonal.
\end{enumerate}
\solution
See Tables \ref{eq:solutions/2017/june/71/table0}
, \ref{eq:solutions/2017/june/71/table1}
and \ref{eq:solutions/2017/june/71/table2}.

%
\onecolumn
%
\begin{longtable}{|l|l|}
\hline
\multirow{3}{*}{Nilpotent Matrix} 
& \\
& 1. If  all the eigen values of matrix is zero then it is said to nilpotent matrix \\
& 
2. Determinant and trace of nilpotent matrix are always zero.\\
\hline
\multirow{3}{*}{Invertible Matrix } & \\
&
A matrix is said to be invertible matrix if its determinant is non zero.\\
\hline
\multirow{3}{*}{Diagonal matrix} & \\
&
diagonal matrix is a matrix in which the entries outside the main diagonal are all zero.\\
\hline
\caption{Definition}
\label{eq:solutions/2017/june/71/table0}
\end{longtable}
\begin{longtable}{|l|l|}
\hline
\multirow{3}{*}{Given       } &\\
& $T$ : $P_3 \xrightarrow{} P_3$ \\ 
& \\
& $T:V\xrightarrow{}V$ be the linear operator given by differentiation wrt $x$\\
& $T(P(x)) \xrightarrow{}$ $P'(x)$ \\
& \\
& $A$ be the matrix of $T$ wrt some basis for $V$ \\
& Assume basis for $V$ be $\{1,x,x^2,x^3\}$ \\
\hline
\caption{Result used}
\label{eq:solutions/2017/june/71/table1}
\end{longtable}
\begin{longtable}{|l|l|}
\hline
\multirow{3}{*}{Checking whether } &\\
& $T:V\rightarrow V$\\matrix of $T$ is nilpotent
& $TP(x) = P'(x)$\\
&Differentiating wrt $x$ to find matrix $A$;\\
& \qquad \qquad \qquad 
$T(1)$ = $0$ = $a_1x+b_1x+c_1x^2+d_1x^3$\\
& \qquad \qquad \qquad 
$T(x)$= $1$ = $a_2+b_2x+c_2x^2+d_2x^3$\\
& \qquad \qquad \qquad 
$T(x^2)$ = $2x$ = $a_3+b_3x+c_3x^2+d_3x^3$\\
& \qquad \qquad \qquad 
$T(x^3)$ = $3x^2$ = $a_4+b_4x+c_4x^2+d_4x^3$\\
& 
Representing $A$ in matrix form ;\\
&
\qquad\qquad\qquad
$A$=$\myvec{0&1&0&0\\0&0&2&0\\0&0&0&3\\0&0&0&0}$\\
&
from the above matrix of $T$ we can say it is nilpotent matrix.\\
\hline
\multirow{3}{*}{ Checking eigen value of matrix $T$ } &\\
&
$A=\myvec{0-\lambda&1&0&0\\0&0-\lambda&2&0\\0&0&0-\lambda&3\\0&0&0&0-\lambda}$\\
&
$\implies \lambda=0$\\
&\\
\hline
\multirow{3}{*}{Checking whether matrix} & \\
&Since $\det{A}$= $0$. \\ of  $T$ is invertible 
&Therefore matrix of $T$ is not invertible \\
&\\
\hline
\multirow{3}{*}{Checking whether Matrix of $T$} & \\
& Let basis be $B'$ = $\{1,1+x,1+x+x^2,1+x+x^2+x^3\}$\\is diagonal matrix
& Differentiating wrt $x$ ;\\
&
$T(1) = 0 = a_1x+b_1(1+x)+c_1(1+x+x^2)+d_1(1+x+x^2+x^3)$\\
&
$T(1+x)= 1 = a_2+b_2(1+x)+c_2(1+x+x^2)+d_2(1+x+x^2x^3)$\\
&
$T(1+x+x^2) = 1+2x = a_3+b_3(1+x)+c_3(1+x+x^2)$\\
&
\qquad\qquad\qquad+$d_3(1+x+x^2+x^3)$\\
&  
$T(1+x+x^2+x^3) = 1+2x+3x^2 = a_4+b_4(1+x)+c_4(1+x+x^2)$\\
&
\qquad\qquad\qquad\qquad+$d_4(1+x+x^2+x^3)$\\
&
\qquad\qquad\qquad $B=\myvec{0&1&-1&-1\\0&0&2&-1\\0&0&0&3\\0&0&0&0}$\\
&
above matrix is not a diagonal matrix\\
&\\
\hline
\multirow{3}{*}{Conclusion} & \\
&
Thus we can conclude \\
&
Option 2) and 3) are correct.\\
&\\
\hline
\caption{Solution}
\label{eq:solutions/2017/june/71/table2}
\end{longtable}

\item Let $m,n,r$ be natural numbers. Let $A$ be an $m\times n$ matrix with real entries such that $(AA^t)^r = I$, where $I$ is the $m \times m$ is identity matrix and $A^t$ is the transpose of the matrix $A$. We can conclude that\\
\begin{enumerate}
\item
$m = n$\\
\item
$AA^t$ is invertible\\
\item
$A^tA$ is invertible\\
\item
if $m=n$, then $A$ is invertible
\end{enumerate}
%
\solution
Options 2) and 4) are correct.  See Table \ref{eq:solutions/2017/june/72/table}
\begin{table}[!ht]
\begin{center}
\begin{tabular}{|c|c|}
\hline
& \\
\textbf{Option} & \textbf{Answer}\\
& \\
\hline
& \\
1) $m=n$ & Let $\vec{A} = \myvec{1 & 0 & 0\\0 & 1 & 0}$ and $r = 1$\\
& $(\vec{AA^T})^r = \myvec{1 & 0\\0 & 1} = I$\\
& Since $m \neq n$\\
& Option 1 is False.\\
& \\ 
\hline
& \\
2) $AA^t$ is invertible & w.k.t det($A^n$) = (det($A$))$^n$\\
 & Since $(AA^t)^r = I$\\
& So det($(AA^T)^r$) = det(I)\\
& (det($AA^T$))$^r$ = 1\\
& $\implies$ det($AA^T$) $\neq 0$\\
& Hence $AA^T$ is invertible\\
& Option 2 is True.\\
& \\
\hline
& \\
3) $A^tA$ is invertible & Let $\vec{A} = \myvec{1 & 0 & 0\\0 & 1 & 0}$ and $r = 1$\\
 & $(\vec{A^TA})^r = \myvec{1 & 0 & 0\\0 & 1 & 0\\0 & 0 & 0}$\\
& But det($AA^T$) = 0.\\
& $\implies$ $AA^T$ is not invertible.\\
& Hence Option 3 is False\\
& \\
\hline
& \\
4) if $m=n$ then $A$ is & Since det($AA^T$) $\neq 0$\\
invertible & det(A).det($A^T$) $\neq 0$\\
 & det(A).det(A) $\neq 0$\\
& $\implies$ $A$ is invertible.\\
& Hence Option 4 is True\\
& \\
\hline
\end{tabular}
\end{center}
\caption{}
\label{eq:solutions/2017/june/72/table}
\end{table}

\item Let $\vec{A}$ be a $n\times n$ real matrix with $\vec{A}^2=\vec{A}$. Then
\begin{enumerate}
	\item the eigenvalues of $\vec{A}$ are either 0 or 1
	\item $\vec{A}$ is a diagonal matrix with diagonal entries 0 or 1
	\item $rank(\vec{A})=trace(\vec{A})$
	\item if $rank(\vec{I-A})=trace(\vec{I-A})$
\end{enumerate}
%
%
\solution
See Table \ref{eq:solutions/2017/june/73/1}
\begin{table*}[!ht]
        \centering
	\begin{tabular}{|m{2.0in}|m{5.0in}|} \hline
		\textbf{Objective} & \textbf{Explanation} \\ \hline
		Eigenvalues of $\vec{A}$ & Since 
		\begin{align}
			\vec{A}^2=\vec{A} \\
		        \implies \vec{A}^2-\vec{A}=\vec{O}
		\end{align} 
From Cayley-Hamilton Theorem we have,
\begin{align}
        \lambda^2-\lambda=0 \\
        \implies \lambda\brak{\lambda-1}=0 \\
        \implies \lambda=0,1
\end{align} 
	A matrix $\vec{A}$ satisfying $\vec{A}^2=\vec{A}$ is an idempotent matrix with eigen values
equal to 0 or 1. 	\\ \hline
Check if $\vec{A}$ is necessary diagonal & Consider
                \begin{align}
                        \vec{A}=\myvec{1&-1\\0&0}\\
                \end{align}
                Then,
                \begin{align}
                        \vec{A}^2=\myvec{1&-1\\0&0}\myvec{1&-1\\0&0}\\
                        =\myvec{1&-1\\0&0} \\
                        =\vec{A}
                \end{align}
                Hence $\vec{A}$ is idempotent but not diagonal. \\ \hline
Relation between rank and trace of $\vec{A}$ & Rank of matrix is defined as the number of non-zero eigenvalues. Since number of non-zero eigenvalues is 1,  
\begin{align} 
	rank(\vec{A})=1 \\
	trace(\vec{A})=\sum_i \lambda_i = 0+1 =1 \\
	\implies rank(\vec{A})=trace(\vec{A})
\end{align} \\ \hline
Relation between rank and trace of $\vec{I}-\vec{A}$ & Now for the matrix $\vec{I}-\vec{A}$ we have,
\begin{align}
	\brak{\vec{I}-\vec{A}}^2 = \brak{\vec{I}-\vec{A}}\brak{\vec{I}-\vec{A}}\\
	= \vec{I}^2-\vec{IA}-\vec{AI}+\vec{A}^2 \\
	= \vec{I}-\vec{A}-\vec{A}+\vec{A} \\
	= \vec{I}-\vec{A}
\end{align}
Hence $\vec{I}-\vec{A}$ is an idempotent matrix. Therefore we conclude,
\begin{align}
        rank(\vec{I}-\vec{A})=trace(\vec{I}-\vec{A})
\end{align} \\ \hline
		Answer& (1),(3) and (4) \\ \hline
        \end{tabular}
        \caption{} \label{eq:solutions/2017/june/73/1}
\end{table*}

\item For any $n\times n$ matrix $B$, let $N(B)=\{X\in \mathbb{R}^n:BX=0\}$ be the null space of $B$. Let $A$ be a $4\times 4$ matrix with $dim(N(A-4I))=2, dim(N(A-2I))=1$ and $rank(A)=3$
Which of the following are true?
\begin{enumerate}
\item 0,2 and 4 are eigenvalues of A
\item determinant(A)=0
\item A is not diagonalizable
\item trace(A)=8
\end{enumerate}
%
\solution
See Table \ref{eq:solutions/2017/june/74/table}.

%
\begin{longtable}{|c|l|}
    \hline
	\multirow{5}{*}{Given} 
	& \\
	& $A$ is a $4\times 4$ matrix.\\
	& $dim\brak{N\brak{A-2I}}=2$,\\
	& $dim\brak{N\brak{A-4I}}=1$, and\\
	& $rank\brak{A}=3$\\
	\hline
	\multirow{3}{*}{Eigenvalues of a matrix} 
	&\\
	& The number $\lambda$ is an eigenvalue of a matrix A if and only if$A-\lambda I$ is singular,\\
	& i.e. $\mydet{A-\lambda I}=0$ \\
	&\\
	& For $\lambda=2$\\
	& Given, $dim\brak{N\brak{A-2I}}=2$\\
	& $\implies nullity(A-2I)=2$\\
	& $rank(A)+nullity(A)=n$\\
	& $\implies rank\brak{A-2I}=4-2=2$\\
	& $\implies \brak{A-2I}$ is not a full rank matrix\\
	& Therefore $\mydet{A-2I}=0$\\
	&\\
	& Also,\\
	& $\implies N\brak{A-2I}=\{X\in \mathbb{R}^4:(A-2I)X=0\}$ \\
	& $\implies (A-2I)X=0$ gives two eigen vectors\\
	& $\implies$ 2 is an eigenvalue of $A$ with multiplicity 2.\\
	&\\
	& Similarly, for $\lambda=4$\\
	& Given, $dim\brak{N\brak{A-4I}}=1$\\
	& $\implies rank\brak{A-4I}=4-1=3$\\
	& $\implies \brak{A-4I}$ is not a full rank matrix\\
	\hline \newpage \hline 
	&\\
	& Therefore $\mydet{A-4I}=0$\\
	& $\implies$ 4 is an eigenvalue of $A$ with multiplicity 1.\\
	&\\
	& For $\lambda=0$\\
	& Given that $rank\brak{A}=3$\\
	& $\implies A$ is not a full rank matrix\\
	& Therefore $\mydet{A}=0$\\
	& $\implies$ 0 is an eigenvalue of $A$ with multiplicity 1.\\
	\hline
	\multirow{3}{*}{Determinant} & \\
	& Given that $rank\brak{A}=3$\\
	& $\implies A$ is not a full rank matrix\\
	& Therefore $\mydet{A}=0$\\
	&\\
	\hline
    \multirow{3}{*}{Diagonalizability} 
    & \\
    & An $n\times n$ matrix $A$ is diagonalizable if and only if $A$ has n linearly independent \\& eigen vectors.\\
	& $rank\brak{A}+nullity\brak{A}=n$\\
	& $\implies$ for $\lambda=0$,\\
	& $nullity\brak{A-\lambda I}=nullity\brak{A}=4-3=1$\\
	& $\implies$ There exists only one linearly independent eigen vector corresponding to \\& $0$ eigen value\\
	& Thus, matrix A is not diagonalizable.\\
	&\\
    \hline
    \multirow{3}{*}{Trace} & \\
	& $Trace\brak{A}$=sum of eigen values\\
	& $\implies Trace\brak{A}=0+2+2+4=8$\\
	&\\
	\hline
	\multirow{3}{*}{Conclusion} & \\
	& Option (1), (2) and (4) are correct\\
	&\\
	\hline
\caption{Solution}
\label{eq:solutions/2017/june/74/table}
\end{longtable}

\item Which of the following 3x3 matrices are diagonizable over $\mathbb{R}?$\\
\begin{enumerate}
    \item \myvec{1&2&3\\0&4&5\\0&0&6}
    \item \myvec{0&1&0\\-1&0&0\\0&0&1}
    \item \myvec{1&2&3\\2&1&4\\3&4&1}
    \item \myvec{0&1&2\\0&0&1\\0&0&0}
\end{enumerate}
%
\solution
See Tables \ref{table:1} and \ref{table:2}


\onecolumn
\begin{longtable}{|l|l|}
\hline
\multirow{3}{*}{Test for diagonalizability} & \\
& Let $\vec{W}_{i}$ be the eigenspace corresponding to eigenvalue $\lambda_{i}$  of $\vec{A}$\\
&\\
& $1)\vec{A}$ is diagonalizable \\
&\\
& $2)$ characteristic polynomial of $\vec{A}$ is \\
&\\
& f = $(\vec{x}-\lambda_1)^{d_1}....(\vec{x}-\lambda_k)^{d_k}$ and $dim(\vec{W}_i) = d_i $\\
&\\
& $3) \sum_{i=1}^{k}\vec{W_i}=n$\\
&\\
\hline
\multirow{3}{*}{Concept} & \\
&
A linear operator $\vec{A}$ on a $n$-dimensional space $\mathbb{V}$ is\\ 
&\\ for diagonalization
& diagonalizable , if and only if $\vec{A}$ has $n$ distinct \\
&\\
& characteristic vectors or null spaces corresponding to the characteristic values\\
\hline
\caption{Illustration of theorem.}
\label{table:1}
\end{longtable}
\begin{longtable}{|l|l|}
\hline
\multirow{3}{*}{Option A} & \\
& Given matrix is  \\
&\\
& $\vec{A}=\myvec{1&2&3\\0&4&5\\0&0&6}$\\
&\\
\hline
\multirow{3}{*}{Finding Characteristics} & \\
&
Characteristics polynomial of the matrix $\vec{A}$ is $det(x\vec{I}-\vec{A})$\\ 
polynomial
& $\det(x\vec{I}-\vec{A})$= $\left|
                \begin{array}{ccc}
                (x-1) & -3 & -2\\
                0 & (x-4) & -5\\
                0 & 0 & x-6
                \end{array} \right|$  \\
&\\
& Characteristic Polynomial = $(x-1)(x-4)(x-6)$\\
&\\
\hline
\multirow{3}{*}{Testing diagonalizability over $\mathbb{R}$} & \\
& 1) As the characteristics  polynomial is product of linear factors\\
&over $\mathbb{R}$ .\\
&\\
&2) To find characteristic values of the operator $\det(xI-A) = 0$ which gives  \\
& $\lambda_1= 1 , \lambda_2= 4, \lambda_3= 6$\\
&\\
& Thus over $\mathbb{R}$ matrix $\vec{A}$ has three distinct characteristic values.\\
&There will be atleast one characteristics vector i.e., one\\ & dimension with each characteristics value .\\
&Thus $dim \vec{W}_i$ = $d_i$\\
&\\
&3) $\sum_{i} \vec{W_i} = n = 3$ , which is equal to $dim$ of $\vec{A}$.\\ 
&\\
\hline
\multirow{3}{*}{Conclusion on Option A} & \\
& Option A satisfy all three condition of Diagonalizability over $\mathbb{R}$. \\
&\\

\hline \hline
\multirow{3}{*}{Option B} & \\
& Given matrix is  \\
&\\
& $\vec{A}$=$\myvec{0&1&0\\-1&0&0\\0&0&1}$\\
&\\
\hline
\multirow{3}{*}{Finding Characteristics} & \\
&
Characteristics polynomial of the matrix $det(x\vec{I}-\vec{A})$\\ 
polynomial
& $det(x\vec{I}-\vec{A})= \left|
                \begin{array}{ccc}
                x & -1 & 0\\
                1 & x & 0\\
                0 & 0 & x-1
                \end{array} \right|$  \\
&\\
& Characteristic Polynomial = $(x-1)(x+i)(x-i)$\\
&\\
\hline
\multirow{3}{*}{Testing diagonalizability over $\mathbb{R}$} & \\
& 1) As the characteristics  polynomial is not the product of linear factors\\
&over $\mathbb{R}$ beacuse roots of characteristic eq are complex .\\ & Thus $\vec{A}$ is not diagonalizable over $\mathbb{R}$.\\

&\\
\hline
\multirow{3}{*}{Conclusion on Option B} & \\
& Option B does not satisfy condition 1. \\
&\\
\hline \hline
\multirow{3}{*}{Option C} & \\
& Given matrix is  \\
&\\
& $\vec{A}$=$\myvec{1&2&3\\2&1&4\\3&4&1}$\\
&\\
\hline
\multirow{3}{*}{Finding Characteristics} & \\
&
Characteristics polynomial of the matrix $\vec{A}$ is $det(x\vec{I}-\vec{A})$\\ 
polynomial
& $det(x\vec{I}-\vec{A})$= $\left|
                \begin{array}{ccc}
                (x-1) & -2 & -3\\
                -2 & (x-1) & -4\\
                -3 & -4 & x-1
                \end{array} \right|$  \\
&\\
& Characteristic Polynomial = $(x+3.19)(x+0.877)(x-7.07)$\\
&\\
\hline
\multirow{3}{*}{Testing diagonalizability over $\mathbb{R}$} & \\
& 1) As the characteristics  polynomial are product of linear factors\\
&over $\mathbb{R}$ .\\
&\\
&2) To find characteristic values of the operator $det(x\vec{I}-\vec{A})=0$ which gives  \\
& $\lambda_1= -3.19 , \lambda_2= -0.887, \lambda_3= 7.07$\\
&\\
& Thus over $\mathbb{R}$ matrix $\vec{A}$ has three distinct characteristic values.\\
&There will be atleast one characteristics vector i.e., one\\ & dimension with each characteristics value .\\
&Thus $dim \vec{W}_i$ = $d_i$\\
&\\
&3) $\sum_{i} \vec{W_i} = n = 3$ , which is equal to $dim$ of $\vec{A}$.\\ 
&\\
\hline
\multirow{3}{*}{Conclusion on Option C} & \\
& Option C satisfy all three condition of Diagonalizability over $\mathbb{R}$. \\
&\\
\hline
%\hline
%\newpage
%\hline
\multirow{3}{*}{Option D} & \\
& Given matrix is  \\
&\\
& $\vec{A}$=$\myvec{0&1&2\\0&0&1\\0&0&0}$\\
&\\
\hline
\multirow{3}{*}{Finding Characteristics} & \\
&Characteristics polynomial of the matrix $\vec{A}$ is $det(x\vec{I}-\vec{A})$\\ 
polynomial
& $det(x\vec{I}-\vec{A})$= $\left|
                \begin{array}{ccc}
                x & -1 & -2\\
                0 & x & -1\\
                0 & 0 & x
                \end{array} \right|$  \\
&\\
& Characteristic Polynomial = $(x)(x)(x)=x^3$\\
&\\
\hline
\multirow{3}{*}{Testing diagonalizability over $\mathbb{R}$} & \\
& 1) As the characteristics  polynomial is product of linear factors\\
&over $\mathbb{R}$ .\\
&\\
&2) To find characteristic values of the operator $\det(x\vec{I}-\vec{A}) = 0$ \\
& $\lambda_1= 0$\\
&$d_1=3$\\
&$\vec{W}_1=\vec{A}-\lambda_1\vec{I}\implies\myvec{0&1&2\\0&0&1\\0&0&0}-0\myvec{1&0&0\\0&1&0\\0&0&1}=\myvec{0&1&2\\0&0&1\\0&0&0}$\\

&$dim \vec{W}_1 = 2$\\
&$dim \vec{W}_i \neq d_i$\\
&Algebric Multiplicity is not equal to Geometric Multiplicity.\\
&\\
\hline
\multirow{3}{*}{Conclusion on Option D} & \\
& Option D  does not satisfy second condition of Diagonalizability.\\
&\\
\hline \hline
\multirow{3}{*}{Answer} & \\
&Option A and Option C are Diagonalizable over $\mathbb{R}$.\\

\hline
\caption{Option Checking Table}
\label{table:2}
\end{longtable}

\twocolumn
\item Let $\vec{A} = \myvec{3 & 1 & 2 \\ 1 & 2 & 3 \\ 2 & 3 & 1  }$ and $\vec{Q(X) = X^TAX}$ for $\vec{X} \in \mathbb{R}^{3}$. Then
\begin{enumerate}
	\item $\vec{A}$ has exactly two positive eigen values.
	\item all the eigen values of $\vec{A}$ are positive.
	\item $\vec{Q(X)} \geq 0 $ $\forall$ $\vec{X}$ $\in$ $\mathbb{R}^3$
	\item $\vec{Q(X)} < 0 $ for some $\vec{X}$ $\in$ $\mathbb{R}^3$
\end{enumerate}
%
%
\solution
See Tables \ref{table:2017/june/77/0} and \ref{table:2017/june/77/1}
\begin{table*}[!ht]
\resizebox{\columnwidth}{!}
{
	\begin{tabular}{|l|l|}
		\hline
		\multirow{3}{*}{Positive Semi} & \\
		& A $n \times n$ symmetric real matrix $\vec{M}$ is said to be positive semi definite if $\vec{x^TMx} \geq 0$ for all  \\
		Definite Matrix	& non-zero $\vec{x}$ in $\mathbb{R}^n$. Formally\\
		& \qquad \qquad  $\vec{M}$ is positive semi-definite $\xLeftrightarrow[]{}$ $\vec{x^TMx} \geq 0$ $\forall$ $\vec{x}$ $\in$ $\mathbb{R}^{n}\backslash\{0\}$\\
		& \\
		\hline
		\multirow{3}{*}{Theorem} & \\
		& For a symmetric $n \times n$ matrix $\vec{M}$ $\in$ $\vec{L(V)}$, following are equivalent.\\
		& \qquad 1). $\vec{x^TMx} \geq 0$ $\forall$ $\vec{x}$ $\in$ $\vec{V}$.\\
		& \qquad 2). All the eigenvalues of $\vec{M}$ are non-negative. \\
		& \\
		\hline
\end{tabular}
}
\caption{Definition and Result used}
\label{table:2017/june/77/0}
\end{table*}	
\begin{table*}[!ht]
\resizebox{\columnwidth}{!}
{
	\begin{tabular}{|l|l|}
		\hline
		\multirow{3}{*}{Calculating eigen} & \\
	     & Given \\
values of $\vec{A}$	& \qquad \qquad \qquad  $\vec{A}= \myvec{3 & 1 &2 \\ 1 & 2 & 3 \\ 2 & 3 &1}$ \qquad \qquad \qquad \qquad \qquad \qquad \qquad \qquad \qquad \qquad \qquad \qquad  \\
         & Calculating, eigen values of $\vec{A}$, ie \\
		 & \qquad \qquad \qquad det($\vec{A-\lambda I}) =0$ \\
		 & \qquad \qquad  $\implies$ $\abs{\myvec{3-\lambda & 1 & 2 \\ 1 & 2-\lambda & 3 \\ 2 & 3 & 1-\lambda}} = 0$ \\
		 & \qquad \qquad $\implies \brak{3-\lambda}\brak{(2-\lambda)(1-\lambda)-9} -1\brak{1-\lambda-6} + 2\brak{3-2(2-\lambda)} = 0$\\
		 & \qquad \qquad $\implies \lambda^3 - 6\lambda^2 -3\lambda + 18 = 0$\\
		 & \qquad \qquad $\implies \lambda_1 = 6$, $ \lambda_2 = \sqrt{3}$ and $\lambda_3 = -\sqrt{3}$\\
		 & Hence, $\vec{A}$ has exactly two positive eigen values.  \\
		 & \\
		\hline
		\multirow{3}{*}{Proving $\vec{x^TAx} < 0$} & \\
		&  Suppose $\vec{x^TAx} \geq 0$ for all $\vec{x}$ $\in$ $\mathbb{R}^{3}$. Then, by theorem above in definition section, matrix $\vec{A}$ \\
for some $\vec{x} \in \mathbb{R}^3$&  is positive semi definite. Hence, all the eigen values of $\vec{A}$      non-negative, but this is not the\\	
using contradiction  & case as one of eigen value is $\lambda_3 = -\sqrt{3}$. So, $\vec{x^TAx} \geq 0$ is not true for all $\vec{x} \in \mathbb{R}^3$. \\
		& Similarly, as $\lambda_i \leq 0 $,$\forall i$ is also not true, so $\vec{x^TAx} \leq 0$ is not true for all $\vec{x} \in \mathbb{R}^3$. \\
		& Thus, $\vec{x^TAx} < 0$ for some $\vec{x} \in \mathbb{R}^3$. \\
		& \\
		\hline
		\multirow{3}{*}{Correct Options} & \\
		& Hence, correct options are $(1)$ and $(4)$.\\
		& \\
		\hline
	\end{tabular}
}
\caption{Solution}\label{table:2017/june/77/1}\end{table*}	

\item Consider the matrix
\begin{align}
A(x) = \myvec{1+x^2&7&11\\3x&2x&4\\8x&17&13} & ;x\in \vec{R}.
\end{align}
Then,
\begin{enumerate}
\item A(x) has eigenvalue 0 for some $x\in \vec{R}$.
\item 0 is not an eigenvalue of A(x) for any $x\in \vec{R}$.
\item A(x) has eigenvalue 0 $\forall x\in \vec{R}$.
\item A(x) is invertible $\forall x\in \vec{R}$.
\end{enumerate}
%
\solution
Let $\lambda = 0$ be an eigenvalue. Hence,
\begin{align}
\mydet{A-\lambda I} = 0\\
\implies \mydet{A} = 0 \\
\implies \mydet{A} = \mydet{1+x^2&7&11\\3x&2x&4\\8x&17&13} = 0
\end{align}
Performing row reduction we get,
\begin{align}
\mydet{1+x^2&7&11\\0&\frac{2x^3 - 19x}{1+x^2}&\frac{4x^2 - 33x +4}{1+x^2}\\0&0&\frac{26x^3 - 244x^2 + 538x - 68}{2x^3 - 19x}} = 0
\end{align}
\begin{align}
\implies 26 x^3 - 244 x^2 + 538 x - 68 = 0\\
\implies x_1 = 6.01, x_2 = 3.23, x_3 = 0.13
\label{eq:solutions/2017/june/78/x}
\end{align} 
%
See Table \ref{table:solutions/2017/june/78/}

\begin{table*}[h!]
\begin{center}
\resizebox{\columnwidth}{!}
{
\begin{tabular}{|c|c|}
\hline
& \\
\textbf{OPTIONS} & \textbf{Explanation}\\
&\\
\hline
& \\
Option (b) & At the Values of x given by \eqref{eq:solutions/2017/june/78/x}, eigen value $\lambda = 0$. \\ &
Hence option (b) can't be correct.\\
&\\
\hline
& \\
Option (c) & If one of the eigenvalue is 0 for A(x) then, $\mydet{A(x)} = 0 \forall x \in R$.\\ & 
But from \eqref{eq:solutions/2017/june/78/x} we have concluded that $\mydet{A} = 0$ only for, 
\\& $x_1 = 6.01, x_2 = 3.23, x_3 = 0.13$.\\ &
Hence, Option (c) is incorrect.\\
& \\
\hline
& \\
Option (d) & Now for the values of x given by \eqref{eq:solutions/2017/june/78/x}, $\mydet{A} = 0$. \\ &
Hence it is not invertible $\forall x\in \vec{R}$ \\&
Hence Option (d) is incorrect. \\ 
& \\
\hline
& \\
Option (a) & Now clearly from above arguments A(x) has eigenvalue 0 for some $x \in R$  \\&
Hence Option (a) is Correct. \\ 
& \\
\hline

\end{tabular}
}
\end{center}
\caption{}
\label{table:solutions/2017/june/78/}
\end{table*}
\begin{comment}
In other words at these values of x, eigenvalue $\lambda = 0$. Hence option (b) is eliminated. Now for these values of x, $\mydet{A} = 0$. Hence it is not invertible $\forall x\in \vec{R} \implies$ option (d) is eliminated. Now, if one of the eigenvalue is 0 for A(x), then $\mydet{A(x)} = 0 \forall x\in \vec{R}$. But from equation \eqref{eq:solutions/2017/june/78/x} we have concluded that $\mydet{A} = 0$ only for $x_1 = 6.01, x_2 = 3.23, x_3 = 0.13$. hence, option (b) is also eliminated. Hence the correct option is (a).
\end{comment} 

\end{enumerate}
