\renewcommand{\theequation}{\theenumi}
\renewcommand{\thefigure}{\theenumi}
\begin{enumerate}[label=\thesection.\arabic*.,ref=\thesection.\theenumi]
\numberwithin{equation}{enumi}
\numberwithin{figure}{enumi}
\numberwithin{table}{enumi}

\item  $N,A_1,A_2\cdots$ are independent real valued random variables such that 
    \begin{align}
        \pr{N=k}=(1-p)p^k,k=0,1,2,3\cdots
    \end{align}
    where $0<p<1$ and $\{A_i:i=1,2,\cdots\}$ is a sequence of independent and identically distributed bounded random variables. Let 
    \begin{align}
        X(w) = 
        \begin{cases}
        0  & \text{if } N(w)=0\\
        \sum_{j=1}^{k} A_j & \text{if } N(w)=k,k=1,2,3\cdots 
        \end{cases}
    \end{align}
    Which of the following are necessarily correct?\\
    \begin{enumerate}
        \item $X$ is a bounded random variable. 
        \item Moment generating function $m_X$ of $X$ is
        \begin{align}
            m_X(t)=\dfrac{1-p}{1-pm_A(t)}, t\in \mathbb{R},
        \end{align}
        where $m_A$ is moment generating function of $A_1$.
        \item Characteristic function $\varphi_X$ of $X$ is
        \begin{align}
            \varphi_X(t)=\dfrac{1-p}{1-p\varphi_A(t)},t\in \mathbb{R},
        \end{align}
        where $\varphi_A$ is the characteristic function of $A_1$.
        \item $X$ is symmetric about 0.
    \end{enumerate}


\item Consider a Markov chain with state space {1,2,....,100}. Suppose states 2i and 2j communicate with each other and states 2i-1 and 2j-1 communicate with each other for every i,j = 1,2,...,50. Further suppose that $p^{(2)}_{3,3}$ > 0,$p^{(3)}_{4,4}$ > 0 and $p^{(7)}_{2,5}$ > 0. Then 
\begin{enumerate}
\item The Markov chain is irreducible.
\item The Markov chain is aperiodic.
\item State 8 is recurrent.
\item State 9 is recurrent.
\end{enumerate}
\solution
\input{solutions/2014/dec/106/LaTex/Assignment_7.tex}
%
\item Suppose $X_1$,$X_2$,$X_3$ and $X_4$ are independent and identically distributed random variables, having density function f. Then,
\begin{enumerate}
\item \pr{X_4 > Max(X_1,X_2) > X_3} = $\frac{1}{6}$
\item \pr{X_4 > Max(X_1,X_2) > X_3} = $\frac{1}{8}$
\item \pr{X_4 > X_3 > Max(X_1,X_2)} = $\frac{1}{12}$
\item \pr{X_4 > X_3 > Max(X_1,X_2)} = $\frac{1}{6}$
\end{enumerate}
%
\solution
The probability density function (pdf) f(x) of a random variable X is defined as the derivative of the cdf F(x): \\
\begin{center}
\begin{math}
    f(x) = \dfrac{d}{d x}F(x).
\end{math}
\end{center} 
It is sometimes useful to consider the cdf F(x) in terms of the pdf f(x):
\begin{center}
    \begin{math}
    F(x) = \int\limits_{-\infty}^x f(t) d t
    \end{math}
\end{center}
The PDF of X is,
\begin{align}
    F_X(x) &= \int\limits_{-\infty}^\infty f(x) d x 
\end{align}
\begin{enumerate}
    \item \pr{X_2 > X_1}
\begin{align}
&= \int\limits_{-\infty}^\infty f_X(x) \int\limits_{-\infty}^x                    f_X(t)\ d t d x \\
&= \int\limits_{-\infty}^\infty f_X(x) F_X(x) d x \\
&= \cfrac{F^{2}_X(x)}{2} \biggr \vert_{-\infty}^{\infty} \\
&= \dfrac{1}{2}.
\end{align}
    \item \pr{X_4>Max(X_1,X_2)>X_3} 
\begin{align}
&= \nonumber \int\limits_{-\infty}^\infty f_X(x) \int\limits_{-\infty}^x f_X(t).\comb{2}{1}. \\ 
&  \sbrak{\int\limits_{-\infty}^t f_X(w) d w} 
   \int\limits_{-\infty}^t f_X(z) d z d t d x \\  
&= \int\limits_{-\infty}^\infty f_X(x) \int\limits_{-\infty}^x 2f_X(t)F^{2}_X(t) d t d x\\
&= \int\limits_{-\infty}^\infty f_X(x). \dfrac{2}{3} F^{3}_X(x) d x \\
&= \dfrac{2}{3}\cfrac{F^{4}_X(x)}{4} \biggr \vert_{-\infty}^{\infty} \\    
&= \dfrac{1}{6}.   
\end{align}
    \item \pr{X_4 > X_3 > Max(X_1,X_2} 
\begin{align}
&= \nonumber \int\limits_{-\infty}^\infty f_X(x) \int\limits_{-\infty}^x f_X(t) 
\int\limits_{-\infty}^t f_X(z). \comb{2}{1}. \\
&  \sbrak{\int\limits_{-\infty}^t f_X(w) d w}\ d z d t d x \\
&= \int\limits_{-\infty}^\infty f_X(x) \int\limits_{-\infty}^x f_X(t) \int\limits_{-\infty}^t 2f_X(z)F_X(t)\ d z d t d x \\
&= \int\limits_{-\infty}^\infty f_X(x) \int\limits_{-\infty}^x f_X(t)F^{2}_X(t) d t d x \\
&= \int\limits_{-\infty}^\infty f_X(x).\dfrac{1}{3}F^{3}_X(x) d x \\
&= \dfrac{1}{3}\cfrac{F^{4}_X(x)}{4} \biggr \vert_{-\infty}^{\infty} \\
&= \dfrac{1}{12}.   
\end{align}
\end{enumerate}
\begin{center}
  $\therefore$ \boxed{\text{\textbf{Option 1,3} are \textbf{correct} answers.}}
\end{center}

\end{enumerate}
