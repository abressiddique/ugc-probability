\renewcommand{\theequation}{\theenumi}
\renewcommand{\thefigure}{\theenumi}
\begin{enumerate}[label=\thesection.\arabic*.,ref=\thesection.\theenumi]
\numberwithin{equation}{enumi}
\numberwithin{figure}{enumi}
\numberwithin{table}{enumi}

\item  $N,A_1,A_2\cdots$ are independent real valued random variables such that 
    \begin{align}
        \pr{N=k}=(1-p)p^k,k=0,1,2,3\cdots
    \end{align}
    where $0<p<1$ and $\{A_i:i=1,2,\cdots\}$ is a sequence of independent and identically distributed bounded random variables. Let 
    \begin{align}
        X(w) = 
        \begin{cases}
        0  & \text{if } N(w)=0\\
        \sum_{j=1}^{k} A_j & \text{if } N(w)=k,k=1,2,3\cdots 
        \end{cases}
    \end{align}
    Which of the following are necessarily correct?\\
    \begin{enumerate}
        \item $X$ is a bounded random variable. 
        \item Moment generating function $m_X$ of $X$ is
        \begin{align}
            m_X(t)=\dfrac{1-p}{1-pm_A(t)}, t\in \mathbb{R},
        \end{align}
        where $m_A$ is moment generating function of $A_1$.
        \item Characteristic function $\varphi_X$ of $X$ is
        \begin{align}
            \varphi_X(t)=\dfrac{1-p}{1-p\varphi_A(t)},t\in \mathbb{R},
        \end{align}
        where $\varphi_A$ is the characteristic function of $A_1$.
        \item $X$ is symmetric about 0.
    \end{enumerate}


\item Consider a Markov chain with state space {1,2,....,100}. Suppose states 2i and 2j communicate with each other and states 2i-1 and 2j-1 communicate with each other for every i,j = 1,2,...,50. Further suppose that $p^{(2)}_{3,3}$ > 0,$p^{(3)}_{4,4}$ > 0 and $p^{(7)}_{2,5}$ > 0. Then 
\begin{enumerate}
\item The Markov chain is irreducible.
\item The Markov chain is aperiodic.
\item State 8 is recurrent.
\item State 9 is recurrent.
\end{enumerate}
\solution
\input{solutions/2014/dec/106/LaTex/Assignment_7.tex}
%
\item Suppose $X_1$,$X_2$,$X_3$ and $X_4$ are independent and identically distributed random variables, having density function f. Then,
\begin{enumerate}
\item \pr{X_4 > Max(X_1,X_2) > X_3} = $\frac{1}{6}$
\item \pr{X_4 > Max(X_1,X_2) > X_3} = $\frac{1}{8}$
\item \pr{X_4 > X_3 > Max(X_1,X_2)} = $\frac{1}{12}$
\item \pr{X_4 > X_3 > Max(X_1,X_2)} = $\frac{1}{6}$
\end{enumerate}
%
\solution
Given joint probability density function of X and Y, marginal probability density functions are as follows:
\begin{align}
    f_X(x) = \int_{-\infty}^{\infty} f(x,y) dy \\[0.4cm]
    f_Y(y) = \int_{-\infty}^{\infty} f(x,y) dx
\end{align}
Calculating $f_X(x)$
\begin{align}
    f_X(x) = & \int_{-\infty}^{\infty} f(x,y) dy \\
    =        & \int_{0}^{x} 6(1-x) dy            
\end{align}
\begin{align}
    f_X(x) =
    \begin{cases}
        6x(1-x) & 0<x<1     \label{june2016-104:0.0.6}\\
        0       & otherwise
    \end{cases}
\end{align}
Calculating $f_Y(y)$
\begin{align}
    f_Y(y) = & \int_{-\infty}^{\infty} f(x,y) dx \\
    =        & \int_{y}^{1} 6(1-x) dx\\
    =        & 6x -3{x}^2 \big|_{y}^{1}\\
    =        & 3 - 6y + 3y^{2}\\
    =        & 3{(y-1)}^{2}      
\end{align}
\begin{align}
    f_Y(y) =
    \begin{cases}
        3{(y-1)}^{2}  & 0<y<1     \label{june2016-104:0.0.12}\\
        0       & otherwise
    \end{cases}
\end{align}
To check whether X and Y are independent, we calculate $f_X(x) \times f_Y(y)$. From \eqref{june2016-104:0.0.6} and \eqref{june2016-104:0.0.12}
\begin{align}
    f_X(x) \times f_Y(y) = &
    \begin{cases}
        18x(1-x){(y-1)}^{2}  \\ \hspace{1.2cm}0<x<1,0<y<1\\
        0 \hspace{1cm} \text{otherwise}
    \end{cases}\\
    \neq & f(x,y)
\end{align}
Since f(x,y) and $f_X(x)\times f_Y(y)$ are different, random variables X and Y are not independent.
\begin{center}
    Options 1 and 2 are correct
\end{center}

\end{enumerate}
