\renewcommand{\theequation}{\theenumi}
\renewcommand{\thefigure}{\theenumi}
\begin{enumerate}[label=\thesection.\arabic*.,ref=\thesection.\theenumi]
\numberwithin{equation}{enumi}
\numberwithin{figure}{enumi}



\item Consider a Markov Chain with state space $\cbrak{0,1,2}$ and transition matrix
\begin{align}
P = 
\begin{blockarray}{c@{\hspace{1pt}}rrr@{\hspace{3pt}}}
         & 0   & 1   & 2 \\
        \begin{block}{r@{\hspace{3pt}}@{\hspace{1pt}}
    (@{\hspace{1pt}}rrr@{\hspace{1pt}}@{\hspace{1pt}})}
        0 & \frac{1}{4} & \frac{5}{8} & \frac{1}{8}  \\[1mm]
        1 & \frac{1}{4} & 0 & \frac{3}{4}  \\[1mm]
        2 &  \frac{1}{2} & \frac{3}{8} & \frac{1}{8}  \\
        \end{block}
    \end{blockarray}
\end{align}
Then which of the following are true?
\begin{enumerate}
\item $\lim_{n \to \infty} p_{12}^{(n)} = 0$
\item $\lim_{n \to \infty} p_{12}^{(n)} = \lim_{n \to \infty} p_{21}^{(n)}$
\item $\lim_{n \to \infty} p_{22}^{(n)} = \frac{1}{8}$
\item $\lim_{n \to \infty} p_{21}^{(n)} = \frac{1}{3}$
\end{enumerate}
%
\item A sample of size $n =2$ is drawn from a population of size $N=4$ using probability proportional to size without replacement scheme , Where the probabilities proportional to size are
\begin{table}[h!]
\resizebox{\columnwidth}{0.6cm}{%
  \begin{tabular}{|c|c|c|c|c|}
    \hline
    i: & 1 & 2 & 3 & 4\\
    \hline
    $p_{i}$ & 0.4 & 0.2 & 0.2 & 0.2\\
    \hline
  \end{tabular}%
} 
   \caption*{Table : Probability vs Size}
\end{table}  
The probability of inclusion of unit (1) in the sample is 
\begin{enumerate}
\begin{multicols}{4}
\setlength\itemsep{2em}
\item $0.4$
\item $0.6$
\item $0.7$
\item $0.75$
\end{multicols}
%

\end{enumerate}
%
\solution
Let $P_{i}(j)$ represent the probability for selecting unit (j) as second unit after selecting  unit (i) 
\begin{align}
    P_{i}(j)&=\frac{p_{j}}{1-p_{i}}
    \label{2019-58:eq:eq2}
\end{align}
Let  $\pr{i,j}$ be probability of selecting sample \{i,j\} ,using \eqref{eq:eq2}  is 
\begin{align}
    \pr{i,j}&=P_{i}(j)+P_{j}(i)
    \\
    &=\brak{p_{i}\times \frac{p_{j}}{1-p_{i}}} + \brak{p_{j}\times \frac{p_{i}}{1-p_{j}}}
    \label{2019-58:eq:eq3}
\end{align}
Total samples(Size $n=2$)are 
\definecolor{green}{RGB}{0 150, 22}
\definecolor{Red}{RGB}{200,60,40}
\definecolor{mycolor}{RGB}{0, 60, 240}
\begin{table}[h!]
\resizebox{\columnwidth}{0.95cm}{%
  \begin{tabular}{|c ||c ||c |c | c| c| c|}
    \hline
    \textcolor{green}{Case }&  \textcolor{Red}{1} & \textcolor{Red}{2} & \textcolor{Red}{3} & \textcolor{Red}{4} & \textcolor{Red}{5} & \textcolor{Red}{6}\\
    \hline
    \textcolor{green}{Sample(size $n=2$)} & \textcolor{mycolor}{\brak{1,2}} & \textcolor{mycolor}{\brak{1,3} }& \textcolor{mycolor}{\brak{1,4} }& \textcolor{mycolor}{\brak{2,3}} & \textcolor{mycolor}{\brak{2,4}}& \textcolor{mycolor}{\brak{3,4}}\\
    \hline
  \end{tabular}%
} 
  \caption{ list of samples}
  \label{2019-58:tab:label1_test}
\end{table}
Let $P_{i}$ be the probability of inclusion of unit (i) in the sample(size $n=2$),Now i will calculate $P_{1}$ ,Favourable cases for inclusion of unit(1) are case (\textcolor{red}{1,2,3}),So
\begin{align}
    P_{1}&=\pr{1,2} +\pr{1,3}+\pr{1,4}
\end{align}
using \eqref{2019-58:eq:eq3} and $p_{i}$ from question ,
\begin{align}
    P_{1}&=\frac{7}{30} + \frac{7}{30} + \frac{7}{30}
    \\
    &=0.7
\end{align}
Therefore Option (3) is correct.

%
\item Consider the function \textit{f(x)} defined as \textit{f(x)} = \textit{ce$^{-x^{4}}$}, $\textit{x} \in R$ . For what value of \textit{c} is \textit{f} a probability density function?\\
\begin{enumerate}
    \item $\displaystyle\frac{2}{\Gamma(1/4)}$\\\label{2019/52/option 1}
    \item $\displaystyle\frac{4}{\Gamma(1/4)}$\\
    \item $\displaystyle\frac{3}{\Gamma(1/3)}$\\
    \item $\displaystyle\frac{1}{4\Gamma(4)}$
\end{enumerate}
%
\solution
Consider a continuous random variable X so that the function \textit{f} can be probability density function if and only if it satisfies the condition 
\begin{align}
    \int_{-\infty}^{\infty}f_{X}(u)du = 1 \label{2019/52/equation 1}
\end{align}
Hence by applying the \eqref{2019/52/equation 1} for the function \textit{f} we get
\begin{align}
    \int_{-\infty}^{\infty}ce^{-u^{4}}du = 1\\
    2c\int_{0}^{\infty}e^{-u^{4}}du = 1\label{2019/52/equation 2}\\
    2c\int_{0}^{\infty}e^{-t}\frac{dt}{4t^{\frac{3}{4}}} = 1\\
    \frac{c}{2}\int_{0}^{\infty}e^{-t}t^{-\frac{3}{4}}dt = 1\label{2019/52/equation 3}
\end{align}
We know that gamma function for any real positive $\alpha$
\begin{align}
    \Gamma(\alpha) = \int_0^\infty x^{\alpha - 1} e^{-x} dx \label{2019/52/gammafunction}
\end{align}
Hence by using \eqref{2019/52/gammafunction} in \eqref{2019/52/equation 3} we get
\begin{align}
    \frac{c}{2}\Gamma(1/4) = 1\\
    c=\frac{2}{\Gamma(1/4)}
\end{align}
Hence $c = \displaystyle\frac{2}{\Gamma(1/4)}$ and option \eqref{2019/52/option 1} is correct.\newline\newline
The CDF of \textit{f} by using \eqref{2019/52/gammafunction} we get
\begin{align}
    F_{X}(x) &= \int_{0}^{x}f(u)du\\
             &= \frac{2}{\Gamma(\frac{1}{4})}\int_{0}^{x}e^{-u^{4}}du\\
             &= \frac{2}{4\Gamma(\frac{1}{4})}\int_{0}^{x^{4}}e^{-t}{t^{\frac{-3}{4}}}dt\\
             &= \frac{1}{2\Gamma(\frac{1}{4})}\int_{0}^{x^{4}}e^{-t}{t^{\frac{-3}{4}}}dt\\
             &= \frac{1}{2\Gamma(\frac{1}{4})}\brak{\Gamma\brak{\frac{1}{4}}-\Gamma\brak{\frac{1}{4},x^{4}}}\\
             &= \frac{1}{2\Gamma(\frac{1}{4})}\gamma\brak{\frac{1}{4},x^{4}}
\end{align}

\end{enumerate}

