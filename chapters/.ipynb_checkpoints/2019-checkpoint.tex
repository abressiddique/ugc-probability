\renewcommand{\theequation}{\theenumi}
\renewcommand{\thefigure}{\theenumi}
\begin{enumerate}[label=\thesection.\arabic*.,ref=\thesection.\theenumi]
\numberwithin{equation}{enumi}
\numberwithin{figure}{enumi}



\item Consider a Markov Chain with state space $\cbrak{0,1,2}$ and transition matrix
\begin{align}
P = 
\begin{blockarray}{c@{\hspace{1pt}}rrr@{\hspace{3pt}}}
         & 0   & 1   & 2 \\
        \begin{block}{r@{\hspace{3pt}}@{\hspace{1pt}}
    (@{\hspace{1pt}}rrr@{\hspace{1pt}}@{\hspace{1pt}})}
        0 & \frac{1}{4} & \frac{5}{8} & \frac{1}{8}  \\[1mm]
        1 & \frac{1}{4} & 0 & \frac{3}{4}  \\[1mm]
        2 &  \frac{1}{2} & \frac{3}{8} & \frac{1}{8}  \\
        \end{block}
    \end{blockarray}
\end{align}
Then which of the following are true?
\begin{enumerate}
\item $\lim_{n \to \infty} p_{12}^{(n)} = 0$
\item $\lim_{n \to \infty} p_{12}^{(n)} = \lim_{n \to \infty} p_{21}^{(n)}$
\item $\lim_{n \to \infty} p_{22}^{(n)} = \frac{1}{8}$
\item $\lim_{n \to \infty} p_{21}^{(n)} = \frac{1}{3}$
\end{enumerate}
%
\item A sample of size $n =2$ is drawn from a population of size $N=4$ using probability proportional to size without replacement scheme , Where the probabilities proportional to size are
\begin{table}[h!]
\resizebox{\columnwidth}{0.6cm}{%
  \begin{tabular}{|c|c|c|c|c|}
    \hline
    i: & 1 & 2 & 3 & 4\\
    \hline
    $p_{i}$ & 0.4 & 0.2 & 0.2 & 0.2\\
    \hline
  \end{tabular}%
} 
   \caption*{Table : Probability vs Size}
\end{table}  
The probability of inclusion of unit (1) in the sample is 
\begin{enumerate}
\begin{multicols}{4}
\setlength\itemsep{2em}
\item $0.4$
\item $0.6$
\item $0.7$
\item $0.75$
\end{multicols}
%

\end{enumerate}
%
\solution
Let N be the population size so, N=120. The given sample size is n.
\textbf{Notations :}
y : student under consideration.
$y_i$ : Maths marks of $i^{th}$ student in the sample.
Y : student of class.
$Y_i$ : Maths marks of $i^{th}$ student in the class.
$\overline{y}=\dfrac{1}{n}\sum_{i=1}^{n}y_i$ : Average of sample class.
$\overline{Y}=\dfrac{1}{N}\sum_{i=1}^{N}Y_i$ : Average of whole class.
$S^2 = \dfrac{1}{N-1} \sum_{i=1}^{N} (Y_i-\bar{Y})^2$ : S=Std dev of the class.
$\sigma^2=\dfrac{1}{N} \sum_{i=1}^{N} (Y_i-\bar{Y})^2$ : Variance of the class.
Standard error of sample mean $SE_{mean}=\dfrac{s}{\sqrt{n}}$.\\
Where 
\begin{align*}
s & = \text{standard deviation of sample mean.}\\
n & = \text{sample class size.}
\end{align*}
\textbf{Variance of the $\overline{y}$}
\begin{align}
& V(\overline{y})= E(\overline{y}-\overline{Y})^2\\
& = E\left[\dfrac{1}{n} \sum_{i=1}^{n}(y_i-\overline{Y})\right]^2\\
& = E\left[\dfrac{1}{n^2} \sum_{i=1}^{n} (y_i-\overline{Y})^2 + \dfrac{1}{n^2} \underset{1\leq i\neq j\leq n}{\sum\sum}\, (y_i-\overline{Y})(y_j-\overline{Y})\right]\\
& = \dfrac{1}{n^2}\sum_{i=1}^{n} E(y_i-\overline{Y})^2+\dfrac{1}{n^2} \underset{1\leq i\neq j\leq n}{\sum\sum}\, E(y_i-\overline{Y})(y_j-\overline{Y})\\
& \text{Let } K=\underset{1\leq i\neq j\leq n}{\sum\sum}\, E(y_i-\overline{Y})(y_j-\overline{Y})\\
& = \dfrac{1}{n^2}\sum_{i=1}^{n} \sigma^2 + \dfrac{K}{n^2}\\
& = \dfrac{1}{n^2} n \sigma^2 +\dfrac{K}{n^2}\\
& = \dfrac{N-1}{Nn} S^2+\dfrac{K}{n^2}\label{june2018-58:eq_1}
\end{align}
Finding the value of K in case of Simple random sampling with repetition (SRSWR)and Simple random sampling without repetition(SRSWOR) allows us to calculate the variance of mean.
\vspace{0.5 cm}
\textbf{K value in case of SRSWOR}
\begin{align*}
&K=\underset{1\leq i\neq j\leq n}{\sum\sum}\, E(y_i-\overline{Y})(y_j-\overline{Y})
\end{align*}
Consider
\begin{multline*}
E(y_i-\overline{Y})(y_j-\overline{Y})= \\
\dfrac{1}{N(N-1)}\underset{1\leq k\neq l\leq n}{\sum\sum}\, E(y_k-\overline{Y})(y_l-\overline{Y})
\end{multline*}
Since
\begin{multline*}
\left[\sum_{k=1}^N(y_k-\overline{Y})\right]^2=\sum_{i=1}^{N}(y_k-\overline{Y})^2+\\
\underset{1\leq k\neq l\leq n}{\sum\sum}\, E(y_k-\overline{Y})(y_l-\overline{Y})
\end{multline*}
\begin{align*}
&\implies 0 = (N-1)S^2+\underset{1\leq k\neq l\leq n}{\sum\sum}\, E(y_k-\overline{Y})(y_l-\overline{Y})\\
& \implies E(y_i-\overline{Y})(y_j-\overline{Y})=\dfrac{1}{N(N-1)}(N-1)(-S^2)\\
& \implies K = n(n-1)\dfrac{(-S^2)}{N}
\end{align*}
Putting this value in (\ref{june2018-58:eq_1}) gives us 
\begin{align}
V(\overline{y})_{WOR} & = \dfrac{N-1}{Nn} S^2+ \dfrac{n-1(-S^2)}{Nn}\\
& = \dfrac{N-n}{Nn} S^2 \label{june2018-58:eq_2}
\end{align}
\textbf{K value in case of SRSWR}
\begin{align*}
&K=\underset{1\leq i\neq j\leq n}{\sum\sum}\, E(y_i-\overline{Y})(y_j-\overline{Y})
\end{align*}
Since we are selecting the samples with replacements choosing $i^{th}$ and $j^{th}$ sample is independent of each other. So,
\begin{align*}
K&=\underset{1\leq i\neq j\leq n}{\sum\sum}\, E(y_i-\overline{Y})E(y_j-\overline{Y})\\
& = 0\\
& \text{(Since deviation about mean is 0)}
\end{align*}
Putting K=0 in (\ref{june2018-58:eq_1}) we get 
\begin{align}
V(\overline{y})_{WR} & = \dfrac{N-1}{Nn} S^2\label{june2018-58:eq_3}
\end{align}
From equation \eqref{june2018-58:eq_2}  standard error of mean of sample class without repetition
\begin{align}
{SE}_{WOR} & = \dfrac{s}{\sqrt{n}}\\
& = \sqrt{\dfrac{V(\overline{y})_{WOR}}{n}}\\
& = \sqrt{\dfrac{N-n}{Nn^2}}S \label{june2018-58:eq_4}
\end{align} 
From equation \eqref{june2018-58:eq_3}  standard error of mean of sample class with repetition
\begin{align}
{SE}_{WR} & = \sqrt{\dfrac{V(\overline{y})_WR}{n}}\\
& = \sqrt{\dfrac{N-1}{Nn^2}}S \label{june2018-58:eq_5}
\end{align}
Given to find the value of n if $2 \times {SE}_{WOR} =  {SE}_{WR}$.
From \eqref{june2018-58:eq_4} and \eqref{june2018-58:eq_5} we can write 
\begin{align}
& 2\sqrt{\dfrac{N-n}{Nn^2}}S= \sqrt{\dfrac{N-1}{Nn^2}}S\\
\implies & 4(N-n) = N-1\\
\implies & 4N+1-N=4n\\
\implies & 4n=3(125)+1\\
\implies & n=94
\end{align}
Therefore the sample size for the given condition to be met is n=94.(\textbf{Option D})

%
\item Consider the function \textit{f(x)} defined as \textit{f(x)} = \textit{ce$^{-x^{4}}$}, $\textit{x} \in R$ . For what value of \textit{c} is \textit{f} a probability density function?\\
\begin{enumerate}
    \item $\displaystyle\frac{2}{\Gamma(1/4)}$\\\label{2019/52/option 1}
    \item $\displaystyle\frac{4}{\Gamma(1/4)}$\\
    \item $\displaystyle\frac{3}{\Gamma(1/3)}$\\
    \item $\displaystyle\frac{1}{4\Gamma(4)}$
\end{enumerate}
%
\solution
Consider a continuous random variable X so that the function \textit{f} can be probability density function if and only if it satisfies the condition 
\begin{align}
    \int_{-\infty}^{\infty}f_{X}(u)du = 1 \label{2019/52/equation 1}
\end{align}
Hence by applying the \eqref{2019/52/equation 1} for the function \textit{f} we get
\begin{align}
    \int_{-\infty}^{\infty}ce^{-u^{4}}du = 1\\
    2c\int_{0}^{\infty}e^{-u^{4}}du = 1\label{2019/52/equation 2}\\
    2c\int_{0}^{\infty}e^{-t}\frac{dt}{4t^{\frac{3}{4}}} = 1\\
    \frac{c}{2}\int_{0}^{\infty}e^{-t}t^{-\frac{3}{4}}dt = 1\label{2019/52/equation 3}
\end{align}
We know that gamma function for any real positive $\alpha$
\begin{align}
    \Gamma(\alpha) = \int_0^\infty x^{\alpha - 1} e^{-x} dx \label{2019/52/gammafunction}
\end{align}
Hence by using \eqref{2019/52/gammafunction} in \eqref{2019/52/equation 3} we get
\begin{align}
    \frac{c}{2}\Gamma(1/4) = 1\\
    c=\frac{2}{\Gamma(1/4)}
\end{align}
Hence $c = \displaystyle\frac{2}{\Gamma(1/4)}$ and option \eqref{2019/52/option 1} is correct.\newline\newline
The CDF of \textit{f} by using \eqref{2019/52/gammafunction} we get
\begin{align}
    F_{X}(x) &= \int_{0}^{x}f(u)du\\
             &= \frac{2}{\Gamma(\frac{1}{4})}\int_{0}^{x}e^{-u^{4}}du\\
             &= \frac{2}{4\Gamma(\frac{1}{4})}\int_{0}^{x^{4}}e^{-t}{t^{\frac{-3}{4}}}dt\\
             &= \frac{1}{2\Gamma(\frac{1}{4})}\int_{0}^{x^{4}}e^{-t}{t^{\frac{-3}{4}}}dt\\
             &= \frac{1}{2\Gamma(\frac{1}{4})}\brak{\Gamma\brak{\frac{1}{4}}-\Gamma\brak{\frac{1}{4},x^{4}}}\\
             &= \frac{1}{2\Gamma(\frac{1}{4})}\gamma\brak{\frac{1}{4},x^{4}}
\end{align}

\end{enumerate}

