\renewcommand{\theequation}{\theenumi}
\renewcommand{\thefigure}{\theenumi}
\renewcommand{\thetable}{\theenumi}
\begin{enumerate}[label=\thesection.\arabic*.,ref=\thesection.\theenumi]
\numberwithin{equation}{enumi}
\numberwithin{figure}{enumi}
\numberwithin{table}{enumi}
%
\item Let $X$ be a non-negative integer valued random variable with probability mass function $f(x)$ satisfying $(x+1)f(x+1)=(\alpha + \beta x)f(x)$, $x=0,1,2,...$; $\beta \neq 1$. You may assume that $E(X)$ and $Var(X)$ exist. Then which of the following statements are true?

\begin{enumerate}
    \item $E(X)=\dfrac{\alpha}{1-\beta}$ \vspace{0.2cm}
    \item $E(X)=\dfrac{\alpha^2}{(1-\beta)(1+\alpha)}$ \vspace{0.2cm}
    \item $Var(X)=\dfrac{\alpha^2}{(1-\beta)^2}$ \vspace{0.2cm}
    \item $Var(X)=\dfrac{\alpha}{(1-\beta)^2}$
\end{enumerate}
%
\solution
For a discrete random variable $X$ with P.D.F. $f(x)$ and which can take values from a set $\mathbb{S}$,
\begin{align} \label{june2013-75:eq-1}
    E(X)= \sum_{x \in \mathbb{S}}xf(x)
\end{align}
And,
\begin{align} \label{june2013-75:eq-2}
    E(X^2) =\sum_{x \in \mathbb{S}}x^2f(x)
\end{align}
Also, as $f(x)$ is the P.D.F.,
\begin{align} \label{june2013-75:eq-3}
    \sum_{x \in \mathbb{S}}f(x) = 1
\end{align}
Given, for $x \in \mathbb{S}=\{0,1,2,...n\}$,
\begin{align} \label{june2013-75:eq-4}
    (x+1)f(x+1)=(\alpha + \beta x)f(x)
\end{align}
Summing both sides for $x \in \mathbb{S}$ we get,
\begin{align}
    \sum_{x=0}^n(x+1)f(x+1)=\sum_{x=0}^n(\alpha +\beta x)f(x)
\end{align}
Replacing $x+1$ with $x$ in L.H.S. we get, 
\begin{align}
    \sum_{x=1}^{n+1}xf(x)=\sum_{x=0}^n(\alpha +\beta x)f(x)
\end{align}
Rewriting LHS, we get,
\begin{align}
    \sum_{x=0}^nxf(x)+(n+1)f(n+1)=\sum_{x=0}^n(\alpha +\beta x)f(x)
\end{align}
But as $x \in \{0,1,2...n\}$, $f(n+1)=0$. So the equation becomes
\begin{align}
    \sum_{x=0}^nxf(x)=\alpha \sum_{x=0}^nf(x) + \beta \sum_{x=0}^nxf(x)
\end{align}
Using \eqref{june2013-75:eq-1} and \eqref{june2013-75:eq-3}, we get,
\begin{align} 
    E(X)=\alpha(1) + \beta E(X)
\end{align}
So,
\begin{align} \label{june2013-75:eq-5}
    E(X)=\dfrac{\alpha}{1-\beta}
\end{align}
Now in \eqref{june2013-75:eq-4}, multiplying both sides by $(x+1)$, we get,
\begin{align}
    (x+1)^2f(x+1)=(\alpha + \beta x)(x+1)f(x)
\end{align}
Summing both sides for $x \in \mathbb{S}$ we get,
\begin{align}
    \sum_{x=0}^n(x+1)^2f(x+1)=\sum_{x=0}^n(\alpha +\beta x)(x+1)f(x)
\end{align}
Replacing $x+1$ with $x$ in L.H.S. we get, 
\begin{align}
    \sum_{x=1}^{n+1}x^2f(x)=\sum_{x=0}^n(\beta x^2f(x) + (\alpha+\beta)xf(x) + \alpha f(x))
\end{align}
Rewriting LHS similarly as before, we get,
\begin{align}
    \sum_{x=0}^nx^2f(x)=\beta \sum_{x=0}^nx^2f(x) + \nonumber \\
    (\alpha + \beta)\sum_{x=0}^nxf(x) + \alpha \sum_{x=0}^nf(x)
\end{align}
Using \eqref{june2013-75:eq-1}, \eqref{june2013-75:eq-2} and \eqref{june2013-75:eq-3}, we get,
\begin{align}
    E(X^2)=\beta E(X^2) + (\alpha + \beta)E(X) + \alpha (1) 
\end{align}
Using \eqref{june2013-75:eq-5}
\begin{align}
    E(X^2)(1-\beta)=\dfrac{\alpha(\alpha+\beta)}{1-\beta} + \alpha
\end{align}
So,
\begin{align} \label{june2013-75:eq-6}
    E(X^2)=\dfrac{\alpha^2+\alpha}{(1-\beta)^2}
\end{align}
Now,
\begin{align}
    Var(X)=E(X^2)-(E(X))^2
\end{align}
Using \eqref{june2013-75:eq-5} and \eqref{june2013-75:eq-6},
\begin{align}
    Var(X)=\dfrac{\alpha^2+\alpha}{(1-\beta)^2}-\dfrac{\alpha^2}{(1-\beta)^2}
\end{align}
So,
\begin{align}
    Var(X)=\dfrac{\alpha}{(1-\beta)^2}
\end{align}
So, options 1 and 4 are correct.
%
\item Let X be a random variable with probability density function,
\begin{align}
    f(x)=\alpha(x-\mu)^{\alpha-1}e^{-(x-\mu)^{\alpha}}
\end{align}
such that $-\infty<\mu<\infty\;;\alpha>0\;;x>\mu$, The hazard function is: 
\begin{enumerate}
    \item constant for all $\alpha$
    \item an increasing function for some $\alpha$
    \item independent of $\alpha$
    \item independent of $\mu$ when $\alpha=1$
\end{enumerate}
%
\solution
\input{solutions/2013/june/71.tex}
%
\item 
A point is chosen at random from a circular disc shown below. What is the probability that the point lies in the sector OAB?\\

\begin{tikzpicture}
\draw (0,0) circle (3cm);
\draw (0,0) node{O}-- (2,2.25) node{A};
\draw (0,0) -- (2.828,1) node{B};
\end{tikzpicture}\\

( where $\angle$AOB = x radians )


    \begin{enumerate}
        \item $\frac{2x}{\pi}$
        \item $\frac{x}{\pi}$
        \item $\frac{x}{2\pi}$
        \item $\frac{x}{4\pi}$
    \end{enumerate}

%
\solution
\begin{lemma}[Sum of Gamma random variables]
    \label{stats/4/gamma}
Suppose that $X_i \sim \Gamma(k, \theta), i = 1, \dots, n$. Then $T = \sum_{i=1}^{n} X_i \sim \Gamma(nk, \theta)$.
\end{lemma}
\begin{definition}[Laplace transform]
    Laplace transform is an integral transform that converts a real function of a real variable $t$ to a function of a complex variable $s$. The laplace transform of a function $f(t)$ evaluated at $s$ is defined by
    \begin{align}
    F(s) = \int_0^{\infty} f(t) e^{-st} dt
    \end{align}
    \end{definition}
    \begin{lemma}
    \label{stats/6/lemma_laplace_transform}
    If the laplace transform of a function $f(t)$ at $s$ is $0$ for all $s>0$, then the function $f(t) = 0$ almost everywhere in $(0, \infty)$.
    \end{lemma}
\begin{definition}[Complete Statistic]
The statistic $T$ is said to be complete for the distribution of $X$ if, for every measurable function $g$, if
\begin{align}
E(g(T)) = 0 \; \forall \; \theta \Rightarrow \pr{g(T)=0} = 1 
\end{align}
\end{definition}
\begin{definition}[Sufficient Statistic]
A statistic $t = T(X)$ is sufficient for underlying parameter $\theta$ precisely if the conditional probability distribution of the data $X$, given the statistic $t = T(X)$, does not depend on the parameter $\theta$. 
\end{definition}
\begin{theorem}[Fischer-Neymann Factorisation theorem]
If the probability density function is $f_\theta (x)$, then $T$ is sufficient for $\theta$ if and only if nonnegative functions $g$ and $h$ can be found such that
\begin{align}
f_\theta (x) = h(x)g_\theta(T(x))
\end{align}
\end{theorem}
\begin{lemma} \label{stats/6/lemma_sufficient_complete}
\begin{align}
T = \sum_{i=1}^{n} X_i
\end{align}
is a complete and sufficient statistic. 
\end{lemma}
\begin{proof}
\begin{enumerate}
\item \textit{(Sufficiency)}
\begin{align}
f_{X}(x_1, x_2, \ldots x_n) &= \prod_{i=1}^{n}f_{X_i}(x_i)\\
&= \prod_{i=1}^{n} \dfrac{1}{\theta} \exp\brak{-\dfrac{x_i}{\theta}} \\
&= 1 \cdot \dfrac{1}{\theta^n} \exp\brak{-\dfrac{T}{\theta}}\\
&= h(X) \cdot g(T, \theta) 
\end{align}
with 
\begin{align}
g(T, \theta) &= \dfrac{1}{\theta^n} \exp\brak{-\dfrac{T}{\theta}}\\
h(X) &= 1
\end{align}
Therefore, $T$ is a sufficient statistic.
\item \textit{(Completeness)}
From Lemma     \ref{stats/4/gamma}, $X_i \sim \Gamma \brak{1, \frac{1}{\theta}} \implies  T = \sum_{i=1}^n X_i \sim \Gamma \brak{n, \frac{1}{\theta}}$. 
\begin{align}
\therefore E(g(T)) &= 
\int_0 ^{\infty} g(t) \dfrac{t^{n-1} e^{-\frac{t}{\theta}}}{\theta^n(n-1)!} dt\\ 
&= \frac{1}{\theta^n (n-1)!} \int_0 ^{\infty} g(t) t^{n-1} e^{-\frac{t}{\theta}} dt\\ 
&= \frac{1}{\theta^n (n-1)!} F \brak{\frac{1}{\theta}}
%&= 0 \text{ for all } \theta > 0.
\end{align}
where, 
\begin{align}
f(t) &= t^{n-1}g(t) \xrightleftharpoons[]{\mathcal{L}} F \brak{\frac{1}{\theta}}
% &= \text{Laplace Transform of} f(t) \text{ at } \theta
\end{align}
is the Laplace transform relationship.
Then
\begin{align}
E\brak{g(T)}=0 \forall \theta > 0 \implies F \brak{\frac{1}{\theta}} = 0 \forall  \theta > 0
\label{stats/6/eqn}
\\ \implies 
f(t)=0 \text{ a.e } (0, \infty)\\
\text{or, }g(t)=0 \text{ a.e }(0, \infty)\\
\implies \pr{g(t)=0} = 1
\end{align}
from Lemma     \ref{stats/6/lemma_laplace_transform}.  
$\therefore T$ is a complete statistic.
\end{enumerate}
\end{proof}
\begin{theorem}[Lehmann–Scheffé theorem]
If $T$ is a complete sufficient statistic for $\theta$ and 
\begin{align}
\label{eqn 2.0.1}
E(g(T)) = \tau(\theta)
\end{align}
then $g(T)$ is the uniformly minimum-variance unbiased estimator (UMVUE) of $\tau(\theta)$.
\end{theorem}
%
\begin{proposition} \label{stats/6/prop_3.1}
$E(X_{(1)}|T)$ is the uniformly minimum-variance unbiased estimator (UMVUE) of $E(X_{(1)})$
\end{proposition}
\begin{proof}
By the law of total expectation, 
\begin{align}
\label{stats/6/eqn 2.0.3}
E\brak{E(X_{(1)} | T )} = E(X_{(1)})
\end{align}
We know that $T = \sum_{i=1}^n X_i$ is a complete and sufficient statistic by Lemma \ref{stats/6/lemma_sufficient_complete}. By Lehmann–Scheffé theorem, with
\begin{align}
\theta &= X_{(1)},\\ 
\tau(x) &= E(x),\\
g(T) &= E(X_{(1)} | T).
\end{align}
it follows from \eqref{stats/6/eqn 2.0.3} that $E(X_{(1)} | T)$ is the UMVUE of $E(X_{(1)})$.
\end{proof}
\begin{proposition} \label{stats/6/prop_3.2}
$\frac{T}{n^2}$ is uniformly minimum-variance unbiased estimator (UMVUE) of $E(X_{(1)})$
\end{proposition}
\begin{proof}
\begin{enumerate}
\item Let's find the probability distribution function and the expectation value of $X_{(1)}$ : 
\begin{align}
\pr{X_{(1)} > x} &= \pr{X_1 > x}\ldots \pr{X_n > x}\\
&= (1-F_{X_{1}}(x))\ldots(1-F_{X_{n}}(x))\\
&= (1-F_{X_{1}}(x))^n \\
&= \exp\brak{-\frac{nx}{\theta}}\\
F_{X_{(1)}}(x) &= 1 - \exp\brak{-\frac{nx}{\theta}}\\
f_{X_{(1)}}(x) &= \frac{n}{\theta} \exp\brak{-\frac{nx}{\theta}}
\end{align}
Therefore, $X_{(1)}$ follows an exponential distribution with mean $\dfrac{\theta}{n}$.
\begin{align}
E(X_{(1)}) = \frac{\theta}{n}
\end{align}
\item $\frac{T}{n^2}$ is uniformly minimum-variance unbiased estimator (UMVUE) of $E(X_{(1)})$, since $E\brak{\frac{T}{n^2}} = E(X_{(1)})$ : \\
Note that,
\begin{align}
E\brak{\frac{T}{n^2}} &= E\brak{\frac{\sum_{i=1}^n X_i}{n^2}}\\
&= \frac{E(\sum_{i=1}^n X_i)}{n^2}\\
&= \sum_{i=1}^n \frac{E(X_i)}{n^2}\\
&= \sum_{i=1}^n \frac{\theta}{n^2}\\
&= \frac{\theta}{n}\\
&= E(X_{(1)})
\end{align}
\end{enumerate}
Therefore, by Lehmann–Scheffé theorem, with
\begin{align}
\theta &= X_{(1)},\\
\tau(x) &= E(x),\\
g(T) &= \frac{T}{n^2},
\end{align}
it follows that $\dfrac{T}{n^2}$ is UMVUE of $E(X_{(1)})$.\\
\end{proof}
Since there exists a unique UMVUE for $E(X_{(1)})$, it follows from Proposition \ref{stats/6/prop_3.1} and Proposition \ref{stats/6/prop_3.2} that 
\begin{align}
E(X_{(1)} | T) = \frac{T}{n^2} 
\end{align}
Hence, option A is correct.
\begin{figure}[!hbt]
    \centering
	\includegraphics[width=\columnwidth]{stats/solutions/6/Figures/Figure_1.png}
    \caption{Theory vs Simulated plot of $E(X_{(1)} |T)$}
    \label{CDF_Y}
\end{figure}

%
\item Let $X$ and $Y$ be independent random variables each following a uniform distribution on $(0,1)$.Let $W=XI_{\{Y\leq X^2\}}$,where $I_A$ denotes the indicator function of set $A$.Then which of the following statements are true? \\
\begin{enumerate}
\item The cumulative distribution function of $W$ is given by
\begin{align}
  F_W(t)=t^2I_{\{0\leq t \leq 1\}}+ I_{\{t > 1\}}
\end{align}
\item $P\sbrak{W>0}=\frac{1}{3}$
\item The cumulative distribution function of $W$ is continuous
\item The cumulative distribution function of $W$ is given by
\begin{align}
  F_W(t)=\brak{\frac{2+t^3}{3}}I_{\{0\leq t \leq 1\}}+ I_{\{t > 1\}}
\end{align}
\end{enumerate}
%
\solution
\input{solutions/2013/june/70/assignment8.tex}
%
\item Let $U_1,U_2\dots,U_n$ be independent and identically distributed random variables each
having a uniform distribution on (0,1). Then,$\lim_{n \to +\infty} \pr{U_1+U_2\dots,U_n\leq \frac{3}{4}n}$
\begin{enumerate}
    \item does not exist
    \item exists and equals 0
    \item exists and equals 1
    \item exists and equals $\frac{3}{4}$
\end{enumerate}
%
\solution
Since $X$ and $Y$ are exponential random variables with means'
\begin{align}
    E[X] = 1 \text{ and } E[Y] = \frac{1}{2}
\end{align}
Marginal PDFs of $X$ and $Y$  are given by
\begin{align}
    f_X(x)= e^{-x} , x>0 \\
    f_Y(y) = 2e^{-2y} , y>0
\end{align}
CDFs for $X$ and $Y$ are
\begin{align}
    F_X(b) &= \int_0^b f_X(x)\,d_x\\
           &= \int_0^b  e^{-x}\,d_x\\
           &= 1-e^{-b}
\end{align}
\begin{align}
    F_Y(b) &= \int_0^b f_Y(y)\,d_y\\
           &= \int_0^b 2e^{-2y}\,d_y\\
           &= \left[-e^{-2y}\right]_0^b\\
           &= 1-e^{-2b}
\end{align}
Now,
\begin{align}
    \pr{X>2Y|X>Y} &= \dfrac{\pr{X>2Y\,,\,X>Y} }{\pr{X>Y}}\\
                  &= \dfrac{\pr{X>2Y}}{\pr{X>Y}} \label{dec2017-59:simul1}
\end{align}
\begin{align}
    \pr{X>Y} &= \pr{Y<X}\\
             &= E[F_Y(X)]\\
             &= \int_0^\infty F_Y(X)\,f_X(x)\,d_x\\
             &= \int_0^\infty (1-e^{-2x})\,e^{-x}d_x\\
             &= \left[\frac{e^{-x}}{-1} - \frac{e^{-3x}}{-3}\right]_0^\infty\\
            &=(0+1)+\frac{1}{3}(0-1) \\
             &= \frac{2}{3} \label{dec2017-59:simul2}
\end{align}
\begin{align}
    \pr{X>2Y} &= \pr{Y<\frac{X}{2}}\\
              &= E[F_Y(X/2)]\\
              &=  \int_0^\infty F_Y(X/2)\,f_X(x)\,d_x\\
              &= \int_{0}^{\infty}(1-e^{-x})\,e^{-x}d_x\\
              &= \left[\frac{e^{-x}}{-1} - \frac{e^{-2x}}{-2}\right]_0^\infty\\
              &= (0+1) + \frac{1}{2}(0-1)\\
              &= \frac{1}{2} \label{dec2017-59:simul3}
\end{align}
Putting \eqref{dec2017-59:simul2} and \eqref{dec2017-59:simul3} in \eqref{dec2017-59:simul1}
\begin{align}
    \pr{X>2Y\,|\,X>Y} &= \frac{1/2}{2/3}\\
                  &= \frac{3}{4}
\end{align}
$\therefore$ Option 4 is the correct answer.
%
\item  Consider the quadratic equation $x^2+2U x+V=0$ where $U$ and $V$ are chosen independently and randomly from $\{1,2,3\}$ with equal probabilities. Then probability that the equation has both roots real
\begin{enumerate}
    \item $\frac{2}{3}$
    \item $\frac{1}{2}$
    \item $\frac{7}{9}$
    \item $\frac{1}{3}$
\end{enumerate}
%
\solution
Let $U\in\{1.2,3\}$ and $V\in\{1,2,3\}$
\begin{table}[h!]
\centering
\caption{Probability of selecting values for $U$}
\resizebox{\columnwidth}{!}{
  \begin{tabular}{||c|c|c|c||}
    \hline
    $k$ & $1$ & $2$ & $3$\\
    \hline
    \hline
    $\pr{U=k}$ & $1/3$ & $1/3$ & $1/3$\\
    \hline
  \end{tabular}
}
\label{june2013-60:Table1}
\end{table}
\begin{table}[h!]
\centering
\caption{Probability of selecting values for $V$}
\resizebox{\columnwidth}{!}{
  \begin{tabular}{||c|c|c|c||}
    \hline
    $k$ & $1$ & $2$ & $3$\\
    \hline
    \hline
    $\pr{V=k}$ & $1/3$ & $1/3$ & $1/3$\\
    \hline
  \end{tabular}
}
\label{june2013-60:Table2}
\end{table}
For $x^2+2U x+V=0$ to have real roots,
\begin{align}
    b^2-4ac\geq0\\
    \brak{2U}^2-4\brak{1}\brak{V}\geq0\\
    U^2\geq V
\end{align}
\begin{align}
    \pr{U^2\geq V}=1-\pr{U^2<V}
\end{align}
The possible pairs \brak{U,V} for \pr{U^2<V},
\vspace{0.00001in}
\begin{table}[h!]
\centering
\caption{Table for \pr{U^2<V}}
\resizebox{\columnwidth}{!}{
  \begin{tabular}{||c|c||}
    \hline
    $\brak{U,V}$ for $U^2<V$ & Probability\\
    \hline
    \hline
    $\brak{1,2}$ & \pr{U=1}\pr{V=2}=1/9\\
    \hline
    $\brak{1,3}$ & $\pr{U=1}\pr{V=3}=\frac{1}{9}$\\
    \hline
    \hline
    Total & $\pr{U^2<V}=\frac{2}{9}$\\
    \hline
  \end{tabular}
}
\label{june2013-60:Table3}
\end{table}
\begin{align}
    \pr{U^2\geq V}=1-\frac{2}{9}\\
    \pr{U^2\geq V}=\frac{7}{9}
\end{align}
Hence, Option 3 is correct.

%

\end{enumerate}
