\renewcommand{\theequation}{\theenumi}
\renewcommand{\thefigure}{\theenumi}
\renewcommand{\thetable}{\theenumi}
\begin{enumerate}[label=\thesection.\arabic*.,ref=\thesection.\theenumi]
\numberwithin{equation}{enumi}
\numberwithin{figure}{enumi}
\numberwithin{table}{enumi}
%
\item Let $X$ be a non-negative integer valued random variable with probability mass function $f(x)$ satisfying $(x+1)f(x+1)=(\alpha + \beta x)f(x)$, $x=0,1,2,...$; $\beta \neq 1$. You may assume that $E(X)$ and $Var(X)$ exist. Then which of the following statements are true?

\begin{enumerate}
    \item $E(X)=\dfrac{\alpha}{1-\beta}$ \vspace{0.2cm}
    \item $E(X)=\dfrac{\alpha^2}{(1-\beta)(1+\alpha)}$ \vspace{0.2cm}
    \item $Var(X)=\dfrac{\alpha^2}{(1-\beta)^2}$ \vspace{0.2cm}
    \item $Var(X)=\dfrac{\alpha}{(1-\beta)^2}$
\end{enumerate}
%
\solution
For a discrete random variable $X$ with P.D.F. $f(x)$ and which can take values from a set $\mathbb{S}$,
\begin{align} \label{june2013-75:eq-1}
    E(X)= \sum_{x \in \mathbb{S}}xf(x)
\end{align}
And,
\begin{align} \label{june2013-75:eq-2}
    E(X^2) =\sum_{x \in \mathbb{S}}x^2f(x)
\end{align}
Also, as $f(x)$ is the P.D.F.,
\begin{align} \label{june2013-75:eq-3}
    \sum_{x \in \mathbb{S}}f(x) = 1
\end{align}
Given, for $x \in \mathbb{S}=\{0,1,2,...n\}$,
\begin{align} \label{june2013-75:eq-4}
    (x+1)f(x+1)=(\alpha + \beta x)f(x)
\end{align}
Summing both sides for $x \in \mathbb{S}$ we get,
\begin{align}
    \sum_{x=0}^n(x+1)f(x+1)=\sum_{x=0}^n(\alpha +\beta x)f(x)
\end{align}
Replacing $x+1$ with $x$ in L.H.S. we get, 
\begin{align}
    \sum_{x=1}^{n+1}xf(x)=\sum_{x=0}^n(\alpha +\beta x)f(x)
\end{align}
Rewriting LHS, we get,
\begin{align}
    \sum_{x=0}^nxf(x)+(n+1)f(n+1)=\sum_{x=0}^n(\alpha +\beta x)f(x)
\end{align}
But as $x \in \{0,1,2...n\}$, $f(n+1)=0$. So the equation becomes
\begin{align}
    \sum_{x=0}^nxf(x)=\alpha \sum_{x=0}^nf(x) + \beta \sum_{x=0}^nxf(x)
\end{align}
Using \eqref{june2013-75:eq-1} and \eqref{june2013-75:eq-3}, we get,
\begin{align} 
    E(X)=\alpha(1) + \beta E(X)
\end{align}
So,
\begin{align} \label{june2013-75:eq-5}
    E(X)=\dfrac{\alpha}{1-\beta}
\end{align}
Now in \eqref{june2013-75:eq-4}, multiplying both sides by $(x+1)$, we get,
\begin{align}
    (x+1)^2f(x+1)=(\alpha + \beta x)(x+1)f(x)
\end{align}
Summing both sides for $x \in \mathbb{S}$ we get,
\begin{align}
    \sum_{x=0}^n(x+1)^2f(x+1)=\sum_{x=0}^n(\alpha +\beta x)(x+1)f(x)
\end{align}
Replacing $x+1$ with $x$ in L.H.S. we get, 
\begin{align}
    \sum_{x=1}^{n+1}x^2f(x)=\sum_{x=0}^n(\beta x^2f(x) + (\alpha+\beta)xf(x) + \alpha f(x))
\end{align}
Rewriting LHS similarly as before, we get,
\begin{align}
    \sum_{x=0}^nx^2f(x)=\beta \sum_{x=0}^nx^2f(x) + \nonumber \\
    (\alpha + \beta)\sum_{x=0}^nxf(x) + \alpha \sum_{x=0}^nf(x)
\end{align}
Using \eqref{june2013-75:eq-1}, \eqref{june2013-75:eq-2} and \eqref{june2013-75:eq-3}, we get,
\begin{align}
    E(X^2)=\beta E(X^2) + (\alpha + \beta)E(X) + \alpha (1) 
\end{align}
Using \eqref{june2013-75:eq-5}
\begin{align}
    E(X^2)(1-\beta)=\dfrac{\alpha(\alpha+\beta)}{1-\beta} + \alpha
\end{align}
So,
\begin{align} \label{june2013-75:eq-6}
    E(X^2)=\dfrac{\alpha^2+\alpha}{(1-\beta)^2}
\end{align}
Now,
\begin{align}
    Var(X)=E(X^2)-(E(X))^2
\end{align}
Using \eqref{june2013-75:eq-5} and \eqref{june2013-75:eq-6},
\begin{align}
    Var(X)=\dfrac{\alpha^2+\alpha}{(1-\beta)^2}-\dfrac{\alpha^2}{(1-\beta)^2}
\end{align}
So,
\begin{align}
    Var(X)=\dfrac{\alpha}{(1-\beta)^2}
\end{align}
So, options 1 and 4 are correct.
%
\item Let X be a random variable with probability density function,
\begin{align}
    f(x)=\alpha(x-\mu)^{\alpha-1}e^{-(x-\mu)^{\alpha}}
\end{align}
such that $-\infty<\mu<\infty\;;\alpha>0\;;x>\mu$, The hazard function is: 
\begin{enumerate}
    \item constant for all $\alpha$
    \item an increasing function for some $\alpha$
    \item independent of $\alpha$
    \item independent of $\mu$ when $\alpha=1$
\end{enumerate}
%
\solution
\input{solutions/2013/june/71.tex}
%
\item 
A point is chosen at random from a circular disc shown below. What is the probability that the point lies in the sector OAB?\\

\begin{tikzpicture}
\draw (0,0) circle (3cm);
\draw (0,0) node{O}-- (2,2.25) node{A};
\draw (0,0) -- (2.828,1) node{B};
\end{tikzpicture}\\

( where $\angle$AOB = x radians )


    \begin{enumerate}
        \item $\frac{2x}{\pi}$
        \item $\frac{x}{\pi}$
        \item $\frac{x}{2\pi}$
        \item $\frac{x}{4\pi}$
    \end{enumerate}

%
\solution
\begin{lemma}[Sum of Gamma random variables]
    \label{stats/4/gamma}
Suppose that $X_i \sim \Gamma(k, \theta), i = 1, \dots, n$. Then $T = \sum_{i=1}^{n} X_i \sim \Gamma(nk, \theta)$.
\end{lemma}
\begin{definition}[Laplace transform]
    Laplace transform is an integral transform that converts a real function of a real variable $t$ to a function of a complex variable $s$. The laplace transform of a function $f(t)$ evaluated at $s$ is defined by
    \begin{align}
    F(s) = \int_0^{\infty} f(t) e^{-st} dt
    \end{align}
    \end{definition}
    \begin{lemma}
    \label{stats/6/lemma_laplace_transform}
    If the laplace transform of a function $f(t)$ at $s$ is $0$ for all $s>0$, then the function $f(t) = 0$ almost everywhere in $(0, \infty)$.
    \end{lemma}
\begin{definition}[Complete Statistic]
The statistic $T$ is said to be complete for the distribution of $X$ if, for every measurable function $g$, if
\begin{align}
E(g(T)) = 0 \; \forall \; \theta \Rightarrow \pr{g(T)=0} = 1 
\end{align}
\end{definition}
\begin{definition}[Sufficient Statistic]
A statistic $t = T(X)$ is sufficient for underlying parameter $\theta$ precisely if the conditional probability distribution of the data $X$, given the statistic $t = T(X)$, does not depend on the parameter $\theta$. 
\end{definition}
\begin{theorem}[Fischer-Neymann Factorisation theorem]
If the probability density function is $f_\theta (x)$, then $T$ is sufficient for $\theta$ if and only if nonnegative functions $g$ and $h$ can be found such that
\begin{align}
f_\theta (x) = h(x)g_\theta(T(x))
\end{align}
\end{theorem}
\begin{lemma} \label{stats/6/lemma_sufficient_complete}
\begin{align}
T = \sum_{i=1}^{n} X_i
\end{align}
is a complete and sufficient statistic. 
\end{lemma}
\begin{proof}
\begin{enumerate}
\item \textit{(Sufficiency)}
\begin{align}
f_{X}(x_1, x_2, \ldots x_n) &= \prod_{i=1}^{n}f_{X_i}(x_i)\\
&= \prod_{i=1}^{n} \dfrac{1}{\theta} \exp\brak{-\dfrac{x_i}{\theta}} \\
&= 1 \cdot \dfrac{1}{\theta^n} \exp\brak{-\dfrac{T}{\theta}}\\
&= h(X) \cdot g(T, \theta) 
\end{align}
with 
\begin{align}
g(T, \theta) &= \dfrac{1}{\theta^n} \exp\brak{-\dfrac{T}{\theta}}\\
h(X) &= 1
\end{align}
Therefore, $T$ is a sufficient statistic.
\item \textit{(Completeness)}
From Lemma     \ref{stats/4/gamma}, $X_i \sim \Gamma \brak{1, \frac{1}{\theta}} \implies  T = \sum_{i=1}^n X_i \sim \Gamma \brak{n, \frac{1}{\theta}}$. 
\begin{align}
\therefore E(g(T)) &= 
\int_0 ^{\infty} g(t) \dfrac{t^{n-1} e^{-\frac{t}{\theta}}}{\theta^n(n-1)!} dt\\ 
&= \frac{1}{\theta^n (n-1)!} \int_0 ^{\infty} g(t) t^{n-1} e^{-\frac{t}{\theta}} dt\\ 
&= \frac{1}{\theta^n (n-1)!} F \brak{\frac{1}{\theta}}
%&= 0 \text{ for all } \theta > 0.
\end{align}
where, 
\begin{align}
f(t) &= t^{n-1}g(t) \xrightleftharpoons[]{\mathcal{L}} F \brak{\frac{1}{\theta}}
% &= \text{Laplace Transform of} f(t) \text{ at } \theta
\end{align}
is the Laplace transform relationship.
Then
\begin{align}
E\brak{g(T)}=0 \forall \theta > 0 \implies F \brak{\frac{1}{\theta}} = 0 \forall  \theta > 0
\label{stats/6/eqn}
\\ \implies 
f(t)=0 \text{ a.e } (0, \infty)\\
\text{or, }g(t)=0 \text{ a.e }(0, \infty)\\
\implies \pr{g(t)=0} = 1
\end{align}
from Lemma     \ref{stats/6/lemma_laplace_transform}.  
$\therefore T$ is a complete statistic.
\end{enumerate}
\end{proof}
\begin{theorem}[Lehmann–Scheffé theorem]
If $T$ is a complete sufficient statistic for $\theta$ and 
\begin{align}
\label{eqn 2.0.1}
E(g(T)) = \tau(\theta)
\end{align}
then $g(T)$ is the uniformly minimum-variance unbiased estimator (UMVUE) of $\tau(\theta)$.
\end{theorem}
%
\begin{proposition} \label{stats/6/prop_3.1}
$E(X_{(1)}|T)$ is the uniformly minimum-variance unbiased estimator (UMVUE) of $E(X_{(1)})$
\end{proposition}
\begin{proof}
By the law of total expectation, 
\begin{align}
\label{stats/6/eqn 2.0.3}
E\brak{E(X_{(1)} | T )} = E(X_{(1)})
\end{align}
We know that $T = \sum_{i=1}^n X_i$ is a complete and sufficient statistic by Lemma \ref{stats/6/lemma_sufficient_complete}. By Lehmann–Scheffé theorem, with
\begin{align}
\theta &= X_{(1)},\\ 
\tau(x) &= E(x),\\
g(T) &= E(X_{(1)} | T).
\end{align}
it follows from \eqref{stats/6/eqn 2.0.3} that $E(X_{(1)} | T)$ is the UMVUE of $E(X_{(1)})$.
\end{proof}
\begin{proposition} \label{stats/6/prop_3.2}
$\frac{T}{n^2}$ is uniformly minimum-variance unbiased estimator (UMVUE) of $E(X_{(1)})$
\end{proposition}
\begin{proof}
\begin{enumerate}
\item Let's find the probability distribution function and the expectation value of $X_{(1)}$ : 
\begin{align}
\pr{X_{(1)} > x} &= \pr{X_1 > x}\ldots \pr{X_n > x}\\
&= (1-F_{X_{1}}(x))\ldots(1-F_{X_{n}}(x))\\
&= (1-F_{X_{1}}(x))^n \\
&= \exp\brak{-\frac{nx}{\theta}}\\
F_{X_{(1)}}(x) &= 1 - \exp\brak{-\frac{nx}{\theta}}\\
f_{X_{(1)}}(x) &= \frac{n}{\theta} \exp\brak{-\frac{nx}{\theta}}
\end{align}
Therefore, $X_{(1)}$ follows an exponential distribution with mean $\dfrac{\theta}{n}$.
\begin{align}
E(X_{(1)}) = \frac{\theta}{n}
\end{align}
\item $\frac{T}{n^2}$ is uniformly minimum-variance unbiased estimator (UMVUE) of $E(X_{(1)})$, since $E\brak{\frac{T}{n^2}} = E(X_{(1)})$ : \\
Note that,
\begin{align}
E\brak{\frac{T}{n^2}} &= E\brak{\frac{\sum_{i=1}^n X_i}{n^2}}\\
&= \frac{E(\sum_{i=1}^n X_i)}{n^2}\\
&= \sum_{i=1}^n \frac{E(X_i)}{n^2}\\
&= \sum_{i=1}^n \frac{\theta}{n^2}\\
&= \frac{\theta}{n}\\
&= E(X_{(1)})
\end{align}
\end{enumerate}
Therefore, by Lehmann–Scheffé theorem, with
\begin{align}
\theta &= X_{(1)},\\
\tau(x) &= E(x),\\
g(T) &= \frac{T}{n^2},
\end{align}
it follows that $\dfrac{T}{n^2}$ is UMVUE of $E(X_{(1)})$.\\
\end{proof}
Since there exists a unique UMVUE for $E(X_{(1)})$, it follows from Proposition \ref{stats/6/prop_3.1} and Proposition \ref{stats/6/prop_3.2} that 
\begin{align}
E(X_{(1)} | T) = \frac{T}{n^2} 
\end{align}
Hence, option A is correct.
\begin{figure}[!hbt]
    \centering
	\includegraphics[width=\columnwidth]{stats/solutions/6/Figures/Figure_1.png}
    \caption{Theory vs Simulated plot of $E(X_{(1)} |T)$}
    \label{CDF_Y}
\end{figure}

%
\item Let $X$ and $Y$ be independent random variables each following a uniform distribution on $(0,1)$.Let $W=XI_{\{Y\leq X^2\}}$,where $I_A$ denotes the indicator function of set $A$.Then which of the following statements are true? \\
\begin{enumerate}
\item The cumulative distribution function of $W$ is given by
\begin{align}
  F_W(t)=t^2I_{\{0\leq t \leq 1\}}+ I_{\{t > 1\}}
\end{align}
\item $P\sbrak{W>0}=\frac{1}{3}$
\item The cumulative distribution function of $W$ is continuous
\item The cumulative distribution function of $W$ is given by
\begin{align}
  F_W(t)=\brak{\frac{2+t^3}{3}}I_{\{0\leq t \leq 1\}}+ I_{\{t > 1\}}
\end{align}
\end{enumerate}
%
\solution
\input{solutions/2013/june/70/assignment8.tex}
%
\item Let $U_1,U_2\dots,U_n$ be independent and identically distributed random variables each
having a uniform distribution on (0,1). Then,$\lim_{n \to +\infty} \pr{U_1+U_2\dots,U_n\leq \frac{3}{4}n}$
\begin{enumerate}
    \item does not exist
    \item exists and equals 0
    \item exists and equals 1
    \item exists and equals $\frac{3}{4}$
\end{enumerate}
%
\solution
Since $X$ and $Y$ are exponential random variables with means'
\begin{align}
    E[X] = 1 \text{ and } E[Y] = \frac{1}{2}
\end{align}
Marginal PDFs of $X$ and $Y$  are given by
\begin{align}
    f_X(x)= e^{-x} , x>0 \\
    f_Y(y) = 2e^{-2y} , y>0
\end{align}
CDFs for $X$ and $Y$ are
\begin{align}
    F_X(b) &= \int_0^b f_X(x)\,d_x\\
           &= \int_0^b  e^{-x}\,d_x\\
           &= 1-e^{-b}
\end{align}
\begin{align}
    F_Y(b) &= \int_0^b f_Y(y)\,d_y\\
           &= \int_0^b 2e^{-2y}\,d_y\\
           &= \left[-e^{-2y}\right]_0^b\\
           &= 1-e^{-2b}
\end{align}
Now,
\begin{align}
    \pr{X>2Y|X>Y} &= \dfrac{\pr{X>2Y\,,\,X>Y} }{\pr{X>Y}}\\
                  &= \dfrac{\pr{X>2Y}}{\pr{X>Y}} \label{dec2017-59:simul1}
\end{align}
\begin{align}
    \pr{X>Y} &= \pr{Y<X}\\
             &= E[F_Y(X)]\\
             &= \int_0^\infty F_Y(X)\,f_X(x)\,d_x\\
             &= \int_0^\infty (1-e^{-2x})\,e^{-x}d_x\\
             &= \left[\frac{e^{-x}}{-1} - \frac{e^{-3x}}{-3}\right]_0^\infty\\
            &=(0+1)+\frac{1}{3}(0-1) \\
             &= \frac{2}{3} \label{dec2017-59:simul2}
\end{align}
\begin{align}
    \pr{X>2Y} &= \pr{Y<\frac{X}{2}}\\
              &= E[F_Y(X/2)]\\
              &=  \int_0^\infty F_Y(X/2)\,f_X(x)\,d_x\\
              &= \int_{0}^{\infty}(1-e^{-x})\,e^{-x}d_x\\
              &= \left[\frac{e^{-x}}{-1} - \frac{e^{-2x}}{-2}\right]_0^\infty\\
              &= (0+1) + \frac{1}{2}(0-1)\\
              &= \frac{1}{2} \label{dec2017-59:simul3}
\end{align}
Putting \eqref{dec2017-59:simul2} and \eqref{dec2017-59:simul3} in \eqref{dec2017-59:simul1}
\begin{align}
    \pr{X>2Y\,|\,X>Y} &= \frac{1/2}{2/3}\\
                  &= \frac{3}{4}
\end{align}
$\therefore$ Option 4 is the correct answer.
%
\item  Consider the quadratic equation $x^2+2U x+V=0$ where $U$ and $V$ are chosen independently and randomly from $\{1,2,3\}$ with equal probabilities. Then probability that the equation has both roots real
\begin{enumerate}
    \item $\frac{2}{3}$
    \item $\frac{1}{2}$
    \item $\frac{7}{9}$
    \item $\frac{1}{3}$
\end{enumerate}
%
\solution
Let $U\in\{1.2,3\}$ and $V\in\{1,2,3\}$
\begin{table}[h!]
\centering
\caption{Probability of selecting values for $U$}
\resizebox{\columnwidth}{!}{
  \begin{tabular}{||c|c|c|c||}
    \hline
    $k$ & $1$ & $2$ & $3$\\
    \hline
    \hline
    $\pr{U=k}$ & $1/3$ & $1/3$ & $1/3$\\
    \hline
  \end{tabular}
}
\label{june2013-60:Table1}
\end{table}
\begin{table}[h!]
\centering
\caption{Probability of selecting values for $V$}
\resizebox{\columnwidth}{!}{
  \begin{tabular}{||c|c|c|c||}
    \hline
    $k$ & $1$ & $2$ & $3$\\
    \hline
    \hline
    $\pr{V=k}$ & $1/3$ & $1/3$ & $1/3$\\
    \hline
  \end{tabular}
}
\label{june2013-60:Table2}
\end{table}
For $x^2+2U x+V=0$ to have real roots,
\begin{align}
    b^2-4ac\geq0\\
    \brak{2U}^2-4\brak{1}\brak{V}\geq0\\
    U^2\geq V
\end{align}
\begin{align}
    \pr{U^2\geq V}=1-\pr{U^2<V}
\end{align}
The possible pairs \brak{U,V} for \pr{U^2<V},
\vspace{0.00001in}
\begin{table}[h!]
\centering
\caption{Table for \pr{U^2<V}}
\resizebox{\columnwidth}{!}{
  \begin{tabular}{||c|c||}
    \hline
    $\brak{U,V}$ for $U^2<V$ & Probability\\
    \hline
    \hline
    $\brak{1,2}$ & \pr{U=1}\pr{V=2}=1/9\\
    \hline
    $\brak{1,3}$ & $\pr{U=1}\pr{V=3}=\frac{1}{9}$\\
    \hline
    \hline
    Total & $\pr{U^2<V}=\frac{2}{9}$\\
    \hline
  \end{tabular}
}
\label{june2013-60:Table3}
\end{table}
\begin{align}
    \pr{U^2\geq V}=1-\frac{2}{9}\\
    \pr{U^2\geq V}=\frac{7}{9}
\end{align}
Hence, Option 3 is correct.

%
\newcommand{\tikzAngleOfLine}{\tikz@AngleOfLine}
  \def\tikz@AngleOfLine(#1)(#2)#3{%
  \pgfmathanglebetweenpoints{%
    \pgfpointanchor{#1}{center}}{%
    \pgfpointanchor{#2}{center}}
  \pgfmathsetmacro{#3}{\pgfmathresult}%
  }

\item A point is chosen at random from a circular disc shown below. What is the probability that the point lies in the sector OAB?\\
\begin{tikzpicture}[
    colorstyle/.style={
       circle, draw=black,fill=black,
       thick, inner sep=0pt, minimum size=2 mm,
       outer sep=0pt
        },
    scale=2]
\draw (0,0) circle (2cm);
\node at (0,0) [colorstyle,label=below:O]{};
\node at (1,1.732) [colorstyle,label=above:A]{};
\node at (1.732,1) [colorstyle,label=above right:B]{};
\draw (0,0) -- (1,1.732);
\draw (0,0) -- (1.732,1);
\coordinate (O) at (0,0);
\coordinate (A) at (1,1.732);
\coordinate (B) at (1.732,1);
\tikzAngleOfLine(O)(B){\AngleStart}
    \tikzAngleOfLine(O)(A){\AngleEnd}
    \draw[red,<->] (O)+(\AngleStart:1cm) arc (\AngleStart:\AngleEnd:1 cm);
    \node[circle,fill=green] at ($(O)+({(\AngleStart+\AngleEnd)/2}:1.5 cm)$) {x};
\end{tikzpicture}\\
( where $\angle$AOB = x radians )
\begin{multicols}{2}
    \begin{enumerate}
        \item $\frac{2x}{\pi}$
        \item $\frac{x}{\pi}$
        \item $\frac{x}{2\pi}$
        \item $\frac{x}{4\pi}$
    \end{enumerate}
\end{multicols}
%
\solution
The joint pdf is given by:
\begin{equation}
 \texorpdfstring{f\textsubscript{r$\theta$}}{f r $\theta$}(r,\theta)= \begin{cases}
                        \dfrac{r}{\pi R^2}  & \text{if 0 $<$ r $<$ R , 0 $<$ $\theta$ $<$ 2$\pi$ }\\
                        0  & \text{otherwise}
                        \end{cases}
\end{equation}
Let A $\equiv$ (R,  $\theta _2$) and B $\equiv$ (R,  $\theta _1$). \\
Hence,
\begin{equation}
(\theta _2 - \theta _1)= x    
\end{equation}
We want $\theta$ $\in$ ($\theta _1$ , $\theta _2$) and r $\in$ (0,R) for point to lie in the sector.
Let the point to be chosen be (r, $\theta$).\\
So, Required probability is:
\begin{align}
 \nonumber  \pr{\theta_1<\theta<\theta_2 , 0<r<R}\\
    =& \Int_{\theta_1}^{\theta_2} \Int_{0}^{R} \dfrac{r}{\pi R^2} \,dr\,d\theta \displaybreak \\
    =& \Int_{\theta_1}^{\theta_2} \dfrac{1}{\pi R^2} \dfrac{r^2}{2} \Bigg|_0^R \\
    =& \Int_{\theta_1}^{\theta_2} \dfrac{R^2}{2\pi R^2} \,d\theta   \\
    =& \Int_{\theta_1}^{\theta_2} \dfrac{1}{2\pi} \,d\theta\\
    =& \dfrac{\theta}{2\pi} \Bigg|_{\theta_1}^{\theta_2} \\
    =& \dfrac{\theta_2 - \theta_1}{2\pi} \\
    =& \dfrac{x}{2\pi}
\end{align}
    
$\therefore$The correct answer is \textbf{option (3)} \Large $\frac{x}{2\pi}$.

%
\item Consider a parallel system with two components. The lifetimes of the two components are independent and identically distributed random variables each following an exponential distribution with mean 1. The expected lifetime of the system is:
\begin{enumerate}[label=\Alph*)]
    \item $1$\\[0.5pt]
    \item $\dfrac{1}{2}$\\
    \item $\dfrac{3}{2}$\\
    \item $2$
\end{enumerate}
%
\solution
\input{solutions/2013/june/42/Assignment7.tex}
%
\item Let $X_1$,$X_2$,... be independent random variables each following exponential distribution with mean 1. Then which of the following statements are correct?
\begin{enumerate}
    \item P($X_n > \log n$ for infinitely many $n \geq 1$) = 1
    \item P($X_n > 2$ for infinitely many $n \geq 1$) = 1
    \item P($X_n > \frac{1}{2}$ for infinitely many $n \geq 1$) = 0
    \item P($X_n > \log n, X_{n+1}>\log (n+1)$ for infinitely many $n \geq 1$) = 0
\end{enumerate}
%
\solution
\newcommand{\Integral}[2]{\ensuremath{\int\limits_{#1}^{#2}}}
PDF of $X_i$ is
\begin{align}
    f_{X_i}(x)=\begin{cases}\lambda_i e^{-\lambda_i x}, &x\geq 0\\
                0, &x<0\nonumber
    \end{cases}    
\end{align} 
Mean of $X_i$ is expressed as
\begin{align}
    \mean{X_i}&=\Integral{-\infty}{\infty}x f_{X_i}(x) dx\nonumber\\
              &=\Integral{-\infty}{0}0 dx + \Integral{0}{\infty}x \lambda_i e^{-\lambda_i x}\nonumber\\
              &=\frac{1}{\lambda_i}\label{june/2013/101a}
\end{align}
From \eqref{june/2013/101a}and $\mean{X_i}=1$, we have $\lambda_i=1 \forall  i \geq1$
Now, for some constant $c\geq0$
\begin{align}
    \pr{X_n>c}&=\Integral{c}{\infty}f_{X_n}(x)dx\nonumber\\
              &=\Integral{c}{\infty}e^{-x}dx\nonumber\\
              %&=-e^{-x}\Big|_c^{\infty}\nonumber\\
              &=e^{-c}\label{june/2013/101b}
\end{align}
\textbf{Borel-Cantelli Lemma}:\\
Let $E_1$,$E_2$,... be a sequence of events in some probability space. The Borel–Cantelli lemma states that, if the sum of the probabilities of the events $E_n$ is finite
\begin{align}
    \sum_{n=1}^{\infty}\pr{E_n}&<\infty
\end{align}
then the probability that infinitely many of them occur is 0
\begin{align}
    \pr{\lim_{n \rightarrow \infty}\sup E_n}&=0
\end{align}
\textbf{Second Borel-Cantelli Lemma}:\\
If the events $E_n$ are independent and the sum of the probabilities of the $E_n$ diverges to infinity, then the probability that infinitely many of them occur is 1.
If for independent events $E_1,E_2,...$
\begin{align}
    \sum_{n=1}^{\infty}\pr{E_n}&=\infty
\end{align}
Then
\begin{align}
    \pr{\lim_{n \rightarrow \infty}\sup E_n}&=1
\end{align}
\bigskip
\begin{enumerate}
    \item \textbf{Option 1:} 
    We can say the events $X_n>\log n$ are independent $\forall n\geq 1$ as $X_n$ are independent random variable.
    
    From \eqref{june/2013/101b}
    \begin{align}
        \sum_{n=1}^{\infty}\pr{X_n > \log n} &=\sum_{n=1}^{\infty}e^{-\log n}\nonumber\\ &=\sum_{n=1}^{\infty}\frac{1}{n}\nonumber\\
                                            &= \infty  \text{ (Cauchy's Criterion)}\nonumber
    \end{align}
    Now, from second Borel-Cantelli lemma
    \begin{align}
        &\pr{X_n>\log n \text{ for infinitely many }n\geq1}\nonumber\\
        &=\pr{\lim_{n \rightarrow \infty}\sup X_n>\log n}\nonumber\\
        &=1\nonumber
    \end{align}
    $\therefore$ Option 1 is correct. 
    
    \item\textbf{Option 2:} We can say the events $X_n>2$ are independent $\forall n\geq 1$ as $X_n$ are independent random variable.
    
    From \eqref{june/2013/101b}
    \begin{align}
        \sum_{n=1}^{\infty}\pr{X_n > 2} &= \sum_{n=1}^{\infty}e^{-2}\nonumber\\
                                            &= \infty\nonumber
    \end{align}
    Now, from second Borel-Cantelli lemma
    \begin{align}
        &\pr{X_n>2 \text{ for infinitely many }n\geq1}\nonumber\\
        &=\pr{\lim_{n \rightarrow \infty}\sup X_n>2}\nonumber\\
        &=1\nonumber
    \end{align}
    $\therefore$ Option 2 is correct.
    
    \item \textbf{Option 3:} We can say the events $X_n>\frac{1}{2}$ are independent $\forall n\geq 1$ as $X_n$ are independent random variable.
    
    From \eqref{june/2013/101b}
    \begin{align}
        \sum_{n=1}^{\infty}\pr{X_n > \frac{1}{2}} &= \sum_{n=1}^{\infty}e^{-\frac{1}{2}}\nonumber\\
                                            &= \infty\nonumber
    \end{align}
    Now, from second Borel-Cantelli lemma
    \begin{align}
        &\pr{X_n>\frac{1}{2} \text{ for infinitely many }n\geq1}\nonumber\\
        &=\pr{\lim_{n \rightarrow \infty}\sup X_n>\frac{1}{2}}\nonumber\\
        &=1\nonumber
    \end{align}
    $\therefore$ Option 3 is incorrect.
    \item \textbf{Option 4:} We can say the events $X_n>\log n$ are independent $\forall n\geq 1$ as $X_n$ are independent random variable.
    
    Let the event $X_n > \log n,X_{n+1}>\log (n+1)$ be represented by $E_n$'
    
    From \eqref{june/2013/101b}
    \begin{align}
        &\sum_{n=1}^{\infty}\pr{E_n}\nonumber\\
        &= \sum_{n=1}^{\infty}\pr{X_n>\log n}\pr{X_{n+1}>\log (n+1)}\nonumber\\
        &=\sum_{n=1}^{\infty}e^{-\log n}e^{-\log (n+1)}\nonumber\\
        &=\sum_{n=1}^{\infty}\frac{1}{n(n+1)}\nonumber\\
        &=\sum_{n=1}^{\infty}\frac{1}{n}-\frac{1}{n+1}\nonumber\\
        &=1
    \end{align}
    Now, from Borel-Cantelli lemma
    \begin{align}
        &\pr{E_n\text{ for infinitely many }n\geq1}\nonumber\\
        &=\pr{\lim_{n \rightarrow \infty}\sup ( X_n>\log n,X_{n+1}>\log (n+1))}\nonumber\\
        &=0\nonumber
    \end{align}
    $\therefore$ Option 4 is correct.
\end{enumerate}
\vspace{0.5cm}\centering \boxed{\solution{\text{Options 1, 2, 4}}}
%
\item Suppose $X_1$ and $X_2$ are independent and identically distributed random variables each following an exponential distribution with mean $\theta$, i.e., the common pdf is given by $f_\theta(x) = \frac{1}{\theta}e^{\frac{-x}{\theta}}, 0<x<\infty,0<\theta<\infty.$ Then which of the following is true? Conditional distribution of $X_2$ given $X_1+X_2=t$ is 
\begin{enumerate}
    \item exponential with mean $\frac{t}{2}$ and hence $X_1+X_2$ is sufficient for $\theta$ \label{june/2013/40/option 1}
    \item exponential with mean $\frac{t\theta}{2}$ and hence $X_1+X_2$ is not sufficient for $\theta$ \label{june/2013/40/option 2}
    \item uniform$(0,t)$ and hence $X_1+X_2$ is sufficient for $\theta$ \label{june/2013/40/option 3}
    \item uniform$(0,t\theta)$ and hence $X_1+X_2$ is not sufficient for $\theta$ \label{june/2013/40/option 4}
\end{enumerate}
%
\solution
Let $f_{X_1,X_2}(x_1,x_2)$ denote the joint probability distribution of random variables $X_1$ and $X_2$. Let $Z$ be another random variable such that $Z=X_1+X_2$. Let $\Phi_{X_1}(\omega)$ and $\Phi_{Z}(\omega)$ be the characteristic functions of the probability density functions $f_{X_1}(x)$ and $f_{Z}(x)$ respectively. The conditional probability density function of $X_2$ can be defined by:
\begin{align}
    f_{X_2|(X_1+X_2=t)}(x_2) &= 
    \begin{cases}
    \frac{f_{X_1,X_2}(x_1,x_2)}{f_{(X_1+X_2)}(t)} &  \text{if }x_1+x_2=t\\ ~\\[-1em]
    0 & \text{otherwise}
    \end{cases}
    \\ x_1 + x_2 &= t
    \\ 0 < x_1, x_2&< \infty \label{june/2013/40/equation 2}
    \\ x_1 &= t-x_2
    \\ t - x_2 &> 0
    \\ x_2 &< t \label{june/2013/40/equation 3}
\end{align}
From equations \eqref{june/2013/40/equation 2} and \eqref{june/2013/40/equation 3}, we can conclude that $x_2 \in (0, t)$ if $x_1+x_2=t$. Also, given in the question,
\begin{align}
    0 &< \theta < \infty
    \\f_{X_1}(x_1) &= \frac{1}{\theta}e^{\frac{-x_1}{\theta}}, 0<x_1<\infty
    \\f_{X_2}(x_2) &= \frac{1}{\theta}e^{\frac{-x_2}{\theta}}, 0<x_2<\infty
\end{align}
Since $X_1$ and $X_2$ are independent, 
\begin{align}
f_{X_1,X_2}(x_1,x_2) &= f_{X_1}(x_1) \times f_{X_2}(x_2)
    \\&= \frac{1}{\theta}e^{\frac{-x_1}{\theta}} \times \frac{1}{\theta}e^{\frac{-x_2}{\theta}}
    \\&= \frac{1}{\theta^2}e^{\frac{-(x_1+x_2)}{\theta}}
    \\\Phi_{X_1}(\omega) &= \frac{1}{\theta} \int_{0}^{\infty}e^{i\omega x} e^{\frac{-x}{\theta}} \,dx
    \\ &= \frac{1}{\theta} \times \frac{1}{i\omega - \frac{1}{\theta}} \brak{e^{x(i\omega - \frac{1}{\theta}}}\bigg\vert_0^{\infty}
    \\ &= \frac{1}{1-i\omega\theta} - \frac{\lim_{x\rightarrow \infty} \brak{e^{x(i\omega - \frac{1}{\theta}}}}{1-i\omega\theta}
    \\&= \frac{1}{1-i\omega\theta} - 0 = \frac{1}{1-i\omega\theta} 
    \\ \Phi_{Z}(\omega) &= \brak{\frac{1}{1-i\omega\theta} }^2
    \\ f_Z(x) &= \frac{1}{2\pi} \int_{-\infty}^{\infty}\frac{e^{-i\omega x}}{\brak{\frac{1}{1-i\omega\theta} }^2} \,d\omega \label{june/2013/40/equation 1}
\end{align}
The equation \eqref{june/2013/40/equation 1} is the characteristic function expression of a gamma random variable with k=2. Thus,
\begin{align}
    f_Z(x) &= \frac{x^{k-1}e^{\frac{-x}{\theta}}}{\Gamma(k)\theta^k}
    \\ &=  \frac{x^{2-1}e^{\frac{-x}{\theta}}}{\Gamma(2)\theta^2}
    \\ &= \frac{xe^{\frac{-x}{\theta}}}{\theta^2}
\end{align}
\begin{align}
    f_{X_2|(X_1+X_2=t)}(x_2) = 
    \begin{cases}
    \frac{f_{X_1,X_2}(x_1,x_2)}{f_Z(t)} &  x_2 \in (0, t)\\ ~\\[-1em]
    0 & \text{otherwise}
    \end{cases}
\end{align}
Let $ x_2 \in (0, t)$.
\begin{align}
    f_{X_2|(X_1+X_2=t)}(x_2) &= \frac{f_{X_1,X_2}(x_1,x_2)}{f_Z(t)}
    \\&= \frac{\frac{1}{\theta^2}e^{\frac{-(x_1+x_2)}{\theta}}}{\frac{1}{\theta^2}e^{\frac{-t}{\theta}}t}
    \\&= \frac{e^{\frac{-(t)}{\theta}}}{e^{\frac{-t}{\theta}}t}
    \\&= \frac{1}{t} \quad \forall x_2 \in (0, t)
\end{align}
The obtained pdf is uniform$(0,t)$. Any distribution is sufficient for underlying parameter $\theta$ if the conditional probability distribution of the data does not depend on the parameter $\theta$.  And since the conditional distribution of $X_2$ does not depend on $\theta$ for any value of $t$, $X_1+X_2$ is sufficient for $\theta$. Verifying the pdf,
\begin{align}
    \text{total probability} &= \int_{0}^{t} f_{X_2|(X_1+X_2=t)}(x_2) \,dx_2
    \\&= \int_{0}^{t} \frac{1}{t} \,dx_2
    \\&= 1
\end{align}
Hence, the correct answer is option \eqref{june/2013/40/option 3}
%
\item Let $X_{1},X_{2},\dots$ be independent and identically distributed random variables each following a uniform distribution on (0,1). Denote $T_{n}=max\{ X_{1},X_{2},\dots,X_{n}\}$. Then, which of the following statements are true?
\begin{enumerate}
    \item $T_{n}$ converges to 1 in probability.
    \item $n(1-T_{n})$ converges in distribution.
    \item $n^{2}(1-T_{n})$ converges in distribution.
    \item $\sqrt{n}(1-T_{n})$ converges to 0 in probability.
\end{enumerate}
%
\solution
The PDF, CDF of each $X_{1},X_{2},X_{3},\dots$ is 
\begin{align}
\tag{72.1}
    f_{X_{i}}(x)=\begin{cases}
	1, & 0< x<1 \\~\\[-1em]
	0, & otherwise
	\end{cases} 
\end{align}
\begin{align}
\tag{72.2}
	F_{X_{i}}(x)=\begin{cases}
	x, & 0< x<1 \\~\\[-1em]
	1, & x\geq 1\\~\\[-1em]
	0, & otherwise
	\end{cases} 
\end{align}
$\forall i\in \mathbb{N}$.
Then, as $T_{n}=max\{ X_{1},X_{2},\dots,X_{n}\}$,
\begin{align}
\tag{72.3}
    f_{T_{n}}(x)=\begin{cases}
	nx^{n-1}, & 0< x<1 \\~\\[-1em]
	0, & otherwise
	\end{cases} \\
\tag{72.4}
	F_{T_{n}}(x)=\begin{cases}
	x^{n}, & 0< x<1 \\~\\[-1em]
	1, & x\geq 1\\~\\[-1em]
	0, & otherwise
	\end{cases} 
\end{align}
NOTE : If $Y=aX+b$ and $a<0$, then
\begin{align}
\tag{72.5}
\label{june/2013/72/eq:form}
    F_{Y}(y)=1-F_{X}\brak{\dfrac{y-b}{a}}
\end{align}
\begin{enumerate}
\item OPTION-1:\\
Convergence in Probability :\\
A sequence of random variables $X_{1},X_{2},X_{3},\dots$ converges in probability to a random variable $X$, shown by $X_{n}\xrightarrow[]{p}X$, if
\begin{align}
\tag{72.6}
    \displaystyle\lim_{n\to\infty}\pr{|X_{n}-X|\geq\epsilon}=0,\forall\epsilon>0
\end{align}
To evaluate : $\displaystyle\lim_{n\to\infty}\pr{|T_{n}-1|\geq\epsilon},\forall\epsilon>0$
\begin{align}
\tag{72.7}
    &\displaystyle\lim_{n\to\infty}\pr{|T_{n}-1|\geq\epsilon}=\displaystyle\lim_{n\to\infty}\pr{1-T_{n}\geq\epsilon}\\
\tag{72.8}
    &=\displaystyle\lim_{n\to\infty}\pr{T_{n}\leq1-\epsilon}=\displaystyle\lim_{n\to\infty}F_{T_{n}}(1-\epsilon)
\end{align}
\begin{align}
\tag{72.9}
    F_{T_{n}}(1-\epsilon)=\begin{cases}
	(1-\epsilon)^{n}, & 0< \epsilon<1 \\~\\[-1em]
	0, & \epsilon\geq 1
	\end{cases}
\end{align}
\begin{align}
\tag{72.10}
    \because\displaystyle\lim_{n\to\infty}(1-\epsilon)^{n}=0 \text{ for } 0< \epsilon<1\\
    \tag{72.11}
    \therefore \displaystyle\lim_{n\to\infty}\pr{|T_{n}-1|\geq\epsilon}=0,\forall\epsilon>0
\end{align}
$\therefore T_{n}$ converges to 1 in probability.
\item OPTION-2:\\
Convergence in Distribution :\\
A sequence of random variables $X_{1},X_{2},X_{3},\dots$ converges in distribution to a random variable $X$, shown by $X_{n}\xrightarrow[]{d}X$, if
\begin{align}
\tag{72.12}
    \displaystyle\lim_{n\to\infty}F_{X_{n}}(x)=F_{X}(x)
\end{align}
for all $x$ at which $F_{X}(x)$ is continuous.\\
To evaluate : $\displaystyle\lim_{n\to\infty}F_{n(1-T_{n})}(x)$\\ 
Substituting $a=-n,b=n$ in \eqref{june/2013/72/eq:form},
\begin{align}
\tag{72.13}
    F_{n(1-T_{n})}(x)=1-F_{T_{n}}\brak{1-\dfrac{x}{n}}
\end{align}
\begin{align}
\tag{72.14}
    F_{T_{n}}\brak{1-\dfrac{x}{n}}=\begin{cases}
	\brak{1-\dfrac{x}{n}}^{n}, & 0< x<n \\~\\[-1em]
	1, & x\leq 0\\~\\[-1em]
	0, & x\geq n
	\end{cases} 
\end{align}
\begin{align}
\tag{72.15}
    \because\displaystyle\lim_{n\to\infty}\brak{1-\dfrac{y}{n}}^{n}=e^{-y}
\end{align}
\begin{align}
\tag{72.16}
\label{june/2013/72/eq:bcdf}
    \therefore\displaystyle\lim_{n\to\infty} F_{T_{n}}\brak{1-\dfrac{x}{n}}=\begin{cases}
	e^{-x}, & 0< x<n \\~\\[-1em]
	1, & x\leq 0\\~\\[-1em]
	0, & x\geq n
	\end{cases} 
\end{align}
\begin{align}
\tag{72.17}
\label{june/2013/72/eq:cdf}
    \therefore F_{n(1-T_{n})}(x)=\begin{cases}
	1-e^{-x}, & 0< x<n \\~\\[-1em]
	0, & x\leq 0\\~\\[-1em]
	1, & x\geq n
	\end{cases} 
\end{align}
\begin{figure}[h!]
\centering
\includegraphics[width=\columnwidth]{solutions/2013/june/72/figures/Assignment9}
\caption{CDF}
\label{june/2013/72/plot}
\end{figure}
$\therefore n(1-T_{n})$ converges in distribution to a random variable with CDF in \eqref{june/2013/72/eq:cdf}.
\item OPTION-3:\\
Convergence in Distribution :\\
A sequence of random variables $X_{1},X_{2},X_{3},\dots$ converges in distribution to a random variable $X$, shown by $X_{n}\xrightarrow[]{d}X$, if
\begin{align}
\tag{72.18}
    \displaystyle\lim_{n\to\infty}F_{X_{n}}(x)=F_{X}(x)
\end{align}
for all $x$ at which $F_{X}(x)$ is continuous.\\
To evaluate : $\displaystyle\lim_{n\to\infty}F_{n^{2}(1-T_{n})}(x)$\\ 
Substituting $a=-n^{2},b=n^{2}$ in \eqref{june/2013/72/eq:form},
\begin{align}
\tag{72.19}
    F_{n^{2}(1-T_{n})}(x)=1-F_{T_{n}}\brak{1-\dfrac{x}{n^{2}}}
\end{align}
\begin{align}
\tag{72.20}
    F_{T_{n}}\brak{1-\dfrac{x}{n^{2}}}=\begin{cases}
	\brak{1-\dfrac{x}{n^{2}}}^{n}, & 0< x<n^{2} \\~\\[-1em]
	1, & x\leq 0\\~\\[-1em]
	0, & x\geq n^{2}
	\end{cases} 
\end{align}
\begin{align}
\tag{72.21}
    \because\displaystyle\lim_{n\to\infty}\brak{1-\dfrac{y}{n^{2}}}^{n}\text{ is not defined}
\end{align}
$\therefore n^{2}(1-T_{n})$ does not converge in distribution.
\item OPTION-4:\\
Convergence in Probability :\\
A sequence of random variables $X_{1},X_{2},X_{3},\dots$ converges in probability to a random variable $X$, shown by $X_{n}\xrightarrow[]{p}X$, if
\begin{align}
\tag{72.22}
    \displaystyle\lim_{n\to\infty}\pr{|X_{n}-X|\geq\epsilon}=0,\forall\epsilon>0
\end{align}
To evaluate :\\ $\displaystyle\lim_{n\to\infty}\pr{|\sqrt{n}(1-T_{n})-0|\geq\epsilon},\forall\epsilon>0$
\begin{align}
\tag{72.23}
    =\displaystyle\lim_{n\to\infty}\pr{1-T_{n}\geq\dfrac{\epsilon}{\sqrt{n}}}\\
\tag{72.24}
    =\displaystyle\lim_{n\to\infty}\pr{T_{n}\leq1-\dfrac{\epsilon}{\sqrt{n}}}\\
\tag{72.25}
    =\displaystyle\lim_{n\to\infty}F_{T_{n}}\brak{ 1-\dfrac{\epsilon}{\sqrt{n}}}
\end{align}
\begin{align}
\tag{72.26}
    F_{T_{n}}\brak{1-\dfrac{\epsilon}{\sqrt{n}}}=\begin{cases}
	\brak{1-\dfrac{\epsilon}{\sqrt{n}}}^{n}, & 0< \epsilon< \sqrt{n}\\~\\[-1em]
	0, & \epsilon\geq \sqrt{n}
	\end{cases}
\end{align}
\begin{align}
\tag{72.27}
    \because\displaystyle\lim_{n\to\infty}\brak{1-\dfrac{\epsilon}{\sqrt{n}}}^{n}=0 \text{ for } 0< \epsilon<\sqrt{n}\\
    \tag{72.28}
    \therefore \displaystyle\lim_{n\to\infty}\pr{|\sqrt{n}(1-T_{n})-0|\geq\epsilon}=0,\forall\epsilon>0
\end{align}
$\therefore\sqrt{n}(1-T_{n})$ converges to 0 in probability.
\end{enumerate}
\begin{lstlisting}
Hence, options 1), 2), 4) are correct.
\end{lstlisting}

\end{enumerate}
