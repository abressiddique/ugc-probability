\begin{definition}
    For a given statistical sample $\{X_1, X_2,\cdots X_n\}$, the order statistics is obtained by sorting the
    sample in ascending order. It denoted as $\{X_{(1)}, X_{(2)},\cdots X_{(n)}\}$. The $k^{th}$ smallest value
    $X_{(k)}$ is called  $k^{th}$ order statistic 
    \end{definition}
    \begin{theorem}
    Let $\{X_1, X_2, \cdots X_n\}$ be n i.i.d random variables with common CDF $= F(x)$ and common PDF $= f(x)$, 
    then the marginal probability distribution of $k^{th}$ order statistic (CDF) is denoted by $F_{(k,n)}(x)$ 
    and is given by
    \begin{align}
    F_{(k,n)}(x) =  \sum_{j=k}^{n}\comb{n}{j} \brak{F(x)}^{j}\brak{1-F(x)}^{n-j} \label{stats/5/eqn_1}
    \end{align}
    \label{stats/5/th1}
    \end{theorem}
    \begin{proof}
    %    \begin{align}
        % F_{(k,n)}(x) &= \pr{X_{(k)}\leq x} \\
        % F_{(k,n)}(x) &= \pr{\text{At least k elements have value $\leq x$}} \label{kcdf}
        % \end{align}
        % Since $\pr{X\leq x} = F(x)$ , 
    %     Let $ Q \sim Bern(F(x))$ such that
    %     \begin{align}
    % \pr_Q{q} = \begin{cases}
    %  F(x)  & q = 1\\
    %      1-F(x) & q = 0
    % \end{cases}
%        \end{align}
        Let $P \sim B(n,F(x))$.  Then
        % \begin{align}
        % \pr{P=i} &= \comb{n}{i}\,\pr{Q=1}^{i}\,\pr{Q=0}^{n-i} \\
        % \pr{P=i} &= \comb{n}{i}\,F(x)^{i}\,(1-F(x))^{n-i}\label{bin}  
        % \end{align}
        % Equation \eqref{bin} is probability of exactly $i$ R.V of given sample have values $\leq x$   
        \begin{align}
         F_{(k,n)}(x) &= \pr{P\geq k} \\
         &=  \sum_{j=k}^{n}\comb{n}{j}\, \brak{F(x)}^{j}\,\brak{1-F(x)}^{n-j} 
        \end{align}
        \end{proof}
    \begin{corollary}
    The marginal probability density of $k^{th}$ order statistic (PDF) is denoted by $f_{(k,n)}(x)$  and 
     given by
    \begin{align}
    f_{(k,n)}(x) = n\;\comb{n-1}{k-1}\,f(x)\,(F(x))^{k-1}\,\brak{1-F(x)}^{n-k} \label{stats/5/eqn_2}
    \end{align}
    \label{stats/5/th2}
    \end{corollary}
%
\begin{definition}[Beta function]
    \label{stats/5/def2}
    % The \textbf{Beta function} is defined for $r, s \in \R^{+}$ by 
    \begin{equation}
    B(r,s) = \Int_{0}^{1} x^{r-1}\,\brak{1-x}^{s-1}\,dx = \dfrac{\Gamma(r)\,\Gamma(s)}{\Gamma(r+s)} 
    \label{stats/5/eqn_5}
    \end{equation}
    \end{definition}
    \begin{definition}[Beta Distribution]
    \label{stats/5/bddef}
    The Beta distribution is a continuous distribution defined on the range $(0,1)$ whose PDF given by 
    \begin{align}
    f(x) &= \dfrac{1}{B(r,s)}\,x^{r-1}\,\brak{1-x}^{s-1} 
    \end{align}
    where $\Int_{0}^1f(x) = 1$ as per definition \eqref{stats/5/def2} 
    CDF, Mean value and Variance of Beta distribution
    \begin{align}
     F(x)   &=\dfrac{\Int_{0}^{x}x^{r-1}\,\brak{1-x}^{s-1}}{B(r,s)} =    \dfrac{B_{x}(r,s)}{B(r,s)} \\
     E(x)   &=   \dfrac{r}{r+s} \\
     \var(x) &= \dfrac{rs}{(r+s)^{2}\,(r+s+1)}
    \end{align}
    \end{definition}
    \begin{definition}[Uniform Order Statistics]
    Let $\{X_1,\cdots X_n\}$ be i.i.d form a uniform distribution on $[0,1]$ such that $f_{X}(x) = 1$ and $F_{X}(x)=x$.
    From Theorem \eqref{stats/5/th2}, equation \eqref{stats/5/eqn_2}
    \begin{align}
    f_{(k,n)}(x) &= n\;\comb{n-1}{k-1}\,x^{k-1}\,\brak{1-x}^{n-k}\label{stats/5/eqn_3}
    \end{align}
    \end{definition}
    \begin{lemma}
    \label{stats/5/lma}
    Uniform order statistics on [0,1] the PDF of $k^{th}$ order statistic follows Beta distribution with $r=k$, $s=n-k+1$
    and  PDF is given by 
    \begin{align}
    f_{(k,n)}(x) &= \dfrac{1}{B(k,n-k+1)}\,x^{k-1}\,\brak{1-x}^{(n-k+1)-1} 
    \end{align}
    \end{lemma}
    \newpage
    \begin{proof}
    Since  \eqref{stats/5/eqn_3} is the PDF,
    \begin{equation}
     \Int_{0}^{1} n\;\comb{n-1}{k-1}\,x^{k-1}\,\brak{1-x}^{n-k} \,dx  = 1    
    \end{equation}
    \begin{align}
    \Int_{0}^{1} x^{k-1}\brak{1-x}^{n-k} \,dx  &= \dfrac{(k-1)!\,(n-k)!}{n!}  \\
    \Int_{0}^{1} x^{k-1}\brak{1-x}^{n-k} \,dx  &= \dfrac{\Gamma(k)\,\Gamma(n-k+1)}{\Gamma\brak{k+(n-k+1))}} \\
    \Int_{0}^{1} x^{k-1}\brak{1-x}^{n-k} \,dx  &= B(k,n-k+1) \\
    \Int_{0}^{1} \dfrac{x^{k-1}\brak{1-x}^{(n-k+1)-1}}{B(k,n-k+1)}\,dx  &= 1
    \end{align}
    from definition \eqref{stats/5/bddef} with $r=k$ and $s=n-k+1$ equation \eqref{stats/5/eqn_3} follows beta distribution
    \end{proof}
    From lemma \eqref{stats/5/lma}, PDF of $k^{th}$ order statistic of a uniform distribution on $[0,1]$ follows 
    beta distribution
    \begin{align}
    \Int_{0}^{2}f_{(k,8)}(x)\,dx &= \Int_{0}^{2} \dfrac{7}{32}\,x^{6}\,\brak{2-x}\,dx \\
    \Int_{0}^{2}f_{(k,8)}(x)\,dx &= \Int_{0}^{2} 56\,\brak{\dfrac{x}{2}}^{6}\,\brak{1-\dfrac{x}{2}}\,d\brak{\dfrac{x}{2}}
    \end{align}
    Let new random variable be $t$ such that $t=x/2$, New sample be $\{T_1,\cdots T_8\}$ such that $T_{i}=X_{i}/2$.
    \begin{align}
    f_{(k,8)}(t) &= 56\,t^{6}\,\brak{1-t} \\
    \Int_{0}^{2}f_{(k,8)}(x)\,dx &=  \Int_{0}^{1}f_{(k,8)}(t)\,dt = 1
    \end{align}
    The Uniform distribution of new random sample is on $[0,1]$ such that   PDF $= 1$ and CDF $= t$ 
    $f(k,8)(x)$ 
    %in equation \eqref{stats/5/qnpdf}  (after conversion)
    \begin{align}
    f_{(k,8)}(t) =
      \begin{cases}
          56\,t^{6}\,\brak{1-t},  & 0<t<1, \\ 
          \hspace{1cm}   0,               & \text{otherwise,} 
      \end{cases}
      \label{stats/5/pdf_t}
    \end{align}
    Since equation \eqref{stats/5/pdf_t} is a Beta distribution with $r=k$, $s=n-k+1$  
    \begin{align}
    r-1 = k-1 &= 6 \\
    \therefore k &= 7 
    \end{align}
    Hence the \textbf{value of k is 7}


    