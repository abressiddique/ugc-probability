\begin{theorem}[Uniform distribution]
    A random variable $X$ is said to be uniformly distributed in $a\leq x\leq b$ if its density function is
    \begin{align}
        f(x)=
        \begin{cases}
        \frac{1}{b-a} & \text{if } a\leq x \leq b\\
        0 & \text{otherwise}
        \end{cases}\label{stats/8/eq:1}
    \end{align}
    and the distribution is called uniform distribution.
    The mean and variance are respectively,
    \begin{align}
        \mu=\frac{a+b}{2}\label{stats/8/eq:2}
    \end{align}
    \begin{align}
         \sigma^2=\frac{(b-a)^2}{12}\label{stats/8/eq:3}
    \end{align}
    \label{stats/8/theorem}
    \end{theorem}
    \begin{theorem}[Beta distribution]
    The Beta distribution is a continuous distribution defined on the range
    (0, 1) where 
    %the parameters are given by\\
    %If $X\sim B(r,s)$, where $B(r,s)$ is a beta function
    \begin{align}
      \label{stats/8/eq:4}  f_X(x)&=\frac{1}{B(r,s)}x^{r-1}(1-x)^{s-1}\\ 
        \label{stats/8/eq:5}F_X(x)&=\int_{0}^{x}\frac{1}{B(r,s)}x^{r-1}(1-x)^{s-1}dx=\frac{B_x(r,s)}{B(r,s)}\\ 
      \label{stats/8/eq:6}  B(r,s)&=\int_{0}^{1}x^{r-1}(1-x)^{s-1}dx=\frac{(r-1)!(s-1)!}{(r+s-1)!}\\ 
         \label{stats/8/eq:7}B_x(r,s)&=\int_{0}^{x}x^{r-1}(1-x)^{s-1}dx\\
       \label{stats/8/eq:8} E(X)&=\frac{r}{r+s}\\ 
        \label{stats/8/var} Var(X)&=\frac{rs}{(r+s)^{2}(r+s+1)} 
    \end{align}
    and $B(r,s)$ is the beta function. 
    \end{theorem}
    \begin{definition}[Order statistics]
    For a given statistical sample $\{X_1, X_2,\cdots X_n\}$, the order statistics is obtained by sorting the sample in ascending order. It denoted as $\{X_{(1)}, X_{(2)},\cdots X_{(n)}\}$.
    \end{definition}
    \begin{definition}[Median of order statistics]
    Median is defined as the middle number of a sorted sample. It is denoted by M and defined using order statistics of a sample as
    \begin{align}
      M =
      \begin{cases}
       X_{((n+1)/2)},                                           &\text{if $n$ is odd,} \\ \\
      \dfrac{ X_{(n/2)} + X_{(n/2+1)}}{2} ,                     &\text{if $n$ is even,} 
      \end{cases}
    \end{align}
    \label{stats/8/median}\label{stats/8/def2}
    \end{definition}
    \begin{remark}
    The order statistics of the uniform distribution on the unit interval have marginal distributions belonging to the Beta distribution family.
    \begin{align}
    X_{(k)} \sim B(k,n+1-k)
    \end{align}\label{stats/8/rem}
    \end{remark}
    \begin{enumerate}
     \item 
    From definition \eqref{stats/8/median} median $M$ is given by
    \begin{align}
      M &= X_{((5+1)/2)}\\
      &=X_{(3)}\label{stats/8/eq:m}
    \end{align}
    From remark \eqref{stats/8/rem} 
    \begin{align}
    X_{(3)} \sim B(3,3)
    \end{align}
    From \eqref{stats/8/eq:6}
    \begin{align}
        B(3,3)&=\frac{(3-1)!(3-1)!}{(3+3-1)!}=\frac{1}{30}
     \end{align}
     From \eqref{stats/8/eq:4}
     \begin{align}
         f(x)=30x^{2}(1-x)^{2}
     \end{align}
     From \eqref{stats/8/eq:5}
     \begin{align}
         F(x)&=\int_{0}^{x}30x^{2}(1-x)^{2}dx\\
         &=30x^{3}\brak{\frac{1}{3}+\frac{x^2}{5}-\frac{x}{2}}
     \end{align}
     \begin{align}
         \pr{M<\frac{1}{3}}&=F\brak{\frac{1}{3}}=0.20987\\
         \pr{M>\frac{2}{3}}&=F(1)-F\brak{\frac{2}{3}}=0.20987
     \end{align}
     \begin{align}
          \therefore \pr{M<\frac{1}{3}}=\pr{M>\frac{2}{3}}
     \end{align}
      Hence \textbf{Option 1 is true.}
        
        
        
        
     \item From \eqref{stats/8/eq:m}, median M is a third order statistic. Clearly from remark \eqref{stats/8/rem}, $M$ is not an uniform distribution.
    \\Hence \textbf{Option 2 is false.}
    \item From \eqref{stats/8/eq:1}
    \begin{align}
        f(x)=
        \begin{cases}
        1 & \text{if } 0\leq x \leq 1\\
        0 & \text{otherwise}
        \end{cases}
    \end{align}
    From \eqref{stats/8/eq:2} 
    \begin{align}
        E(X_1)&=\frac{1}{2}
    \end{align}
    From \eqref{stats/8/eq:8}
    \begin{align}
        E(M)=\frac{3}{3+3}=\frac{1}{2}
    \end{align}
     \begin{align}
          \therefore E(M)=E(X_1)
     \end{align}
       Hence \textbf{Option 3 is true.}
    \item From \eqref{stats/8/eq:3}
    \begin{align}
          V(X_1)&=\frac{1}{12}
    \end{align}
    From \eqref{stats/8/var}
    \begin{align}
        V(M)=\frac{1}{28}
    \end{align}
    \begin{align}
          \therefore V(M)\neq V(X_1)
     \end{align}
       Hence \textbf{Option 4 is false.}
    \end{enumerate}